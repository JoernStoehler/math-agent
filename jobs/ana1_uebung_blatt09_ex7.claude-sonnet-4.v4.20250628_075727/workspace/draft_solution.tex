\documentclass{article}
\usepackage[utf8]{inputenc}
\usepackage{amsmath}
\usepackage{amssymb}
\usepackage{enumitem}
\usepackage{tikz}

% Commands from exercise
\newcommand{\im}{\mathrm{i}}
\newcommand{\C}{\mathbb{C}}
\DeclareMathOperator{\Im}{Im}
\DeclareMathOperator{\Re}{Re}
\newcommand{\abs}[1]{\left|#1\right|}
\newcommand{\conj}[1]{\overline{#1}}
\newcommand{\Set}[1]{\left\{#1\right\}}

\begin{document}

\subsection*{Aufgabe}
\begin{exercise}[Komplexe Zahlenmengen zeichnen]
  Skizziere die folgenden Mengen in der komplexen Zahlenebene:
  \begin{enumerate}[label=(\alph*)]
  \item $\Set{ z \in \C | -3 \leq \Im(z+5-3\im) \leq 2 }$
  \item $\Set{ z \in \C | \abs{z + 2 - \im} \geq 3 }$
  \item $\Set{ z \in \C | z \conj{z} - (z + \conj{z} )^2 \leq 1 }$
  \end{enumerate}
\end{exercise}

\subsection*{Lösung}

\textbf{(a)} Wir betrachten die Menge $\Set{ z \in \C | -3 \leq \Im(z+5-3\im) \leq 2 }$.

Sei $z = x + y\im$ mit $x, y \in \mathbb{R}$. Dann gilt:
\begin{align}
z + 5 - 3\im &= (x + y\im) + 5 - 3\im \\
&= (x + 5) + (y - 3)\im
\end{align}

Der Imaginärteil von $z + 5 - 3\im$ ist also $y - 3$. Die Bedingung lautet somit:
$$-3 \leq y - 3 \leq 2$$

Wir addieren 3 auf allen Seiten:
$$0 \leq y \leq 5$$

Die gesuchte Menge ist also ein horizontaler Streifen in der komplexen Ebene zwischen den Geraden $y = 0$ und $y = 5$ (einschließlich der Randgeraden).

\textbf{(b)} Wir betrachten die Menge $\Set{ z \in \C | \abs{z + 2 - \im} \geq 3 }$.

Sei wieder $z = x + y\im$. Dann ist:
\begin{align}
z + 2 - \im &= (x + y\im) + 2 - \im \\
&= (x + 2) + (y - 1)\im
\end{align}

Der Betrag ist:
$$\abs{z + 2 - \im} = \abs{(x + 2) + (y - 1)\im} = \sqrt{(x + 2)^2 + (y - 1)^2}$$

Die Bedingung $\abs{z + 2 - \im} \geq 3$ bedeutet:
$$\sqrt{(x + 2)^2 + (y - 1)^2} \geq 3$$

Dies ist äquivalent zu:
$$(x + 2)^2 + (y - 1)^2 \geq 9$$

Dies beschreibt das Äußere (einschließlich Rand) eines Kreises mit Mittelpunkt $(-2, 1)$ und Radius $3$.

\textbf{(c)} Wir betrachten die Menge $\Set{ z \in \C | z \conj{z} - (z + \conj{z} )^2 \leq 1 }$.

Sei $z = x + y\im$. Dann ist $\conj{z} = x - y\im$. Wir berechnen:
\begin{align}
z \conj{z} &= (x + y\im)(x - y\im) = x^2 + y^2 \\
z + \conj{z} &= (x + y\im) + (x - y\im) = 2x \\
(z + \conj{z})^2 &= (2x)^2 = 4x^2
\end{align}

Die Ungleichung wird zu:
$$x^2 + y^2 - 4x^2 \leq 1$$
$$-3x^2 + y^2 \leq 1$$
$$y^2 - 3x^2 \leq 1$$

Dies kann umgeschrieben werden als:
$$\frac{y^2}{1} - \frac{x^2}{1/3} \leq 1$$

Dies ist eine Hyperbel mit der Gleichung $\frac{y^2}{1} - \frac{x^2}{1/3} = 1$ für den Rand. Die Ungleichung beschreibt den Bereich zwischen den beiden Hyperbelästen.

\end{document}