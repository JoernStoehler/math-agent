\documentclass{article}
\usepackage[utf8]{inputenc}
\usepackage{amsmath}
\usepackage{amssymb}

% Commands from exercise
\newcommand{\im}{\mathrm{i}}
\newcommand{\C}{\mathbb{C}}
% \Im and \Re are already defined by amsmath
\newcommand{\abs}[1]{\left|#1\right|}
\newcommand{\conj}[1]{\overline{#1}}
\newcommand{\Set}[1]{\left\{#1\right\}}

\begin{document}

\subsection*{Aufgabe}
\textbf{Komplexe Zahlenmengen zeichnen}

Skizziere die folgenden Mengen in der komplexen Zahlenebene:
\begin{enumerate}
  \item[(a)] $\Set{ z \in \C | -3 \leq \Im(z+5-3\im) \leq 2 }$
  \item[(b)] $\Set{ z \in \C | \abs{z + 2 - \im} \geq 3 }$
  \item[(c)] $\Set{ z \in \C | z \conj{z} - (z + \conj{z} )^2 \leq 1 }$
\end{enumerate}

\subsection*{Lösung}

\textbf{(a)} Wir betrachten die Menge $M_a = \Set{ z \in \C | -3 \leq \Im(z+5-3\im) \leq 2 }$.

Sei $z = x + y\im$ mit $x, y \in \mathbb{R}$. Dann berechnen wir schrittweise:
\begin{align}
z + 5 - 3\im &= (x + y\im) + 5 - 3\im \\
&= (x + 5) + (y - 3)\im
\end{align}

Der Imaginärteil von $z + 5 - 3\im$ ist also $\Im(z + 5 - 3\im) = y - 3$. 

Die Bedingung $-3 \leq \Im(z+5-3\im) \leq 2$ wird somit zu:
$$-3 \leq y - 3 \leq 2$$

Wir addieren 3 auf allen Seiten der Ungleichung:
$$-3 + 3 \leq y - 3 + 3 \leq 2 + 3$$
$$0 \leq y \leq 5$$

Die gesuchte Menge $M_a$ ist also ein horizontaler Streifen in der komplexen Ebene zwischen den Geraden $y = 0$ (reelle Achse) und $y = 5$, wobei beide Randgeraden zur Menge gehören.

\textbf{Skizze:} Ein horizontaler Streifen zwischen der reellen Achse und der Geraden $\Im(z) = 5$.

\textbf{(b)} Wir betrachten die Menge $M_b = \Set{ z \in \C | \abs{z + 2 - \im} \geq 3 }$.

Sei wieder $z = x + y\im$. Dann ist:
\begin{align}
z + 2 - \im &= (x + y\im) + 2 - \im \\
&= (x + 2) + (y - 1)\im
\end{align}

Der Betrag dieser komplexen Zahl ist:
$$\abs{z + 2 - \im} = \abs{(x + 2) + (y - 1)\im} = \sqrt{(x + 2)^2 + (y - 1)^2}$$

Die Bedingung $\abs{z + 2 - \im} \geq 3$ bedeutet:
$$\sqrt{(x + 2)^2 + (y - 1)^2} \geq 3$$

Durch Quadrieren beider Seiten (erlaubt, da beide Seiten nicht-negativ sind) erhalten wir:
$$(x + 2)^2 + (y - 1)^2 \geq 9$$

Dies beschreibt das Äußere (einschließlich Rand) eines Kreises mit Mittelpunkt $z_0 = -2 + \im$ (also dem Punkt $(-2, 1)$ in der komplexen Ebene) und Radius $r = 3$.

\textbf{Skizze:} Das Äußere eines Kreises mit Mittelpunkt bei $-2 + \im$ und Radius $3$ (einschließlich des Kreisrandes).

\textbf{(c)} Wir betrachten die Menge $M_c = \Set{ z \in \C | z \conj{z} - (z + \conj{z} )^2 \leq 1 }$.

Sei $z = x + y\im$. Dann ist $\conj{z} = x - y\im$. Wir berechnen die einzelnen Terme:

Zuerst $z \conj{z}$:
\begin{align}
z \conj{z} &= (x + y\im)(x - y\im) \\
&= x^2 - xy\im + xy\im - y^2\im^2 \\
&= x^2 - y^2(-1) \\
&= x^2 + y^2
\end{align}

Dann $z + \conj{z}$:
\begin{align}
z + \conj{z} &= (x + y\im) + (x - y\im) \\
&= 2x
\end{align}

Somit ist:
$$(z + \conj{z})^2 = (2x)^2 = 4x^2$$

Die Ungleichung $z \conj{z} - (z + \conj{z} )^2 \leq 1$ wird zu:
$$x^2 + y^2 - 4x^2 \leq 1$$

Wir vereinfachen schrittweise:
$$x^2 + y^2 - 4x^2 \leq 1$$
$$-3x^2 + y^2 \leq 1$$
$$y^2 - 3x^2 \leq 1$$

Dies können wir umschreiben als:
$$\frac{y^2}{1} - \frac{x^2}{1/3} \leq 1$$

Der Rand dieser Menge (für Gleichheit) ist eine Hyperbel mit der Gleichung:
$$y^2 - 3x^2 = 1$$

Diese Hyperbel hat ihre Scheitelpunkte bei $(0, \pm 1)$ und öffnet sich in $y$-Richtung. Die Asymptoten haben die Steigung $\pm\sqrt{3}$, also die Gleichungen $y = \pm\sqrt{3}x$.

Die Ungleichung $y^2 - 3x^2 \leq 1$ beschreibt den Bereich zwischen den beiden Hyperbelästen (einschließlich der Hyperbel selbst).

\textbf{Skizze:} Der Bereich zwischen den beiden Ästen einer Hyperbel mit Scheitelpunkten bei $(0, \pm 1)$ und Asymptoten $y = \pm\sqrt{3}x$.

\end{document}