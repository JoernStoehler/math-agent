\documentclass{article}
\usepackage[utf8]{inputenc}
\usepackage{amsmath}
\usepackage{amssymb}
\usepackage{enumitem}

\begin{document}

\section*{Problem}

Untersuche die folgenden Reihen auf Konvergenz:

\begin{enumerate}[label=(\alph*)]

\item $\displaystyle \sum_{n=1}^\infty \frac{2^n n!}{n^n}$

\item $\displaystyle \sum_{n=1}^\infty \frac{3^n n!}{n^n}$

\item $\displaystyle \sum_{n=1}^\infty \frac{1 + (-1)^n n}{n^2}$

\item $\displaystyle \sum_{n=0}^\infty \frac{x^n}{1+x^{2n}} \text{ für } x \in \mathbb{R}$

\end{enumerate}

\section*{Lösung}

\begin{enumerate}[label=(\alph*)]

\item Für die Reihe $\displaystyle \sum_{n=1}^\infty \frac{2^n n!}{n^n}$ verwenden wir das Quotientenkriterium.

Sei $a_n = \frac{2^n n!}{n^n}$. Dann gilt:

\begin{align}
\frac{a_{n+1}}{a_n} &= \frac{2^{n+1} (n+1)!}{(n+1)^{n+1}} \cdot \frac{n^n}{2^n n!}\\
&= \frac{2 \cdot (n+1) \cdot n!}{(n+1)^{n+1}} \cdot \frac{n^n}{n!}\\
&= \frac{2 \cdot (n+1) \cdot n^n}{(n+1)^{n+1}}\\
&= \frac{2 \cdot (n+1) \cdot n^n}{(n+1) \cdot (n+1)^n}\\
&= \frac{2 \cdot n^n}{(n+1)^n}\\
&= 2 \cdot \left(\frac{n}{n+1}\right)^n\\
&= 2 \cdot \left(1 - \frac{1}{n+1}\right)^n
\end{align}

Für den Grenzwert nutzen wir die bekannte Formel $\lim_{n \to \infty} \left(1 - \frac{1}{n+1}\right)^n = \lim_{n \to \infty} \left(1 - \frac{1}{n+1}\right)^{(n+1) \cdot \frac{n}{n+1}} = e^{-1}$.

Daher gilt:
$$\lim_{n \to \infty} \frac{a_{n+1}}{a_n} = 2 \cdot e^{-1} = \frac{2}{e} \approx 0.736 < 1$$

Nach dem Quotientenkriterium konvergiert die Reihe.

\item Für die Reihe $\displaystyle \sum_{n=1}^\infty \frac{3^n n!}{n^n}$ verwenden wir ebenfalls das Quotientenkriterium.

Sei $b_n = \frac{3^n n!}{n^n}$. Analog zu Teilaufgabe (a) erhalten wir:

\begin{align}
\frac{b_{n+1}}{b_n} &= \frac{3^{n+1} (n+1)!}{(n+1)^{n+1}} \cdot \frac{n^n}{3^n n!}\\
&= 3 \cdot \left(\frac{n}{n+1}\right)^n\\
&= 3 \cdot \left(1 - \frac{1}{n+1}\right)^n
\end{align}

Wie in Teilaufgabe (a) gezeigt, gilt:
$$\lim_{n \to \infty} \frac{b_{n+1}}{b_n} = 3 \cdot e^{-1} = \frac{3}{e} \approx 1.104 > 1$$

Nach dem Quotientenkriterium divergiert die Reihe.

\item Für die Reihe $\displaystyle \sum_{n=1}^\infty \frac{1 + (-1)^n n}{n^2}$ spalten wir die Reihe auf:

$$\sum_{n=1}^\infty \frac{1 + (-1)^n n}{n^2} = \sum_{n=1}^\infty \frac{1}{n^2} + \sum_{n=1}^\infty \frac{(-1)^n n}{n^2} = \sum_{n=1}^\infty \frac{1}{n^2} + \sum_{n=1}^\infty \frac{(-1)^n}{n}$$

Die erste Reihe $\sum_{n=1}^\infty \frac{1}{n^2}$ ist eine konvergente p-Reihe mit $p = 2 > 1$.

Die zweite Reihe $\sum_{n=1}^\infty \frac{(-1)^n}{n}$ ist die alternierende harmonische Reihe. Nach dem Leibniz-Kriterium konvergiert sie, da:
\begin{itemize}
\item $\frac{1}{n} > 0$ für alle $n \geq 1$
\item $\frac{1}{n+1} < \frac{1}{n}$ (die Folge ist monoton fallend)
\item $\lim_{n \to \infty} \frac{1}{n} = 0$
\end{itemize}

Da beide Teilreihen konvergieren, konvergiert auch die ursprüngliche Reihe.

\item Für die Reihe $\displaystyle \sum_{n=0}^\infty \frac{x^n}{1+x^{2n}}$ müssen wir verschiedene Fälle für $x \in \mathbb{R}$ betrachten.

\textbf{Fall 1: $x = 0$}\\
Für $n = 0$ ist der Term $\frac{0^0}{1+0^0} = \frac{1}{2}$ (mit der Konvention $0^0 = 1$).
Für $n \geq 1$ sind alle Terme gleich 0. Die Reihe konvergiert gegen $\frac{1}{2}$.

\textbf{Fall 2: $|x| < 1$}\\
Für große $n$ gilt $x^{2n} \to 0$, also $\frac{x^n}{1+x^{2n}} \approx x^n$. 
Da $\sum_{n=0}^\infty x^n$ eine geometrische Reihe mit $|x| < 1$ ist und konvergiert, konvergiert auch unsere Reihe nach dem Majorantenkriterium.

\textbf{Fall 3: $|x| = 1$}\\
Für $x = 1$: $\frac{1^n}{1+1^{2n}} = \frac{1}{1+1} = \frac{1}{2}$ für alle $n \geq 0$.
Die Reihe wird zu $\sum_{n=0}^\infty \frac{1}{2}$, welche divergiert.

Für $x = -1$: $\frac{(-1)^n}{1+(-1)^{2n}} = \frac{(-1)^n}{1+1} = \frac{(-1)^n}{2}$.
Die Reihe wird zu $\sum_{n=0}^\infty \frac{(-1)^n}{2} = \frac{1}{2} \sum_{n=0}^\infty (-1)^n$, welche divergiert (die Partialsummen alternieren zwischen $\frac{1}{2}$ und 0).

\textbf{Fall 4: $|x| > 1$}\\
Für große $n$ gilt $\frac{x^n}{1+x^{2n}} \approx \frac{x^n}{x^{2n}} = x^{-n} = \frac{1}{x^n}$.
Die Reihe verhält sich also wie $\sum_{n=0}^\infty \frac{1}{x^n} = \sum_{n=0}^\infty \left(\frac{1}{x}\right)^n$.
Dies ist eine geometrische Reihe mit Quotient $\frac{1}{x}$. Da $|x| > 1$ ist $\left|\frac{1}{x}\right| < 1$, und die Reihe konvergiert.

\textbf{Zusammenfassung:}\\
Die Reihe $\sum_{n=0}^\infty \frac{x^n}{1+x^{2n}}$ konvergiert für $x \in \mathbb{R} \setminus \{-1, 1\}$ und divergiert für $x \in \{-1, 1\}$.

\end{enumerate}

\end{document}