\documentclass{article}
\usepackage[utf8]{inputenc}
\usepackage{amsmath}
\usepackage{amssymb}
\usepackage{amsthm}

\begin{document}

% Aufgabe
\subsection*{Aufgabe}
Bestimme mittels Polynomdivision und Partialbruchzerlegung Stammfunktionen der folgenden Funktionen:
$$
(a)\ \frac{x^5}{x-1};\quad
(b)\ \frac{x}{x^3+x^2-x-1};\quad
(c)\ \frac{x}{x^3-x^2+x-1}.
$$

\textit{Vorsicht!} Die Nenner bei (b) und (c) sind verschieden!

\subsection*{Lösung}

\textbf{(a)} Wir bestimmen die Stammfunktion von $\frac{x^5}{x-1}$ mittels Polynomdivision.

Wir führen die Polynomdivision $x^5 : (x-1)$ durch:
\begin{align*}
x^5 &= (x-1) \cdot (x^4 + x^3 + x^2 + x + 1) + 1
\end{align*}

Dies können wir schrittweise nachrechnen:
\begin{align*}
(x-1)(x^4 + x^3 + x^2 + x + 1) &= x^5 + x^4 + x^3 + x^2 + x - x^4 - x^3 - x^2 - x - 1\\
&= x^5 - 1
\end{align*}
Also ist $x^5 = (x-1)(x^4 + x^3 + x^2 + x + 1) + 1$, und damit:
$$\frac{x^5}{x-1} = x^4 + x^3 + x^2 + x + 1 + \frac{1}{x-1}$$

Die Stammfunktion ergibt sich durch Integration:
\begin{align*}
\int \frac{x^5}{x-1} \, dx &= \int \left(x^4 + x^3 + x^2 + x + 1 + \frac{1}{x-1}\right) dx\\
&= \frac{x^5}{5} + \frac{x^4}{4} + \frac{x^3}{3} + \frac{x^2}{2} + x + \ln|x-1| + C
\end{align*}

\textbf{(b)} Wir bestimmen die Stammfunktion von $\frac{x}{x^3+x^2-x-1}$ mittels Partialbruchzerlegung.

Zunächst faktorisieren wir den Nenner:
\begin{align*}
x^3 + x^2 - x - 1 &= x^2(x+1) - 1(x+1)\\
&= (x+1)(x^2-1)\\
&= (x+1)(x-1)(x+1)\\
&= (x-1)(x+1)^2
\end{align*}

Für die Partialbruchzerlegung setzen wir an:
$$\frac{x}{(x-1)(x+1)^2} = \frac{A}{x-1} + \frac{B}{x+1} + \frac{C}{(x+1)^2}$$

Multiplizieren wir beide Seiten mit $(x-1)(x+1)^2$:
$$x = A(x+1)^2 + B(x-1)(x+1) + C(x-1)$$

Durch Einsetzen spezieller Werte:
\begin{itemize}
\item Für $x = 1$: $1 = A \cdot 4 + 0 + 0$, also $A = \frac{1}{4}$
\item Für $x = -1$: $-1 = 0 + 0 + C \cdot (-2)$, also $C = \frac{1}{2}$
\end{itemize}

Durch Koeffizientenvergleich (Koeffizient von $x^2$):
$$0 = A + B$$
Also $B = -A = -\frac{1}{4}$.

Damit erhalten wir:
$$\frac{x}{x^3+x^2-x-1} = \frac{1/4}{x-1} - \frac{1/4}{x+1} + \frac{1/2}{(x+1)^2}$$

Die Stammfunktion ist:
\begin{align*}
\int \frac{x}{x^3+x^2-x-1} \, dx &= \int \left(\frac{1/4}{x-1} - \frac{1/4}{x+1} + \frac{1/2}{(x+1)^2}\right) dx\\
&= \frac{1}{4}\ln|x-1| - \frac{1}{4}\ln|x+1| - \frac{1}{2(x+1)} + C\\
&= \frac{1}{4}\ln\left|\frac{x-1}{x+1}\right| - \frac{1}{2(x+1)} + C
\end{align*}

\textbf{(c)} Wir bestimmen die Stammfunktion von $\frac{x}{x^3-x^2+x-1}$ mittels Partialbruchzerlegung.

Zunächst faktorisieren wir den Nenner:
\begin{align*}
x^3 - x^2 + x - 1 &= x^2(x-1) + 1(x-1)\\
&= (x-1)(x^2+1)
\end{align*}

Da $x^2 + 1$ über den reellen Zahlen irreduzibel ist, setzen wir für die Partialbruchzerlegung an:
$$\frac{x}{(x-1)(x^2+1)} = \frac{A}{x-1} + \frac{Bx + C}{x^2+1}$$

Multiplizieren wir beide Seiten mit $(x-1)(x^2+1)$:
$$x = A(x^2+1) + (Bx + C)(x-1)$$

Für $x = 1$: $1 = A \cdot 2 + 0$, also $A = \frac{1}{2}$

Ausmultiplizieren der rechten Seite:
$$x = Ax^2 + A + Bx^2 - Bx + Cx - C$$

Koeffizientenvergleich:
\begin{itemize}
\item Koeffizient von $x^2$: $0 = A + B = \frac{1}{2} + B$, also $B = -\frac{1}{2}$
\item Koeffizient von $x$: $1 = -B + C = \frac{1}{2} + C$, also $C = \frac{1}{2}$
\item Konstanter Term: $0 = A - C = \frac{1}{2} - \frac{1}{2} = 0$ \checkmark
\end{itemize}

Damit erhalten wir:
$$\frac{x}{x^3-x^2+x-1} = \frac{1/2}{x-1} + \frac{-x/2 + 1/2}{x^2+1} = \frac{1/2}{x-1} + \frac{1-x}{2(x^2+1)}$$

Die Stammfunktion ist:
\begin{align*}
\int \frac{x}{x^3-x^2+x-1} \, dx &= \int \left(\frac{1/2}{x-1} + \frac{1-x}{2(x^2+1)}\right) dx\\
&= \frac{1}{2}\ln|x-1| + \frac{1}{2}\int \frac{1}{x^2+1} \, dx - \frac{1}{2}\int \frac{x}{x^2+1} \, dx\\
&= \frac{1}{2}\ln|x-1| + \frac{1}{2}\arctan(x) - \frac{1}{4}\ln(x^2+1) + C
\end{align*}

\end{document}