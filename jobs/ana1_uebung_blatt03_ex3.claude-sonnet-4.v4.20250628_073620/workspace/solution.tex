\documentclass{article}
\usepackage[utf8]{inputenc}
\usepackage{amsmath}
\usepackage{amssymb}
\usepackage{amsthm}

\begin{document}

% Aufgabe
\subsection*{Aufgabe}
Überprüfe die folgenden Funktionen auf Injektivität, Surjektivität und
Bijektivität:
\begin{enumerate}
\item $f_{1} \colon \mathbb{Z} \to \mathbb{Z}, \: z \mapsto z$
\item $f_{2} \colon \mathbb{Z} \to \mathbb{Q}, \: z \mapsto z$
\item $f_{3}\colon \mathbb{Z} \to \mathbb{N}, \: z \mapsto \binom{n}{z}$ für gegebenes
  $n \in \mathbb{N}$
\item $f_{4}\colon \mathbb{R} \to f_{4}(\mathbb{R}), \: y \mapsto y^{2}$
\end{enumerate}

\subsection*{Lösung}

\textbf{1. Funktion $f_1: \mathbb{Z} \to \mathbb{Z}, z \mapsto z$}

Diese Funktion ist die Identitätsfunktion auf den ganzen Zahlen.

\textit{Injektivität:} Seien $z_1, z_2 \in \mathbb{Z}$ mit $f_1(z_1) = f_1(z_2)$. Dann gilt:
$$f_1(z_1) = f_1(z_2) \Rightarrow z_1 = z_2$$
Also ist $f_1$ injektiv.

\textit{Surjektivität:} Sei $y \in \mathbb{Z}$ beliebig. Wir müssen ein $z \in \mathbb{Z}$ finden mit $f_1(z) = y$. 
Wähle $z = y$. Dann gilt:
$$f_1(z) = f_1(y) = y$$
Also ist $f_1$ surjektiv.

\textit{Bijektivität:} Da $f_1$ sowohl injektiv als auch surjektiv ist, ist $f_1$ bijektiv.

\textbf{2. Funktion $f_2: \mathbb{Z} \to \mathbb{Q}, z \mapsto z$}

Diese Funktion bettet die ganzen Zahlen in die rationalen Zahlen ein.

\textit{Injektivität:} Seien $z_1, z_2 \in \mathbb{Z}$ mit $f_2(z_1) = f_2(z_2)$. Dann gilt:
$$f_2(z_1) = f_2(z_2) \Rightarrow z_1 = z_2$$
(in $\mathbb{Q}$, aber da $z_1, z_2$ ganze Zahlen sind, folgt die Gleichheit auch in $\mathbb{Z}$).
Also ist $f_2$ injektiv.

\textit{Surjektivität:} Die Funktion $f_2$ ist nicht surjektiv. 
Gegenbeispiel: Es gibt kein $z \in \mathbb{Z}$ mit $f_2(z) = \frac{1}{2}$, da $\frac{1}{2} \notin \mathbb{Z}$.

\textit{Bijektivität:} Da $f_2$ nicht surjektiv ist, ist $f_2$ auch nicht bijektiv.

\textbf{3. Funktion $f_3: \mathbb{Z} \to \mathbb{N}, z \mapsto \binom{n}{z}$ für gegebenes $n \in \mathbb{N}$}

Zunächst müssen wir klären, wie der Binomialkoeffizient $\binom{n}{z}$ für $z \in \mathbb{Z}$ definiert ist.
Für $z < 0$ oder $z > n$ ist $\binom{n}{z} = 0$ per Definition.
Für $0 \leq z \leq n$ ist $\binom{n}{z} = \frac{n!}{z!(n-z)!}$.

\textit{Injektivität:} Die Funktion $f_3$ ist im Allgemeinen nicht injektiv.
Gegenbeispiel: Für $n \geq 1$ gilt $f_3(-1) = \binom{n}{-1} = 0$ und $f_3(n+1) = \binom{n}{n+1} = 0$.
Da $-1 \neq n+1$ aber $f_3(-1) = f_3(n+1)$, ist $f_3$ nicht injektiv.

\textit{Surjektivität:} Die Funktion $f_3$ ist im Allgemeinen nicht surjektiv.
Für $n = 0$ gilt $f_3(z) = \binom{0}{z} = 1$ für $z = 0$ und $f_3(z) = 0$ für $z \neq 0$.
Also nimmt $f_3$ nur die Werte 0 und 1 an, aber nicht alle natürlichen Zahlen.
Selbst für größere $n$ werden nicht alle natürlichen Zahlen erreicht.

\textit{Bijektivität:} Da $f_3$ weder injektiv noch surjektiv ist, ist $f_3$ nicht bijektiv.

\textbf{4. Funktion $f_4: \mathbb{R} \to f_4(\mathbb{R}), y \mapsto y^2$}

Zunächst bestimmen wir $f_4(\mathbb{R})$: Da $y^2 \geq 0$ für alle $y \in \mathbb{R}$, ist $f_4(\mathbb{R}) = \mathbb{R}_{\geq 0} = [0, \infty)$.

\textit{Injektivität:} Die Funktion $f_4$ ist nicht injektiv.
Gegenbeispiel: $f_4(2) = 4 = f_4(-2)$, aber $2 \neq -2$.

\textit{Surjektivität:} Per Definition des Zielbereichs als $f_4(\mathbb{R})$ ist $f_4$ surjektiv.
Für jedes $z \in f_4(\mathbb{R}) = [0, \infty)$ existiert ein $y \in \mathbb{R}$ mit $f_4(y) = z$, 
nämlich $y = \sqrt{z}$ (oder $y = -\sqrt{z}$).

\textit{Bijektivität:} Da $f_4$ nicht injektiv ist, ist $f_4$ nicht bijektiv.

\end{document}