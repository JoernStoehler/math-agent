\documentclass[11pt]{article}
\usepackage[utf8]{inputenc}
\usepackage{amsmath,amssymb,amsthm}
\usepackage{enumitem}

\newtheorem*{satz}{Satz}

\begin{document}

\section*{Lösung}

Wir untersuchen die gegebenen Reihen auf Konvergenz.

\begin{enumerate}[label=(\alph*)]

\item $\displaystyle \sum_{n=1}^\infty \frac{2^n n!}{n^n}$

Wir verwenden das Quotientenkriterium. Sei $a_n = \frac{2^n n!}{n^n}$. Dann gilt:
\begin{align}
\frac{a_{n+1}}{a_n} &= \frac{2^{n+1} (n+1)!}{(n+1)^{n+1}} \cdot \frac{n^n}{2^n n!}\\
&= \frac{2 \cdot (n+1) \cdot n^n}{(n+1)^{n+1}}\\
&= \frac{2 \cdot (n+1) \cdot n^n}{(n+1) \cdot (n+1)^n}\\
&= \frac{2 \cdot n^n}{(n+1)^n}\\
&= 2 \cdot \left(\frac{n}{n+1}\right)^n\\
&= 2 \cdot \left(1 - \frac{1}{n+1}\right)^n
\end{align}

Für den Grenzwert $n \to \infty$ verwenden wir, dass $\lim_{n \to \infty} \left(1 - \frac{1}{n+1}\right)^n = \lim_{n \to \infty} \left(1 - \frac{1}{n+1}\right)^{(n+1) \cdot \frac{n}{n+1}} = e^{-1}$.

Somit gilt:
\[
\lim_{n \to \infty} \frac{a_{n+1}}{a_n} = \frac{2}{e} \approx 0.736 < 1
\]

Nach dem Quotientenkriterium konvergiert die Reihe.

\item $\displaystyle \sum_{n=1}^\infty \frac{3^n n!}{n^n}$

Analog zu Teil (a) erhalten wir mit $a_n = \frac{3^n n!}{n^n}$:
\[
\frac{a_{n+1}}{a_n} = 3 \cdot \left(\frac{n}{n+1}\right)^n
\]

Für den Grenzwert gilt:
\[
\lim_{n \to \infty} \frac{a_{n+1}}{a_n} = \frac{3}{e} \approx 1.104 > 1
\]

Nach dem Quotientenkriterium divergiert die Reihe.

\item $\displaystyle \sum_{n=1}^\infty \frac{1 + (-1)^n n}{n^2}$

Wir zerlegen die Reihe:
\[
\sum_{n=1}^\infty \frac{1 + (-1)^n n}{n^2} = \sum_{n=1}^\infty \frac{1}{n^2} + \sum_{n=1}^\infty \frac{(-1)^n n}{n^2} = \sum_{n=1}^\infty \frac{1}{n^2} + \sum_{n=1}^\infty \frac{(-1)^n}{n}
\]

Die erste Reihe $\sum_{n=1}^\infty \frac{1}{n^2}$ ist eine konvergente $p$-Reihe mit $p = 2 > 1$.

Die zweite Reihe $\sum_{n=1}^\infty \frac{(-1)^n}{n}$ ist die alternierende harmonische Reihe, welche nach dem Leibniz-Kriterium konvergiert, da:
\begin{itemize}
\item $\frac{1}{n}$ ist monoton fallend
\item $\lim_{n \to \infty} \frac{1}{n} = 0$
\end{itemize}

Da beide Teilreihen konvergieren, konvergiert auch die ursprüngliche Reihe.

\item $\displaystyle \sum_{n=0}^\infty \frac{x^n}{1+x^{2n}}$ für $x \in \mathbb{R}$

Wir unterscheiden verschiedene Fälle:

\textbf{Fall 1: $|x| < 1$}

Für $|x| < 1$ gilt $x^{2n} \to 0$ für $n \to \infty$. Daher gilt für große $n$:
\[
\frac{x^n}{1+x^{2n}} \approx x^n
\]

Die Reihe verhält sich asymptotisch wie die geometrische Reihe $\sum_{n=0}^\infty x^n$, welche für $|x| < 1$ konvergiert.

\textbf{Fall 2: $|x| > 1$}

Für $|x| > 1$ können wir schreiben:
\[
\frac{x^n}{1+x^{2n}} = \frac{1}{x^{-n} + x^n} = \frac{1}{x^n(x^{-2n} + 1)} = \frac{1}{x^n} \cdot \frac{1}{1 + x^{-2n}}
\]

Für große $n$ gilt $x^{-2n} \to 0$, also:
\[
\frac{x^n}{1+x^{2n}} \approx \frac{1}{x^n} = \left(\frac{1}{|x|}\right)^n
\]

Da $|x| > 1$ ist $\frac{1}{|x|} < 1$, und die Reihe konvergiert wie eine geometrische Reihe.

\textbf{Fall 3: $x = 1$}

Für $x = 1$ erhalten wir:
\[
\sum_{n=0}^\infty \frac{1}{1+1} = \sum_{n=0}^\infty \frac{1}{2}
\]

Diese Reihe divergiert, da der allgemeine Term nicht gegen 0 konvergiert.

\textbf{Fall 4: $x = -1$}

Für $x = -1$ erhalten wir:
\[
\sum_{n=0}^\infty \frac{(-1)^n}{1+(-1)^{2n}} = \sum_{n=0}^\infty \frac{(-1)^n}{1+1} = \sum_{n=0}^\infty \frac{(-1)^n}{2}
\]

Diese alternierende Reihe divergiert, da der allgemeine Term $\frac{(-1)^n}{2}$ nicht gegen 0 konvergiert.

\textbf{Zusammenfassung:} Die Reihe konvergiert für $x \in \mathbb{R} \setminus \{-1, 1\}$ und divergiert für $x \in \{-1, 1\}$.

\end{enumerate}

\end{document}