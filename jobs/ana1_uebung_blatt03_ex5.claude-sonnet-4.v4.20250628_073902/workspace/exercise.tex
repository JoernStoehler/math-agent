\begin{exercise}
  In dieser Aufgabe betrachten wir die Menge
  \[
    \adjunction{\Q}{\sqrt 2}
    \coloneqq
    \left\{ a + b \sqrt{2} \mid  a,b \in \Q \right\}.
  \]
  Folgende Aussagen sind zu beweisen:
  \begin{enumerate}
  \item \( \Q \subsetneq \adjunction{\Q}{\sqrt 2} \subsetneq \R\)
  \item
    \( \fa{x,y \in \adjunction{\Q}{\sqrt 2}} x + y, x \cdot y \in
    \adjunction{\Q}{\sqrt 2}\)
  \item
    \( \fa{x \in \adjunction{\Q}{\sqrt{2}}} \ex{y \in \adjunction{\Q}{\sqrt{2}}}
      x + y = 0\)
    \item
      \( \fa{0 \neq x \in \adjunction{\Q}{\sqrt{2}}} \ex{y \in
        \adjunction{\Q}{\sqrt{2}}} x \cdot y = 1\)
  \end{enumerate}
  Wegen der Teilmengenbeziehung \(\adjunction{\Q}{\sqrt 2} \subset \R\) gelten
  die Assoziativ-, Kommutativ- und Distributivgesetze, die in \(\R\) gelten, auch
  in \(\adjunction{\Q}{\sqrt 2}\). Außerdem folgt daraus auch, dass
  \[
    \fa{x \in \adjunction{\Q}{\sqrt{2}}} x + 0 = x \land x \cdot 1 = x
  \]
  gilt. Damit ist \(\adjunction{\Q}{\sqrt{2}}\) insbesondere ein Körper.
\end{exercise}
