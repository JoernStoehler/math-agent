\documentclass{article}
\usepackage[utf8]{inputenc}
\usepackage{amsmath}
\usepackage{amssymb}
\usepackage{amsthm}

% Define commands to match the exercise notation
\newcommand{\Q}{\mathbb{Q}}
\newcommand{\R}{\mathbb{R}}
\newcommand{\adjunction}[2]{#1(#2)}
\newcommand{\fa}[1]{\forall #1 \colon}
\newcommand{\ex}[1]{\exists #1 \colon}

\begin{document}

% Aufgabe
\subsection*{Aufgabe}
In dieser Aufgabe betrachten wir die Menge
\[
  \adjunction{\Q}{\sqrt 2}
  :=
  \left\{ a + b \sqrt{2} \mid  a,b \in \Q \right\}.
\]
Folgende Aussagen sind zu beweisen:
\begin{enumerate}
\item \( \Q \subsetneq \adjunction{\Q}{\sqrt 2} \subsetneq \R\)
\item
  \( \fa{x,y \in \adjunction{\Q}{\sqrt 2}} x + y, x \cdot y \in
  \adjunction{\Q}{\sqrt 2}\)
\item
  \( \fa{x \in \adjunction{\Q}{\sqrt{2}}} \ex{y \in \adjunction{\Q}{\sqrt{2}}}
    x + y = 0\)
  \item
    \( \fa{0 \neq x \in \adjunction{\Q}{\sqrt{2}}} \ex{y \in
      \adjunction{\Q}{\sqrt{2}}} x \cdot y = 1\)
\end{enumerate}
Wegen der Teilmengenbeziehung \(\adjunction{\Q}{\sqrt 2} \subset \R\) gelten
die Assoziativ-, Kommutativ- und Distributivgesetze, die in \(\R\) gelten, auch
in \(\adjunction{\Q}{\sqrt 2}\). Außerdem folgt daraus auch, dass
\[
  \fa{x \in \adjunction{\Q}{\sqrt{2}}} x + 0 = x \land x \cdot 1 = x
\]
gilt. Damit ist \(\adjunction{\Q}{\sqrt{2}}\) insbesondere ein Körper.

\subsection*{Lösung}

\begin{enumerate}
\item \textbf{Behauptung:} \( \Q \subsetneq \adjunction{\Q}{\sqrt 2} \subsetneq \R\)

\textbf{Beweis:}

Zuerst zeigen wir \(\Q \subseteq \adjunction{\Q}{\sqrt 2}\):

Für jedes \(q \in \Q\) können wir schreiben: \(q = q + 0 \cdot \sqrt{2}\) mit \(q, 0 \in \Q\).
Somit ist \(q \in \adjunction{\Q}{\sqrt 2}\), also \(\Q \subseteq \adjunction{\Q}{\sqrt 2}\).

Um zu zeigen, dass die Inklusion echt ist, betrachten wir \(\sqrt{2} = 0 + 1 \cdot \sqrt{2} \in \adjunction{\Q}{\sqrt 2}\).
Wir zeigen, dass \(\sqrt{2} \notin \Q\):

Angenommen, \(\sqrt{2} \in \Q\). Dann existieren \(p, q \in \mathbb{Z}\) mit \(q \neq 0\) und \(\gcd(p,q) = 1\), sodass \(\sqrt{2} = \frac{p}{q}\).
Durch Quadrieren erhalten wir:
\[2 = \frac{p^2}{q^2} \implies 2q^2 = p^2\]

Dies zeigt, dass \(p^2\) gerade ist, also muss auch \(p\) gerade sein. Sei \(p = 2k\) für ein \(k \in \mathbb{Z}\).
Einsetzen liefert:
\[2q^2 = (2k)^2 = 4k^2 \implies q^2 = 2k^2\]

Also ist auch \(q^2\) gerade, und damit \(q\) gerade. Dies ist ein Widerspruch zu \(\gcd(p,q) = 1\).
Somit ist \(\sqrt{2} \notin \Q\) und daher \(\Q \subsetneq \adjunction{\Q}{\sqrt 2}\).

Nun zeigen wir \(\adjunction{\Q}{\sqrt 2} \subsetneq \R\):

Offensichtlich ist \(\adjunction{\Q}{\sqrt 2} \subseteq \R\), da alle Elemente der Form \(a + b\sqrt{2}\) mit \(a,b \in \Q\) reelle Zahlen sind.

Um zu zeigen, dass die Inklusion echt ist, betrachten wir \(\sqrt{3} \in \R\).
Wir zeigen, dass \(\sqrt{3} \notin \adjunction{\Q}{\sqrt 2}\):

Angenommen, \(\sqrt{3} \in \adjunction{\Q}{\sqrt 2}\). Dann existieren \(a, b \in \Q\) mit \(\sqrt{3} = a + b\sqrt{2}\).
Durch Quadrieren erhalten wir:
\[3 = a^2 + 2ab\sqrt{2} + 2b^2 = (a^2 + 2b^2) + 2ab\sqrt{2}\]

Da \(3 \in \Q\) und die Darstellung \(c + d\sqrt{2}\) mit \(c, d \in \Q\) eindeutig ist (was wir gleich zeigen werden), folgt:
\[a^2 + 2b^2 = 3 \text{ und } 2ab = 0\]

Aus \(2ab = 0\) folgt \(a = 0\) oder \(b = 0\).

Fall 1: \(a = 0\). Dann ist \(2b^2 = 3 \implies b^2 = \frac{3}{2}\).
Dies würde bedeuten, dass \(\sqrt{\frac{3}{2}} \in \Q\), was ein Widerspruch ist.

Fall 2: \(b = 0\). Dann ist \(a^2 = 3 \implies a = \sqrt{3}\).
Dies würde bedeuten, dass \(\sqrt{3} \in \Q\), was ein Widerspruch ist.

Somit ist \(\sqrt{3} \notin \adjunction{\Q}{\sqrt 2}\) und daher \(\adjunction{\Q}{\sqrt 2} \subsetneq \R\).

\textit{Anmerkung zur Eindeutigkeit der Darstellung:} Seien \(a_1 + b_1\sqrt{2} = a_2 + b_2\sqrt{2}\) mit \(a_1, a_2, b_1, b_2 \in \Q\).
Dann ist \((a_1 - a_2) + (b_1 - b_2)\sqrt{2} = 0\).
Falls \(b_1 \neq b_2\), dann wäre \(\sqrt{2} = \frac{a_2 - a_1}{b_1 - b_2} \in \Q\), was ein Widerspruch ist.
Also ist \(b_1 = b_2\) und damit auch \(a_1 = a_2\).

\item \textbf{Behauptung:} \( \fa{x,y \in \adjunction{\Q}{\sqrt 2}} x + y, x \cdot y \in \adjunction{\Q}{\sqrt 2}\)

\textbf{Beweis:}

Seien \(x = a_1 + b_1\sqrt{2}\) und \(y = a_2 + b_2\sqrt{2}\) mit \(a_1, a_2, b_1, b_2 \in \Q\).

\textbf{Addition:}
\begin{align}
x + y &= (a_1 + b_1\sqrt{2}) + (a_2 + b_2\sqrt{2})\\
&= (a_1 + a_2) + (b_1 + b_2)\sqrt{2}
\end{align}

Da \(a_1 + a_2 \in \Q\) und \(b_1 + b_2 \in \Q\) (denn \(\Q\) ist unter Addition abgeschlossen), 
ist \(x + y \in \adjunction{\Q}{\sqrt 2}\).

\textbf{Multiplikation:}
\begin{align}
x \cdot y &= (a_1 + b_1\sqrt{2}) \cdot (a_2 + b_2\sqrt{2})\\
&= a_1a_2 + a_1b_2\sqrt{2} + b_1a_2\sqrt{2} + b_1b_2(\sqrt{2})^2\\
&= a_1a_2 + (a_1b_2 + b_1a_2)\sqrt{2} + 2b_1b_2\\
&= (a_1a_2 + 2b_1b_2) + (a_1b_2 + b_1a_2)\sqrt{2}
\end{align}

Da \(a_1a_2 + 2b_1b_2 \in \Q\) und \(a_1b_2 + b_1a_2 \in \Q\) (denn \(\Q\) ist unter Addition und Multiplikation abgeschlossen),
ist \(x \cdot y \in \adjunction{\Q}{\sqrt 2}\).

\item \textbf{Behauptung:} \( \fa{x \in \adjunction{\Q}{\sqrt{2}}} \ex{y \in \adjunction{\Q}{\sqrt{2}}} x + y = 0\)

\textbf{Beweis:}

Sei \(x = a + b\sqrt{2} \in \adjunction{\Q}{\sqrt{2}}\) mit \(a, b \in \Q\).

Wir suchen \(y \in \adjunction{\Q}{\sqrt{2}}\) mit \(x + y = 0\).

Setze \(y = -a - b\sqrt{2}\). Da \(-a \in \Q\) und \(-b \in \Q\) (denn \(\Q\) enthält additive Inverse),
ist \(y \in \adjunction{\Q}{\sqrt{2}}\).

Wir prüfen:
\begin{align}
x + y &= (a + b\sqrt{2}) + (-a - b\sqrt{2})\\
&= (a - a) + (b - b)\sqrt{2}\\
&= 0 + 0\sqrt{2}\\
&= 0
\end{align}

Somit existiert für jedes \(x \in \adjunction{\Q}{\sqrt{2}}\) ein additives Inverses \(y = -x \in \adjunction{\Q}{\sqrt{2}}\).

\item \textbf{Behauptung:} \( \fa{0 \neq x \in \adjunction{\Q}{\sqrt{2}}} \ex{y \in \adjunction{\Q}{\sqrt{2}}} x \cdot y = 1\)

\textbf{Beweis:}

Sei \(x = a + b\sqrt{2} \in \adjunction{\Q}{\sqrt{2}}\) mit \(x \neq 0\), also \(a \neq 0\) oder \(b \neq 0\).

Wir suchen \(y \in \adjunction{\Q}{\sqrt{2}}\) mit \(x \cdot y = 1\).

Wir verwenden die Methode der Konjugation. Definiere das Konjugat von \(x\) als \(\overline{x} = a - b\sqrt{2}\).

Berechne:
\begin{align}
x \cdot \overline{x} &= (a + b\sqrt{2})(a - b\sqrt{2})\\
&= a^2 - ab\sqrt{2} + ab\sqrt{2} - b^2(\sqrt{2})^2\\
&= a^2 - 2b^2
\end{align}

Wir zeigen, dass \(a^2 - 2b^2 \neq 0\):

Angenommen, \(a^2 - 2b^2 = 0\), also \(a^2 = 2b^2\).

Falls \(b = 0\), dann \(a^2 = 0\), also \(a = 0\), was \(x = 0\) bedeutet - Widerspruch.

Falls \(b \neq 0\), dann \(\left(\frac{a}{b}\right)^2 = 2\), also \(\frac{a}{b} = \pm\sqrt{2}\).
Da \(a, b \in \Q\) mit \(b \neq 0\), ist \(\frac{a}{b} \in \Q\).
Dies würde bedeuten, dass \(\sqrt{2} \in \Q\), was ein Widerspruch ist.

Somit ist \(a^2 - 2b^2 \neq 0\).

Setze nun:
\[y = \frac{\overline{x}}{x \cdot \overline{x}} = \frac{a - b\sqrt{2}}{a^2 - 2b^2} = \frac{a}{a^2 - 2b^2} - \frac{b}{a^2 - 2b^2}\sqrt{2}\]

Da \(a^2 - 2b^2 \neq 0\) und \(a, b, a^2 - 2b^2 \in \Q\), sind \(\frac{a}{a^2 - 2b^2} \in \Q\) und \(\frac{-b}{a^2 - 2b^2} \in \Q\).
Somit ist \(y \in \adjunction{\Q}{\sqrt{2}}\).

Wir verifizieren:
\begin{align}
x \cdot y &= x \cdot \frac{\overline{x}}{x \cdot \overline{x}}\\
&= \frac{x \cdot \overline{x}}{x \cdot \overline{x}}\\
&= 1
\end{align}

Somit existiert für jedes \(x \in \adjunction{\Q}{\sqrt{2}}\) mit \(x \neq 0\) ein multiplikatives Inverses \(y = x^{-1} \in \adjunction{\Q}{\sqrt{2}}\).
\end{enumerate}

\textbf{Fazit:} Wir haben gezeigt, dass \(\adjunction{\Q}{\sqrt{2}}\) eine echte Teilmenge von \(\R\) ist, die \(\Q\) echt enthält,
und dass \(\adjunction{\Q}{\sqrt{2}}\) unter Addition und Multiplikation abgeschlossen ist sowie additive und multiplikative Inverse besitzt.
Zusammen mit den gegebenen Eigenschaften folgt, dass \(\adjunction{\Q}{\sqrt{2}}\) ein Körper ist.

\end{document}