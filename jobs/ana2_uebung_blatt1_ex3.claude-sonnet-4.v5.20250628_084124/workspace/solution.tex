\documentclass{article}
\usepackage[utf8]{inputenc}
\usepackage{amsmath}
\usepackage{amssymb}
\usepackage{amsthm}

\begin{document}

% Aufgabe
\subsection*{Aufgabe}
Betrachte die Integrale
$$
   I_m := \int_0^{\pi/2}\sin^mx\,dx,\qquad m\in\mathbb{N}_0.
$$
\begin{enumerate}
\renewcommand{\labelenumi}{(\alph{enumi})}
\item Beweise durch partielle Integration die Rekursionsformel
$$
   I_m = \frac{m-1}{m}I_{m-2},\qquad m\geq 2.
$$
\item Folgere aus (a) folgende Formeln für $n\in\mathbb{N}_0$:
\begin{align*}
   I_{2n} &= \frac{(2n-1)(2n-3)\cdots 3\cdot 1}{(2n-0)(2n-2)\cdots 4\cdot 2}\cdot\frac{\pi}{2},\\
   I_{2n+1} &= \frac{(2n-0)(2n-2)\cdots 4\cdot 2}{(2n+1)(2n-1)\cdots 5\cdot 3}\cdot 1.
\end{align*}
\item Beweise: Die Folge $(I_m)$ ist streng monoton fallend und $\lim_{m\to\infty}I_m=0$.
\item Zeige $\lim_{n\to\infty}\frac{I_{2n+1}}{I_{2n}}=1$ und folgere
daraus die {\em Wallis'sche Produktdarstellung}
$$
   \frac{\pi}{2} = \prod_{n=1}^\infty \frac{4n^2}{4n^2-1} = \frac{2}{1} \cdot \frac{2}{3} \cdot \frac{4}{3} \cdot \frac{4}{5} \cdot \frac{6}{5} \cdot \frac{6}{7} \cdot \cdots
$$
\end{enumerate}

\subsection*{Lösung}

\textbf{(a)} Wir beweisen die Rekursionsformel $I_m = \frac{m-1}{m}I_{m-2}$ für $m \geq 2$ mittels partieller Integration.

Für das Integral $I_m = \int_0^{\pi/2} \sin^m(x) \, dx$ schreiben wir
$$I_m = \int_0^{\pi/2} \sin^{m-1}(x) \cdot \sin(x) \, dx$$

Wir wenden partielle Integration an mit:
\begin{align*}
u &= \sin^{m-1}(x), \quad dv = \sin(x) \, dx\\
du &= (m-1)\sin^{m-2}(x)\cos(x) \, dx, \quad v = -\cos(x)
\end{align*}

Damit erhalten wir:
\begin{align*}
I_m &= \left[\sin^{m-1}(x) \cdot (-\cos(x))\right]_0^{\pi/2} + \int_0^{\pi/2} \cos(x) \cdot (m-1)\sin^{m-2}(x)\cos(x) \, dx\\
&= \left[-\sin^{m-1}(x)\cos(x)\right]_0^{\pi/2} + (m-1)\int_0^{\pi/2} \sin^{m-2}(x)\cos^2(x) \, dx
\end{align*}

Der Randterm verschwindet, da $\cos(\pi/2) = 0$ und $\sin(0) = 0$:
$$I_m = 0 + (m-1)\int_0^{\pi/2} \sin^{m-2}(x)\cos^2(x) \, dx$$

Mit der trigonometrischen Identität $\cos^2(x) = 1 - \sin^2(x)$ erhalten wir:
\begin{align*}
I_m &= (m-1)\int_0^{\pi/2} \sin^{m-2}(x)(1-\sin^2(x)) \, dx\\
&= (m-1)\int_0^{\pi/2} \sin^{m-2}(x) \, dx - (m-1)\int_0^{\pi/2} \sin^m(x) \, dx\\
&= (m-1)I_{m-2} - (m-1)I_m
\end{align*}

Umstellen nach $I_m$ liefert:
\begin{align*}
I_m + (m-1)I_m &= (m-1)I_{m-2}\\
m \cdot I_m &= (m-1)I_{m-2}\\
I_m &= \frac{m-1}{m}I_{m-2}
\end{align*}

\textbf{(b)} Wir folgern die expliziten Formeln für $I_{2n}$ und $I_{2n+1}$.

Zunächst berechnen wir die Basiswerte:
\begin{align*}
I_0 &= \int_0^{\pi/2} 1 \, dx = \left[x\right]_0^{\pi/2} = \frac{\pi}{2}\\
I_1 &= \int_0^{\pi/2} \sin(x) \, dx = \left[-\cos(x)\right]_0^{\pi/2} = -\cos(\pi/2) + \cos(0) = 0 + 1 = 1
\end{align*}

Für gerade Indizes $I_{2n}$ wenden wir die Rekursionsformel wiederholt an:
\begin{align*}
I_{2n} &= \frac{2n-1}{2n} \cdot I_{2n-2}\\
&= \frac{2n-1}{2n} \cdot \frac{2n-3}{2n-2} \cdot I_{2n-4}\\
&= \frac{2n-1}{2n} \cdot \frac{2n-3}{2n-2} \cdot \frac{2n-5}{2n-4} \cdot I_{2n-6}\\
&\vdots\\
&= \frac{(2n-1)(2n-3)(2n-5)\cdots 3 \cdot 1}{(2n)(2n-2)(2n-4)\cdots 4 \cdot 2} \cdot I_0\\
&= \frac{(2n-1)(2n-3)\cdots 3 \cdot 1}{(2n)(2n-2)\cdots 4 \cdot 2} \cdot \frac{\pi}{2}
\end{align*}

Für ungerade Indizes $I_{2n+1}$ verfahren wir analog:
\begin{align*}
I_{2n+1} &= \frac{2n}{2n+1} \cdot I_{2n-1}\\
&= \frac{2n}{2n+1} \cdot \frac{2n-2}{2n-1} \cdot I_{2n-3}\\
&= \frac{2n}{2n+1} \cdot \frac{2n-2}{2n-1} \cdot \frac{2n-4}{2n-3} \cdot I_{2n-5}\\
&\vdots\\
&= \frac{(2n)(2n-2)(2n-4)\cdots 4 \cdot 2}{(2n+1)(2n-1)(2n-3)\cdots 5 \cdot 3} \cdot I_1\\
&= \frac{(2n)(2n-2)\cdots 4 \cdot 2}{(2n+1)(2n-1)\cdots 5 \cdot 3} \cdot 1
\end{align*}

\textbf{(c)} Wir beweisen, dass die Folge $(I_m)$ streng monoton fallend ist und $\lim_{m\to\infty}I_m=0$.

Für die strenge Monotonie betrachten wir: Für $x \in (0, \pi/2)$ gilt $0 < \sin(x) < 1$ (außer bei $x = \pi/2$, wo $\sin(x) = 1$).

Daher gilt für alle $x \in (0, \pi/2)$:
$$\sin^{m+1}(x) = \sin^m(x) \cdot \sin(x) < \sin^m(x)$$

Da die Ungleichung auf einem Intervall positiven Maßes strikt ist, folgt:
$$I_{m+1} = \int_0^{\pi/2} \sin^{m+1}(x) \, dx < \int_0^{\pi/2} \sin^m(x) \, dx = I_m$$

Also ist $(I_m)$ streng monoton fallend.

Für den Grenzwert: Da $0 \leq \sin(x) \leq 1$ auf $[0, \pi/2]$, gilt $0 \leq \sin^m(x) \leq 1$ für alle $m$.

Für jedes feste $x \in [0, \pi/2)$ gilt $\sin(x) < 1$, also $\lim_{m\to\infty} \sin^m(x) = 0$.

Die Funktionenfolge $\sin^m(x)$ wird durch die integrierbare Funktion $1$ dominiert. Nach dem Satz von der dominierten Konvergenz gilt:
$$\lim_{m\to\infty} I_m = \lim_{m\to\infty} \int_0^{\pi/2} \sin^m(x) \, dx = \int_0^{\pi/2} \lim_{m\to\infty} \sin^m(x) \, dx = \int_0^{\pi/2} 0 \, dx = 0$$

\textbf{(d)} Wir zeigen $\lim_{n\to\infty}\frac{I_{2n+1}}{I_{2n}}=1$ und folgern die Wallis'sche Produktdarstellung.

Aus Teil (c) wissen wir, dass $(I_m)$ streng monoton fallend ist. Daher gilt:
$$I_{2n+1} < I_{2n} < I_{2n-1}$$

Aus der Rekursionsformel folgt:
$$I_{2n+1} = \frac{2n}{2n+1} I_{2n-1}$$

Also:
$$\frac{2n}{2n+1} I_{2n-1} < I_{2n} < I_{2n-1}$$

Division durch $I_{2n-1}$ liefert:
$$\frac{2n}{2n+1} < \frac{I_{2n}}{I_{2n-1}} < 1$$

Daraus folgt:
$$\frac{2n}{2n+1} < \frac{I_{2n}}{I_{2n+1}} \cdot \frac{I_{2n+1}}{I_{2n-1}} < 1$$

Mit $I_{2n+1} = \frac{2n}{2n+1} I_{2n-1}$ erhalten wir $\frac{I_{2n+1}}{I_{2n-1}} = \frac{2n}{2n+1}$, also:
$$\frac{2n}{2n+1} < \frac{I_{2n}}{I_{2n+1}} \cdot \frac{2n}{2n+1} < 1$$

Dies vereinfacht sich zu:
$$1 < \frac{I_{2n}}{I_{2n+1}} < \frac{2n+1}{2n}$$

Invertieren liefert:
$$\frac{2n}{2n+1} < \frac{I_{2n+1}}{I_{2n}} < 1$$

Da $\lim_{n\to\infty} \frac{2n}{2n+1} = 1$, folgt nach dem Sandwich-Theorem:
$$\lim_{n\to\infty} \frac{I_{2n+1}}{I_{2n}} = 1$$

Nun berechnen wir $\frac{I_{2n+1}}{I_{2n}}$ explizit:
\begin{align*}
\frac{I_{2n+1}}{I_{2n}} &= \frac{\frac{(2n)(2n-2)\cdots 4 \cdot 2}{(2n+1)(2n-1)\cdots 5 \cdot 3} \cdot 1}{\frac{(2n-1)(2n-3)\cdots 3 \cdot 1}{(2n)(2n-2)\cdots 4 \cdot 2} \cdot \frac{\pi}{2}}\\
&= \frac{(2n)(2n-2)\cdots 4 \cdot 2}{(2n+1)(2n-1)\cdots 5 \cdot 3} \cdot \frac{(2n)(2n-2)\cdots 4 \cdot 2}{(2n-1)(2n-3)\cdots 3 \cdot 1} \cdot \frac{2}{\pi}\\
&= \frac{[(2n)(2n-2)\cdots 4 \cdot 2]^2}{(2n+1)(2n-1)\cdots 5 \cdot 3 \cdot (2n-1)(2n-3)\cdots 3 \cdot 1} \cdot \frac{2}{\pi}\\
&= \frac{[(2n)!!]^2}{(2n+1)! / 2^n} \cdot \frac{2}{\pi}
\end{align*}

Wir können dies umschreiben als:
$$\frac{I_{2n+1}}{I_{2n}} = \frac{2}{\pi} \cdot \frac{[(2n)!!]^2}{(2n+1)!!^2} \cdot (2n+1)$$

Mit etwas Umformung erhalten wir:
$$\frac{I_{2n+1}}{I_{2n}} = \frac{2}{\pi} \cdot \prod_{k=1}^n \frac{(2k)^2}{(2k-1)(2k+1)}$$

Da $\lim_{n\to\infty} \frac{I_{2n+1}}{I_{2n}} = 1$, folgt:
$$1 = \frac{2}{\pi} \cdot \prod_{k=1}^\infty \frac{(2k)^2}{(2k-1)(2k+1)}$$

Umstellen nach $\pi/2$ liefert:
$$\frac{\pi}{2} = \prod_{k=1}^\infty \frac{(2k)^2}{(2k-1)(2k+1)} = \prod_{k=1}^\infty \frac{4k^2}{4k^2-1}$$

Dies ist die Wallis'sche Produktdarstellung. Ausgeschrieben:
$$\frac{\pi}{2} = \frac{2 \cdot 2}{1 \cdot 3} \cdot \frac{4 \cdot 4}{3 \cdot 5} \cdot \frac{6 \cdot 6}{5 \cdot 7} \cdot \cdots = \frac{2}{1} \cdot \frac{2}{3} \cdot \frac{4}{3} \cdot \frac{4}{5} \cdot \frac{6}{5} \cdot \frac{6}{7} \cdot \cdots$$

\end{document}