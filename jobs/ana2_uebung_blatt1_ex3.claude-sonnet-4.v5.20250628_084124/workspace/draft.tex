% Draft solution for the exercise

Part (a): Prove I_m = (m-1)/m * I_{m-2}

For I_m = ∫₀^{π/2} sin^m(x) dx, we use partial integration.

Let u = sin^{m-1}(x) and dv = sin(x)dx
Then du = (m-1)sin^{m-2}(x)cos(x)dx and v = -cos(x)

I_m = [sin^{m-1}(x)(-cos(x))]₀^{π/2} + ∫₀^{π/2} cos(x)(m-1)sin^{m-2}(x)cos(x)dx
    = 0 + (m-1)∫₀^{π/2} sin^{m-2}(x)cos²(x)dx
    = (m-1)∫₀^{π/2} sin^{m-2}(x)(1-sin²(x))dx
    = (m-1)∫₀^{π/2} sin^{m-2}(x)dx - (m-1)∫₀^{π/2} sin^m(x)dx
    = (m-1)I_{m-2} - (m-1)I_m

So: I_m + (m-1)I_m = (m-1)I_{m-2}
    m·I_m = (m-1)I_{m-2}
    I_m = (m-1)/m · I_{m-2}

Part (b): Derive formulas for I_{2n} and I_{2n+1}

First, we need base cases:
I_0 = ∫₀^{π/2} 1 dx = π/2
I_1 = ∫₀^{π/2} sin(x) dx = [-cos(x)]₀^{π/2} = 1

For even indices:
I_{2n} = (2n-1)/(2n) · I_{2n-2} = (2n-1)/(2n) · (2n-3)/(2n-2) · I_{2n-4} = ...
       = (2n-1)(2n-3)···3·1 / (2n)(2n-2)···4·2 · I_0
       = (2n-1)!!! / (2n)!! · π/2

For odd indices:
I_{2n+1} = (2n)/(2n+1) · I_{2n-1} = (2n)/(2n+1) · (2n-2)/(2n-1) · I_{2n-3} = ...
         = (2n)(2n-2)···4·2 / (2n+1)(2n-1)···5·3 · I_1
         = (2n)!! / (2n+1)!! · 1

Part (c): Prove monotonicity and limit

For monotonicity: For x ∈ (0, π/2), we have 0 < sin(x) < 1
Therefore sin^{m+1}(x) < sin^m(x)
Thus I_{m+1} = ∫₀^{π/2} sin^{m+1}(x)dx < ∫₀^{π/2} sin^m(x)dx = I_m

For the limit: Since 0 ≤ sin(x) ≤ 1 on [0, π/2], we have:
0 ≤ I_m ≤ ∫₀^{π/2} sin(x)^m dx

For x ∈ (0, π/2), sin(x) < 1 except at x = π/2.
As m → ∞, sin(x)^m → 0 for all x ∈ [0, π/2) 
By dominated convergence theorem, lim_{m→∞} I_m = 0

Part (d): Show lim I_{2n+1}/I_{2n} = 1 and derive Wallis product

I_{2n+1}/I_{2n} = [(2n)!!/(2n+1)!! · 1] / [(2n-1)!!/(2n)!! · π/2]
                = (2n)!! · (2n)!! · 2 / [(2n+1)!! · (2n-1)!! · π]
                = 2(2n)!!² / [(2n+1)(2n-1)!! · (2n-1)!! · π]
                = 2(2n)!!² / [(2n+1) · ((2n-1)!!)² · π]

Need to show this → 1 as n → ∞