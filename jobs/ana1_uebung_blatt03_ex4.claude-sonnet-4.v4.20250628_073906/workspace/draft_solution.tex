\documentclass{article}
\usepackage[utf8]{inputenc}
\usepackage{amsmath}
\usepackage{amssymb}
\usepackage{amsthm}

% Commands from exercise
\newcommand{\abs}[1]{|#1|}
\newcommand{\R}{\mathbb{R}}
\newcommand{\fa}[1]{\forall #1 \colon}

\begin{document}

\subsection*{Aufgabe}
Aus der Vorlesung ist die Betragsfunktion 
\[
  \abs{{-}} \colon \R \to \R^{+}_{0}, x \mapsto
  \begin{cases}
    x  & x \geq 0 \\
    -x & x < 0    \\
  \end{cases}
\]
bekannt. Beweise folgende Eigenschaften:
\begin{enumerate}
\item \(\fa{x \in \mathbb{R}} \abs{x} \geq 0\)
\item \(\fa{x \in \R} \abs{x} = 0 \implies x = 0\)
\item \(\fa{x, y \in \R} \abs{x \cdot y} = \abs{x} \cdot \abs{y}\)
\item \(\fa{x, y \in \R} \abs{x + y} \leq \abs{x} + \abs{y}\)
\item \(\fa{a,b \in \R} \abs{a + b} + \abs{a - b} \geq \abs{a} + \abs{b}\)
\item \(\fa{a,b \in \R^*} \abs{\frac{a}{b} + \frac{b}{a}} \geq 2\)
\end{enumerate}

\subsection*{Lösung}

\textbf{1. Beweis von \(\fa{x \in \mathbb{R}} \abs{x} \geq 0\):}

Sei $x \in \R$ beliebig. Wir unterscheiden zwei Fälle gemäß der Definition der Betragsfunktion:

Fall 1: $x \geq 0$. Dann ist $\abs{x} = x \geq 0$ nach Voraussetzung.

Fall 2: $x < 0$. Dann ist $\abs{x} = -x$. Da $x < 0$ ist, folgt $-x > 0$, also $\abs{x} > 0 \geq 0$.

In beiden Fällen gilt $\abs{x} \geq 0$. $\square$

\textbf{2. Beweis von \(\fa{x \in \R} \abs{x} = 0 \implies x = 0\):}

Sei $x \in \R$ mit $\abs{x} = 0$. Wir unterscheiden wieder zwei Fälle:

Fall 1: $x \geq 0$. Dann ist $\abs{x} = x = 0$, also $x = 0$.

Fall 2: $x < 0$. Dann ist $\abs{x} = -x = 0$, also $x = 0$. Dies ist ein Widerspruch zur Annahme $x < 0$.

Also kann nur Fall 1 eintreten und es folgt $x = 0$. $\square$

\textbf{3. Beweis von \(\fa{x, y \in \R} \abs{x \cdot y} = \abs{x} \cdot \abs{y}\):}

Seien $x, y \in \R$ beliebig. Wir betrachten vier Fälle:

Fall 1: $x \geq 0$ und $y \geq 0$. Dann ist $xy \geq 0$, also
\[\abs{xy} = xy = x \cdot y = \abs{x} \cdot \abs{y}.\]

Fall 2: $x \geq 0$ und $y < 0$. Dann ist $xy \leq 0$, genauer $xy < 0$ falls $x > 0$ oder $xy = 0$ falls $x = 0$.
Für $x > 0$: $\abs{xy} = -(xy) = x \cdot (-y) = \abs{x} \cdot \abs{y}$.
Für $x = 0$: $\abs{xy} = \abs{0} = 0 = 0 \cdot \abs{y} = \abs{x} \cdot \abs{y}$.

Fall 3: $x < 0$ und $y \geq 0$. Analog zu Fall 2 (mit vertauschten Rollen von $x$ und $y$).

Fall 4: $x < 0$ und $y < 0$. Dann ist $xy > 0$, also
\[\abs{xy} = xy = (-x) \cdot (-y) = \abs{x} \cdot \abs{y}.\]

In allen Fällen gilt $\abs{xy} = \abs{x} \cdot \abs{y}$. $\square$

\textbf{4. Beweis von \(\fa{x, y \in \R} \abs{x + y} \leq \abs{x} + \abs{y}\) (Dreiecksungleichung):}

Seien $x, y \in \R$ beliebig. Wir nutzen folgende Beobachtung:
Für alle $a \in \R$ gilt $-\abs{a} \leq a \leq \abs{a}$.

Daher haben wir:
\begin{align}
-\abs{x} &\leq x \leq \abs{x} \\
-\abs{y} &\leq y \leq \abs{y}
\end{align}

Addition der beiden Ungleichungen ergibt:
\[-(\abs{x} + \abs{y}) \leq x + y \leq \abs{x} + \abs{y}\]

Dies bedeutet $\abs{x + y} \leq \abs{x} + \abs{y}$. $\square$

\textbf{5. Beweis von \(\fa{a,b \in \R} \abs{a + b} + \abs{a - b} \geq \abs{a} + \abs{b}\):}

Seien $a, b \in \R$ beliebig. Wir betrachten vier Fälle basierend auf den Vorzeichen von $a$ und $b$:

Fall 1: $a \geq 0$ und $b \geq 0$. 
Dann ist $a + b \geq 0$, also $\abs{a + b} = a + b$.
Falls $a \geq b$, ist $a - b \geq 0$, also $\abs{a - b} = a - b$.
Somit: $\abs{a + b} + \abs{a - b} = (a + b) + (a - b) = 2a = 2\abs{a} \geq \abs{a} + \abs{b}$.
Falls $a < b$, ist $a - b < 0$, also $\abs{a - b} = -(a - b) = b - a$.
Somit: $\abs{a + b} + \abs{a - b} = (a + b) + (b - a) = 2b = 2\abs{b} \geq \abs{a} + \abs{b}$.

Fall 2: $a \geq 0$ und $b < 0$. Dann ist $\abs{a} = a$ und $\abs{b} = -b$.
Falls $a + b \geq 0$, ist $\abs{a + b} = a + b$ und $\abs{a - b} = a - b = a + (-b) = \abs{a} + \abs{b}$.
Falls $a + b < 0$, ist $\abs{a + b} = -(a + b) = -a - b$ und $\abs{a - b} = a - b$.
Somit: $\abs{a + b} + \abs{a - b} = (-a - b) + (a - b) = -2b = 2\abs{b} \geq \abs{a} + \abs{b}$.

Fall 3: $a < 0$ und $b \geq 0$. Analog zu Fall 2 (mit vertauschten Rollen).

Fall 4: $a < 0$ und $b < 0$. Dann ist $\abs{a} = -a$, $\abs{b} = -b$, und $a + b < 0$, also $\abs{a + b} = -(a + b)$.
Falls $a \leq b$, ist $a - b \leq 0$, also $\abs{a - b} = -(a - b) = b - a$.
Somit: $\abs{a + b} + \abs{a - b} = (-(a + b)) + (b - a) = -2a = 2\abs{a} \geq \abs{a} + \abs{b}$.
Falls $a > b$, ist $a - b > 0$, also $\abs{a - b} = a - b$.
Somit: $\abs{a + b} + \abs{a - b} = (-(a + b)) + (a - b) = -2b = 2\abs{b} \geq \abs{a} + \abs{b}$.

In allen Fällen gilt die behauptete Ungleichung. $\square$

\textbf{6. Beweis von \(\fa{a,b \in \R^*} \abs{\frac{a}{b} + \frac{b}{a}} \geq 2\):}

Seien $a, b \in \R^*$ (also $a \neq 0$ und $b \neq 0$). Setze $x = \frac{a}{b}$. Dann ist $x \neq 0$ und wir müssen zeigen:
\[\abs{x + \frac{1}{x}} \geq 2\]

Fall 1: $x > 0$. Die Funktion $f(x) = x + \frac{1}{x}$ hat für $x > 0$ ein Minimum bei $x = 1$ mit $f(1) = 2$.
Dies kann man durch Ableitung zeigen: $f'(x) = 1 - \frac{1}{x^2} = 0 \iff x = 1$ (da $x > 0$).
Für $0 < x < 1$ ist $f'(x) < 0$ und für $x > 1$ ist $f'(x) > 0$, also hat $f$ bei $x = 1$ ein Minimum.
Daher ist $x + \frac{1}{x} \geq 2$ für alle $x > 0$, also $\abs{x + \frac{1}{x}} = x + \frac{1}{x} \geq 2$.

Fall 2: $x < 0$. Setze $y = -x > 0$. Dann ist
\[\abs{x + \frac{1}{x}} = \abs{-y + \frac{1}{-y}} = \abs{-y - \frac{1}{y}} = \abs{-(y + \frac{1}{y})} = y + \frac{1}{y} \geq 2\]
nach Fall 1.

In beiden Fällen gilt $\abs{\frac{a}{b} + \frac{b}{a}} \geq 2$. $\square$

\end{document}