\documentclass{article}
\usepackage[utf8]{inputenc}
\usepackage{amsmath,amssymb,amsfonts}
\usepackage{enumerate}
\usepackage{graphicx}

% Define custom commands to match the exercise
\newcommand{\C}{\mathbb{C}}
\newcommand{\im}{\mathrm{i}}
\newcommand{\Set}[1]{\left\{#1\right\}}
\newcommand{\abs}[1]{\left|#1\right|}
\newcommand{\conj}[1]{\overline{#1}}

\begin{document}

% Aufgabe
\subsection*{Aufgabe: Komplexe Zahlenmengen zeichnen}
Skizziere die folgenden Mengen in der komplexen Zahlenebene:
\begin{enumerate}[(a)]
\item $\Set{ z \in \C | -3 \leq \Im(z+5-3\im) \leq 2 }$
\item $\Set{ z \in \C | \abs{z + 2 - \im} \geq 3 }$
\item $\Set{ z \in \C | z \conj{z} - (z + \conj{z} )^2 \leq 1 }$
\end{enumerate}

\subsection*{Lösung}

\textbf{(a)} $\Set{ z \in \C | -3 \leq \Im(z+5-3\im) \leq 2 }$

Sei $z = x + \im y$ mit $x, y \in \mathbb{R}$. Dann gilt:
\begin{align}
z + 5 - 3\im &= (x + \im y) + 5 - 3\im\\
&= (x + 5) + \im(y - 3)
\end{align}

Der Imaginärteil von $z + 5 - 3\im$ ist also $\Im(z + 5 - 3\im) = y - 3$.

Die Bedingung $-3 \leq \Im(z+5-3\im) \leq 2$ wird zu:
\begin{align}
-3 &\leq y - 3 \leq 2\\
-3 + 3 &\leq y \leq 2 + 3\\
0 &\leq y \leq 5
\end{align}

Die gesuchte Menge ist also ein horizontaler Streifen in der komplexen Ebene:
$$M_a = \Set{z = x + \im y \in \C | x \in \mathbb{R}, 0 \leq y \leq 5}$$

Dies ist ein horizontaler Streifen zwischen den Geraden $y = 0$ und $y = 5$.

\textbf{(b)} $\Set{ z \in \C | \abs{z + 2 - \im} \geq 3 }$

Die Bedingung $\abs{z + 2 - \im} \geq 3$ beschreibt alle komplexen Zahlen $z$, deren Abstand zum Punkt $-2 + \im$ mindestens 3 beträgt. Dies ist das Äußere (einschließlich des Randes) eines Kreises mit Mittelpunkt $z_0 = -2 + \im$ und Radius $r = 3$.

Die gesuchte Menge ist:
$$M_b = \Set{z \in \C | \abs{z - (-2 + \im)} \geq 3}$$

Dies ist das Äußere (einschließlich Rand) eines Kreises mit Mittelpunkt $z_0 = -2 + \im$ und Radius $r = 3$.

\textbf{(c)} $\Set{ z \in \C | z \conj{z} - (z + \conj{z} )^2 \leq 1 }$

Sei wieder $z = x + \im y$ mit $x, y \in \mathbb{R}$. Dann ist $\conj{z} = x - \im y$, und wir berechnen:
\begin{align}
z \cdot \conj{z} &= (x + \im y)(x - \im y) = x^2 + y^2 = |z|^2\\
z + \conj{z} &= (x + \im y) + (x - \im y) = 2x\\
(z + \conj{z})^2 &= (2x)^2 = 4x^2
\end{align}

Die Bedingung $z \conj{z} - (z + \conj{z} )^2 \leq 1$ wird zu:
\begin{align}
x^2 + y^2 - 4x^2 &\leq 1\\
y^2 - 3x^2 &\leq 1
\end{align}

Dies ist die Ungleichung einer Hyperbel. Die Randkurve ist gegeben durch $y^2 - 3x^2 = 1$, was umgeformt werden kann zu:
$$\frac{y^2}{1} - \frac{x^2}{1/3} = 1$$

Dies ist eine Hyperbel mit den Scheitelpunkten bei $(0, \pm 1)$. 

Die gesuchte Menge ist:
$$M_c = \Set{z = x + \im y \in \C | y^2 - 3x^2 \leq 1}$$

Dies ist der Bereich zwischen den beiden Ästen der Hyperbel $y^2 - 3x^2 = 1$.

\subsection*{Skizzen}

Die drei Mengen sind in der komplexen Zahlenebene wie folgt dargestellt:

\textbf{(a)} Horizontaler Streifen $0 \leq y \leq 5$:
\begin{itemize}
\item Ein unendlich breiter horizontaler Streifen
\item Untere Grenze: die reelle Achse ($y = 0$)
\item Obere Grenze: die Gerade $y = 5$
\end{itemize}

\textbf{(b)} Äußeres des Kreises um $-2 + \im$ mit Radius 3:
\begin{itemize}
\item Mittelpunkt: $z_0 = -2 + \im$
\item Radius: $r = 3$
\item Die Menge umfasst alle Punkte außerhalb und auf dem Kreisrand
\end{itemize}

\textbf{(c)} Bereich zwischen den Hyperbelästen:
\begin{itemize}
\item Hyperbel: $y^2 - 3x^2 = 1$
\item Scheitelpunkte: $(0, \pm 1)$
\item Die Menge ist der Bereich zwischen den beiden Ästen (einschließlich der Hyperbeläste selbst)
\end{itemize}

\end{document}