\begin{prob}
  Beweise für alle $n \in \N$ und alle $x \in \R$ die beiden folgenden
  Identitäten:
  \begin{equation*}
    \begin{split}
      \cos(nx) &= \sum_{k=0}^{\floor{\frac{n}{2}}} (-1)^k\binom{n}{2k} \cos(x)^{n-2k}\sin(x)^{2k}\\
      \sin(nx) &= \sum_{k=0}^{\floor{\frac{n}{2}}} (-1)^k\binom{n}{2k+1} \cos(x)^{n-2k-1}\sin(x)^{2k+1}
    \end{split}
  \end{equation*}
\end{prob}

\textbf{Lösung:}

Wir beweisen beide Identitäten gleichzeitig mithilfe der komplexen Exponentialfunktion und des Satzes von De Moivre.

Nach dem Satz von De Moivre gilt für alle $n \in \mathbb{N}$ und $x \in \mathbb{R}$:
\begin{equation}
(\cos x + i\sin x)^n = \cos(nx) + i\sin(nx)
\end{equation}

Wir entwickeln die linke Seite mit dem binomischen Lehrsatz:
\begin{equation}
(\cos x + i\sin x)^n = \sum_{k=0}^{n} \binom{n}{k} \cos^{n-k}(x) \cdot (i\sin x)^k
\end{equation}

Dies können wir umschreiben zu:
\begin{equation}
(\cos x + i\sin x)^n = \sum_{k=0}^{n} \binom{n}{k} \cos^{n-k}(x) \cdot i^k \cdot \sin^k(x)
\end{equation}

Nun betrachten wir die Potenzen von $i$:
\begin{itemize}
  \item $i^0 = 1$
  \item $i^1 = i$
  \item $i^2 = -1$
  \item $i^3 = -i$
  \item $i^4 = 1$
  \item ...
\end{itemize}

Allgemein gilt:
\begin{align}
  i^{4m} &= 1\\
  i^{4m+1} &= i\\
  i^{4m+2} &= -1\\
  i^{4m+3} &= -i
\end{align}

Insbesondere haben wir:
\begin{align}
  i^{2k} &= (-1)^k \quad \text{(reell)}\\
  i^{2k+1} &= i \cdot (-1)^k \quad \text{(rein imaginär)}
\end{align}

Trennen wir nun die Summe in gerade und ungerade Indizes auf:
\begin{equation}
\begin{split}
(\cos x + i\sin x)^n &= \sum_{k=0}^{n} \binom{n}{k} \cos^{n-k}(x) \cdot i^k \cdot \sin^k(x)\\
&= \sum_{k=0}^{\lfloor n/2 \rfloor} \binom{n}{2k} \cos^{n-2k}(x) \cdot i^{2k} \cdot \sin^{2k}(x)\\
&\quad + \sum_{k=0}^{\lfloor (n-1)/2 \rfloor} \binom{n}{2k+1} \cos^{n-2k-1}(x) \cdot i^{2k+1} \cdot \sin^{2k+1}(x)
\end{split}
\end{equation}

Setzen wir $i^{2k} = (-1)^k$ und $i^{2k+1} = i(-1)^k$ ein:
\begin{equation}
\begin{split}
(\cos x + i\sin x)^n &= \sum_{k=0}^{\lfloor n/2 \rfloor} (-1)^k \binom{n}{2k} \cos^{n-2k}(x) \sin^{2k}(x)\\
&\quad + i \sum_{k=0}^{\lfloor (n-1)/2 \rfloor} (-1)^k \binom{n}{2k+1} \cos^{n-2k-1}(x) \sin^{2k+1}(x)
\end{split}
\end{equation}

Da $(\cos x + i\sin x)^n = \cos(nx) + i\sin(nx)$, erhalten wir durch Vergleich von Real- und Imaginärteil:

\textbf{Realteil:}
\begin{equation}
\cos(nx) = \sum_{k=0}^{\lfloor n/2 \rfloor} (-1)^k \binom{n}{2k} \cos^{n-2k}(x) \sin^{2k}(x)
\end{equation}

\textbf{Imaginärteil:}
\begin{equation}
\sin(nx) = \sum_{k=0}^{\lfloor (n-1)/2 \rfloor} (-1)^k \binom{n}{2k+1} \cos^{n-2k-1}(x) \sin^{2k+1}(x)
\end{equation}

Für die zweite Formel bemerken wir, dass für $n$ gerade gilt: $\lfloor (n-1)/2 \rfloor = n/2 - 1$, und für $n$ ungerade: $\lfloor (n-1)/2 \rfloor = (n-1)/2$. In beiden Fällen ist dies gleich $\lfloor n/2 \rfloor$, wenn wir beachten, dass für $k = \lfloor n/2 \rfloor$ und gerades $n$ der Index $2k+1 = n+1$ größer als $n$ ist und somit $\binom{n}{2k+1} = 0$ gilt.

Daher können wir die zweite Formel auch schreiben als:
\begin{equation}
\sin(nx) = \sum_{k=0}^{\lfloor n/2 \rfloor} (-1)^k \binom{n}{2k+1} \cos^{n-2k-1}(x) \sin^{2k+1}(x)
\end{equation}

Damit sind beide Identitäten bewiesen. $\square$