\documentclass{article}
\usepackage[utf8]{inputenc}
\usepackage{amsmath}
\usepackage{amssymb}
\usepackage{amsthm}

\newtheorem{prob}{Problem}

\begin{document}

% Aufgabe
\subsection*{Aufgabe}
Berechne den Wert von $\zeta(2)=\sum_{n=1}^\infty n^{-2}$ mit dem
Integralvergleichskriterium bis auf zwei Stellen hinter dem Komma genau.

\subsection*{Lösung}

Wir wollen $\zeta(2) = \sum_{n=1}^{\infty} \frac{1}{n^2}$ mit dem Integralvergleichskriterium auf zwei Nachkommastellen genau berechnen.

Das Integralvergleichskriterium besagt: Sei $f: [1,\infty) \to \mathbb{R}$ eine monoton fallende, positive Funktion. Dann gilt für alle $n \geq 1$:
\[
\int_{n+1}^{\infty} f(x) \, dx \leq \sum_{k=n+1}^{\infty} f(k) \leq \int_{n}^{\infty} f(x) \, dx
\]

Für $f(x) = \frac{1}{x^2}$ berechnen wir zunächst das Integral:
\[
\int_{n}^{\infty} \frac{1}{x^2} \, dx = \left[-\frac{1}{x}\right]_{n}^{\infty} = 0 - \left(-\frac{1}{n}\right) = \frac{1}{n}
\]

Sei $S_n = \sum_{k=1}^{n} \frac{1}{k^2}$ die $n$-te Partialsumme und $R_n = \sum_{k=n+1}^{\infty} \frac{1}{k^2}$ der Reihenrest. Dann ist $\zeta(2) = S_n + R_n$.

Aus dem Integralvergleichskriterium folgt:
\[
\frac{1}{n+1} \leq R_n \leq \frac{1}{n}
\]

Somit erhalten wir die Abschätzung:
\[
S_n + \frac{1}{n+1} \leq \zeta(2) \leq S_n + \frac{1}{n}
\]

Für eine Genauigkeit von zwei Nachkommastellen benötigen wir:
\[
\left(S_n + \frac{1}{n}\right) - \left(S_n + \frac{1}{n+1}\right) = \frac{1}{n} - \frac{1}{n+1} = \frac{n+1-n}{n(n+1)} = \frac{1}{n(n+1)} < 0.01
\]

Dies ist äquivalent zu $n(n+1) > 100$. Für $n = 10$ haben wir $10 \cdot 11 = 110 > 100$, also ist $n = 10$ ausreichend.

Berechnung von $S_{10}$:
\begin{align}
S_{10} &= \sum_{k=1}^{10} \frac{1}{k^2} \\
&= 1 + \frac{1}{4} + \frac{1}{9} + \frac{1}{16} + \frac{1}{25} + \frac{1}{36} + \frac{1}{49} + \frac{1}{64} + \frac{1}{81} + \frac{1}{100} \\
&= 1 + 0.25 + 0.1111... + 0.0625 + 0.04 + 0.0278... + 0.0204... + 0.0156... + 0.0123... + 0.01 \\
&\approx 1.5498
\end{align}

Die genaue Berechnung ergibt:
\[
S_{10} = \frac{1968329}{1270080} \approx 1.549767731
\]

Damit erhalten wir die Schranken:
\begin{align}
\text{Untere Schranke: } & S_{10} + \frac{1}{11} \approx 1.549767731 + 0.090909091 \approx 1.640676822 \\
\text{Obere Schranke: } & S_{10} + \frac{1}{10} \approx 1.549767731 + 0.1 \approx 1.649767731
\end{align}

Die Differenz der Schranken beträgt:
\[
1.649767731 - 1.640676822 = 0.009090909 < 0.01 \quad \checkmark
\]

Da beide Schranken zwischen 1.64 und 1.65 liegen und ihr Mittelwert $\frac{1.640676822 + 1.649767731}{2} \approx 1.645$ beträgt, können wir mit Sicherheit sagen:

\[
\boxed{\zeta(2) \approx 1.65}
\]

auf zwei Nachkommastellen genau.

\textit{Anmerkung:} Der exakte Wert ist $\zeta(2) = \frac{\pi^2}{6} \approx 1.6449$, was unsere Approximation bestätigt.

\end{document}