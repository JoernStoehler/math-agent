\documentclass{article}
\usepackage[utf8]{inputenc}
\usepackage{amsmath}
\usepackage{amssymb}
\usepackage{amsthm}

\newtheorem{prob}{Aufgabe}
\newtheorem{wrongsol}{Falsche L\"osung}[prob]
\newtheorem{sol}{L\"osung}


\begin{document}

% Aufgabe
\subsection*{Aufgabe}

\begin{prob} 
Berechne die folgenden Integrale f\"ur $a<b$ mittels partieller Integration:

\begin{enumerate}
\item $\int_a^b x\sin(x)\,dx$;
\item $\int_a^b x^2\log(x)\,dx$, wobei $0<a$;
\item $\int_a^b x^n e^x\,dx$, wobei $n\in\mathbb{N}$;
\item $\int_a^b \sin(x)\cos(x)\,dx$;
\item $\int_a^b e^x\sin(x)\,dx$;
\item $\int_a^b \arctan(x)\,dx$.
\end{enumerate}
\end{prob}

\begin{prob}
Betrachte die Integrale
$$
   I_m := \int_0^{\pi/2}\sin^mx\,dx,\qquad m\in\mathbb{N}_0.
$$
\begin{enumerate}
\item Beweise durch partielle Integration die Rekursionsformel
$$
   I_m = \frac{m-1}{m}I_{m-2},\qquad m\geq 2.
$$
\item Folgere aus (a) folgende Formeln f\"ur $n\in\mathbb{N}_0$:
\begin{align*}
   I_{2n} &= \frac{(2n-1)(2n-3)\cdots 3\cdot 1}{(2n-0)(2n-2)\cdots 4\cdot 2}\cdot\frac{\pi}{2},\qquad
   I_{2n+1} = \frac{(2n-0)(2n-2)\cdots 4\cdot 2}{(2n+1)(2n-1)\cdots 5\cdot 3}\cdot 1.
\end{align*}
\item Beweise: 
Die Folge $(I_m)$ ist streng monoton fallend und $\lim_{m\to\infty}I_m=0$.
\item Zeige $\lim_{n\to\infty}\frac{I_{2n+1}}{I_{2n}}=1$ und folgere
daraus die {\em Wallis'sche Produktdarstellung}
$$
   \frac{\pi}{2} = \prod_{n=1}^\infty \frac{4n^2}{4n^2-1} = \frac{2}{1} \cdot \frac{2}{3} \cdot \frac{4}{3} \cdot \frac{4}{5} \cdot \frac{6}{5} \cdot \frac{6}{7} \cdot \cdots
$$
\end{enumerate}
\end{prob}

\begin{prob}
Berechne die folgenden Integrale mittels Substitution:
\begin{enumerate}
\item $\int_a^b x\sin(x^2+1)\,dx$ mit dem Ansatz $u=x^2+1$;
\item $\int_a^b \sqrt{1-x^2}\,dx$ f\"ur $-1\leq a<b\leq1$ mit dem Ansatz $x=\sin(t)$;
\item $\int_a^b\frac{1}{1-x^2}\,dx$ f\"ur $-1<a<b<1$ mit dem Ansatz $x=\tanh(t)$;
\item Zeige mit Hilfe einer Substitution, dass für jede Riemann-integrierbare Funktion $f:[a,b]\to\mathbb{R}$ gilt:
$$
\int_a^b\frac{f'(x)}{f(x)}\,dx = \log(|f(b)|)- \log(|f(a)|).
$$
 und bestimme damit das Integral $\int_a^b\tan(x)\,dx$ für $[a,b]\subset(-\pi/2,\pi/2)$.
\end{enumerate} 
\end{prob}

\begin{prob}
Bestimme mittels Polynomdivision und Partialbruchzerlegung Stammfunktionen der folgenden Funktionen:
$$
(a)\ \frac{x^5}{x-1};\quad
(b)\ \frac{x}{x^3+x^2-x-1};\quad
(c)\ \frac{x}{x^3-x^2+x-1}.
$$

\textit{Vorsicht!} Die Nenner bei (b) und (c) sind verschieden!
\end{prob}

\begin{prob}
Berechne den Wert von $\zeta(2)=\sum_{n=1}^\infty n^{-2}$ mit dem
Integralvergleichskriterium bis auf zwei Stellen hinter dem Komma genau.
\end{prob}

\subsection*{Lösung}

\textbf{Aufgabe 1:} Partielle Integration

Wir verwenden die Formel der partiellen Integration: $\int u\,dv = uv - \int v\,du$.

(a) $\int_a^b x\sin(x)\,dx$

Wir wählen $u = x$, $dv = \sin(x)\,dx$. Dann ist $du = dx$ und $v = -\cos(x)$.

\begin{align}
\int_a^b x\sin(x)\,dx &= \Big[x \cdot (-\cos(x))\Big]_a^b - \int_a^b (-\cos(x))\,dx \\
&= -b\cos(b) + a\cos(a) + \int_a^b \cos(x)\,dx \\
&= -b\cos(b) + a\cos(a) + \Big[\sin(x)\Big]_a^b \\
&= -b\cos(b) + a\cos(a) + \sin(b) - \sin(a)
\end{align}

(b) $\int_a^b x^2\log(x)\,dx$ mit $0 < a$

Wir wählen $u = \log(x)$, $dv = x^2\,dx$. Dann ist $du = \frac{1}{x}\,dx$ und $v = \frac{x^3}{3}$.

\begin{align}
\int_a^b x^2\log(x)\,dx &= \Big[\log(x) \cdot \frac{x^3}{3}\Big]_a^b - \int_a^b \frac{x^3}{3} \cdot \frac{1}{x}\,dx \\
&= \frac{b^3\log(b)}{3} - \frac{a^3\log(a)}{3} - \int_a^b \frac{x^2}{3}\,dx \\
&= \frac{b^3\log(b)}{3} - \frac{a^3\log(a)}{3} - \Big[\frac{x^3}{9}\Big]_a^b \\
&= \frac{b^3\log(b)}{3} - \frac{a^3\log(a)}{3} - \frac{b^3}{9} + \frac{a^3}{9}
\end{align}

(c) $\int_a^b x^n e^x\,dx$ mit $n \in \mathbb{N}$

Wir verwenden partielle Integration mit $u = x^n$, $dv = e^x\,dx$. Dann ist $du = nx^{n-1}\,dx$ und $v = e^x$.

\begin{align}
\int_a^b x^n e^x\,dx &= \Big[x^n e^x\Big]_a^b - \int_a^b nx^{n-1}e^x\,dx \\
&= b^n e^b - a^n e^a - n\int_a^b x^{n-1}e^x\,dx
\end{align}

Dies gibt uns eine Rekursionsformel. Durch wiederholte Anwendung erhalten wir:
\begin{align}
\int_a^b x^n e^x\,dx = e^x \sum_{k=0}^{n} (-1)^k \frac{n!}{(n-k)!} x^{n-k} \Big|_a^b
\end{align}

(d) $\int_a^b \sin(x)\cos(x)\,dx$

Wir wählen $u = \sin(x)$, $dv = \cos(x)\,dx$. Dann ist $du = \cos(x)\,dx$ und $v = \sin(x)$.

\begin{align}
\int_a^b \sin(x)\cos(x)\,dx &= \Big[\sin^2(x)\Big]_a^b - \int_a^b \sin(x)\cos(x)\,dx
\end{align}

Daraus folgt:
\begin{align}
2\int_a^b \sin(x)\cos(x)\,dx &= \sin^2(b) - \sin^2(a) \\
\int_a^b \sin(x)\cos(x)\,dx &= \frac{\sin^2(b) - \sin^2(a)}{2}
\end{align}

Alternativ können wir die Identität $\sin(x)\cos(x) = \frac{1}{2}\sin(2x)$ nutzen:
\begin{align}
\int_a^b \sin(x)\cos(x)\,dx = \frac{1}{2}\int_a^b \sin(2x)\,dx = -\frac{1}{4}\Big[\cos(2x)\Big]_a^b = \frac{\cos(2a) - \cos(2b)}{4}
\end{align}

(e) $\int_a^b e^x\sin(x)\,dx$

Wir wählen $u = \sin(x)$, $dv = e^x\,dx$. Dann ist $du = \cos(x)\,dx$ und $v = e^x$.

\begin{align}
\int_a^b e^x\sin(x)\,dx &= \Big[e^x\sin(x)\Big]_a^b - \int_a^b e^x\cos(x)\,dx
\end{align}

Für das zweite Integral wählen wir $u = \cos(x)$, $dv = e^x\,dx$. Dann ist $du = -\sin(x)\,dx$ und $v = e^x$.

\begin{align}
\int_a^b e^x\cos(x)\,dx &= \Big[e^x\cos(x)\Big]_a^b + \int_a^b e^x\sin(x)\,dx
\end{align}

Setzen wir dies in die erste Gleichung ein:
\begin{align}
\int_a^b e^x\sin(x)\,dx &= e^b\sin(b) - e^a\sin(a) - \Big(e^b\cos(b) - e^a\cos(a) + \int_a^b e^x\sin(x)\,dx\Big) \\
2\int_a^b e^x\sin(x)\,dx &= e^b\sin(b) - e^a\sin(a) - e^b\cos(b) + e^a\cos(a) \\
\int_a^b e^x\sin(x)\,dx &= \frac{e^b(\sin(b) - \cos(b)) - e^a(\sin(a) - \cos(a))}{2}
\end{align}

(f) $\int_a^b \arctan(x)\,dx$

Wir wählen $u = \arctan(x)$, $dv = dx$. Dann ist $du = \frac{1}{1+x^2}\,dx$ und $v = x$.

\begin{align}
\int_a^b \arctan(x)\,dx &= \Big[x\arctan(x)\Big]_a^b - \int_a^b \frac{x}{1+x^2}\,dx \\
&= b\arctan(b) - a\arctan(a) - \frac{1}{2}\int_a^b \frac{2x}{1+x^2}\,dx \\
&= b\arctan(b) - a\arctan(a) - \frac{1}{2}\Big[\log(1+x^2)\Big]_a^b \\
&= b\arctan(b) - a\arctan(a) - \frac{1}{2}\log(1+b^2) + \frac{1}{2}\log(1+a^2)
\end{align}

\textbf{Aufgabe 2:} Wallis'sche Produktdarstellung

(a) Beweis der Rekursionsformel $I_m = \frac{m-1}{m}I_{m-2}$ für $m \geq 2$

Wir verwenden partielle Integration mit $u = \sin^{m-1}(x)$ und $dv = \sin(x)\,dx$. 
Dann ist $du = (m-1)\sin^{m-2}(x)\cos(x)\,dx$ und $v = -\cos(x)$.

\begin{align}
I_m &= \int_0^{\pi/2} \sin^m(x)\,dx \\
&= \Big[-\sin^{m-1}(x)\cos(x)\Big]_0^{\pi/2} + \int_0^{\pi/2} (m-1)\sin^{m-2}(x)\cos^2(x)\,dx \\
&= 0 + (m-1)\int_0^{\pi/2} \sin^{m-2}(x)\cos^2(x)\,dx
\end{align}

Da $\cos^2(x) = 1 - \sin^2(x)$:
\begin{align}
I_m &= (m-1)\int_0^{\pi/2} \sin^{m-2}(x)(1-\sin^2(x))\,dx \\
&= (m-1)\int_0^{\pi/2} \sin^{m-2}(x)\,dx - (m-1)\int_0^{\pi/2} \sin^m(x)\,dx \\
&= (m-1)I_{m-2} - (m-1)I_m
\end{align}

Umformen ergibt:
\begin{align}
I_m + (m-1)I_m &= (m-1)I_{m-2} \\
mI_m &= (m-1)I_{m-2} \\
I_m &= \frac{m-1}{m}I_{m-2}
\end{align}

(b) Formeln für $I_{2n}$ und $I_{2n+1}$

Wir berechnen zunächst $I_0$ und $I_1$:
\begin{align}
I_0 &= \int_0^{\pi/2} 1\,dx = \frac{\pi}{2} \\
I_1 &= \int_0^{\pi/2} \sin(x)\,dx = \Big[-\cos(x)\Big]_0^{\pi/2} = 1
\end{align}

Für gerade Indizes $I_{2n}$:
\begin{align}
I_{2n} &= \frac{2n-1}{2n}I_{2n-2} = \frac{2n-1}{2n} \cdot \frac{2n-3}{2n-2}I_{2n-4} = \ldots \\
&= \frac{(2n-1)(2n-3)\cdots 3 \cdot 1}{2n(2n-2)\cdots 4 \cdot 2} \cdot I_0 \\
&= \frac{(2n-1)(2n-3)\cdots 3 \cdot 1}{2n(2n-2)\cdots 4 \cdot 2} \cdot \frac{\pi}{2}
\end{align}

Für ungerade Indizes $I_{2n+1}$:
\begin{align}
I_{2n+1} &= \frac{2n}{2n+1}I_{2n-1} = \frac{2n}{2n+1} \cdot \frac{2n-2}{2n-1}I_{2n-3} = \ldots \\
&= \frac{2n(2n-2)\cdots 4 \cdot 2}{(2n+1)(2n-1)\cdots 5 \cdot 3} \cdot I_1 \\
&= \frac{2n(2n-2)\cdots 4 \cdot 2}{(2n+1)(2n-1)\cdots 5 \cdot 3} \cdot 1
\end{align}

(c) Monotonie und Grenzwert

Für $0 < x < \pi/2$ gilt $0 < \sin(x) < 1$, also $\sin^{m+1}(x) < \sin^m(x)$.
Daher ist
$$I_{m+1} = \int_0^{\pi/2} \sin^{m+1}(x)\,dx < \int_0^{\pi/2} \sin^m(x)\,dx = I_m$$

Die Folge $(I_m)$ ist also streng monoton fallend.

Für den Grenzwert: Da $0 \leq \sin(x) \leq 1$ für $x \in [0, \pi/2]$, gilt
$$0 \leq I_m = \int_0^{\pi/2} \sin^m(x)\,dx \leq \int_0^{\pi/2} \sin^m(\pi/2)\,dx = \frac{\pi}{2} \cdot 0 = 0 \text{ für } m \to \infty$$

Genauer: Für $0 < x < \pi/2$ ist $\sin(x) < 1$, also $\lim_{m \to \infty} \sin^m(x) = 0$.
Mit dominierter Konvergenz folgt $\lim_{m \to \infty} I_m = 0$.

(d) Wallis'sche Produktdarstellung

Wir betrachten das Verhältnis:
\begin{align}
\frac{I_{2n+1}}{I_{2n}} &= \frac{\frac{2n(2n-2)\cdots 4 \cdot 2}{(2n+1)(2n-1)\cdots 5 \cdot 3}}{\frac{(2n-1)(2n-3)\cdots 3 \cdot 1}{2n(2n-2)\cdots 4 \cdot 2} \cdot \frac{\pi}{2}} \\
&= \frac{[2n(2n-2)\cdots 4 \cdot 2]^2}{(2n+1)(2n-1)\cdots 5 \cdot 3 \cdot (2n-1)(2n-3)\cdots 3 \cdot 1} \cdot \frac{2}{\pi} \\
&= \frac{[2n(2n-2)\cdots 4 \cdot 2]^2}{(2n+1)[(2n-1)]^2[(2n-3)]^2\cdots [3]^2 \cdot 1} \cdot \frac{2}{\pi}
\end{align}

Da $I_m$ streng monoton fallend ist, gilt $I_{2n+2} < I_{2n+1} < I_{2n}$, also
$$\frac{I_{2n+2}}{I_{2n}} < \frac{I_{2n+1}}{I_{2n}} < 1$$

Nun ist $\frac{I_{2n+2}}{I_{2n}} = \frac{2n+1}{2n+2}$, also
$$\frac{2n+1}{2n+2} < \frac{I_{2n+1}}{I_{2n}} < 1$$

Für $n \to \infty$ geht $\frac{2n+1}{2n+2} \to 1$, also $\lim_{n \to \infty} \frac{I_{2n+1}}{I_{2n}} = 1$.

Aus der obigen Formel folgt dann:
\begin{align}
\frac{\pi}{2} &= \lim_{n \to \infty} \frac{[2n(2n-2)\cdots 4 \cdot 2]^2}{(2n+1)[(2n-1)]^2[(2n-3)]^2\cdots [3]^2 \cdot 1} \\
&= \lim_{n \to \infty} \prod_{k=1}^n \frac{4k^2}{4k^2-1} \\
&= \prod_{n=1}^\infty \frac{4n^2}{4n^2-1}
\end{align}

\textbf{Aufgabe 3:} Integration durch Substitution

(a) $\int_a^b x\sin(x^2+1)\,dx$ mit $u = x^2 + 1$

Wir substituieren $u = x^2 + 1$, also $du = 2x\,dx$ und $x\,dx = \frac{1}{2}du$.

Die Grenzen transformieren sich zu: $u(a) = a^2 + 1$ und $u(b) = b^2 + 1$.

\begin{align}
\int_a^b x\sin(x^2+1)\,dx &= \int_{a^2+1}^{b^2+1} \sin(u) \cdot \frac{1}{2}\,du \\
&= \frac{1}{2}\int_{a^2+1}^{b^2+1} \sin(u)\,du \\
&= \frac{1}{2}\Big[-\cos(u)\Big]_{a^2+1}^{b^2+1} \\
&= \frac{1}{2}\Big(-\cos(b^2+1) + \cos(a^2+1)\Big) \\
&= \frac{\cos(a^2+1) - \cos(b^2+1)}{2}
\end{align}

(b) $\int_a^b \sqrt{1-x^2}\,dx$ für $-1 \leq a < b \leq 1$ mit $x = \sin(t)$

Wir substituieren $x = \sin(t)$, also $dx = \cos(t)\,dt$ und $\sqrt{1-x^2} = \sqrt{1-\sin^2(t)} = |\cos(t)| = \cos(t)$ 
(da wir $t \in [-\pi/2, \pi/2]$ wählen können).

Die Grenzen sind: $t(a) = \arcsin(a)$ und $t(b) = \arcsin(b)$.

\begin{align}
\int_a^b \sqrt{1-x^2}\,dx &= \int_{\arcsin(a)}^{\arcsin(b)} \cos(t) \cdot \cos(t)\,dt \\
&= \int_{\arcsin(a)}^{\arcsin(b)} \cos^2(t)\,dt \\
&= \int_{\arcsin(a)}^{\arcsin(b)} \frac{1 + \cos(2t)}{2}\,dt \\
&= \frac{1}{2}\Big[t + \frac{\sin(2t)}{2}\Big]_{\arcsin(a)}^{\arcsin(b)} \\
&= \frac{1}{2}\Big(\arcsin(b) - \arcsin(a) + \frac{\sin(2\arcsin(b)) - \sin(2\arcsin(a))}{2}\Big)
\end{align}

Mit $\sin(2\arcsin(x)) = 2x\sqrt{1-x^2}$ erhalten wir:
\begin{align}
\int_a^b \sqrt{1-x^2}\,dx = \frac{1}{2}\Big(\arcsin(b) - \arcsin(a) + b\sqrt{1-b^2} - a\sqrt{1-a^2}\Big)
\end{align}

(c) $\int_a^b \frac{1}{1-x^2}\,dx$ für $-1 < a < b < 1$ mit $x = \tanh(t)$

Wir substituieren $x = \tanh(t)$, also $dx = \mathrm{sech}^2(t)\,dt = (1-\tanh^2(t))\,dt = (1-x^2)\,dt$.

Die Grenzen sind: $t(a) = \mathrm{artanh}(a)$ und $t(b) = \mathrm{artanh}(b)$.

\begin{align}
\int_a^b \frac{1}{1-x^2}\,dx &= \int_{\mathrm{artanh}(a)}^{\mathrm{artanh}(b)} \frac{1}{1-\tanh^2(t)} \cdot (1-\tanh^2(t))\,dt \\
&= \int_{\mathrm{artanh}(a)}^{\mathrm{artanh}(b)} 1\,dt \\
&= \mathrm{artanh}(b) - \mathrm{artanh}(a) \\
&= \frac{1}{2}\log\left(\frac{1+b}{1-b}\right) - \frac{1}{2}\log\left(\frac{1+a}{1-a}\right) \\
&= \frac{1}{2}\log\left(\frac{(1+b)(1-a)}{(1-b)(1+a)}\right)
\end{align}

(d) Zeige $\int_a^b \frac{f'(x)}{f(x)}\,dx = \log(|f(b)|) - \log(|f(a)|)$

Wir substituieren $u = f(x)$, also $du = f'(x)\,dx$.

Die Grenzen sind: $u(a) = f(a)$ und $u(b) = f(b)$.

\begin{align}
\int_a^b \frac{f'(x)}{f(x)}\,dx &= \int_{f(a)}^{f(b)} \frac{1}{u}\,du \\
&= \Big[\log(|u|)\Big]_{f(a)}^{f(b)} \\
&= \log(|f(b)|) - \log(|f(a)|)
\end{align}

Für $\int_a^b \tan(x)\,dx$ mit $[a,b] \subset (-\pi/2, \pi/2)$:

\begin{align}
\int_a^b \tan(x)\,dx &= \int_a^b \frac{\sin(x)}{\cos(x)}\,dx \\
&= -\int_a^b \frac{-\sin(x)}{\cos(x)}\,dx \\
&= -\int_a^b \frac{(\cos(x))'}{\cos(x)}\,dx \\
&= -\Big(\log(|\cos(b)|) - \log(|\cos(a)|)\Big) \\
&= \log(|\cos(a)|) - \log(|\cos(b)|)
\end{align}

Da $\cos(x) > 0$ für $x \in (-\pi/2, \pi/2)$:
$$\int_a^b \tan(x)\,dx = \log(\cos(a)) - \log(\cos(b)) = \log\left(\frac{\cos(a)}{\cos(b)}\right)$$

\textbf{Aufgabe 4:} Partialbruchzerlegung

(a) $\frac{x^5}{x-1}$

Zuerst führen wir Polynomdivision durch:
\begin{align}
x^5 : (x-1) = x^4 + x^3 + x^2 + x + 1 + \frac{1}{x-1}
\end{align}

Verifikation: $(x-1)(x^4 + x^3 + x^2 + x + 1) = x^5 - 1$, also $x^5 = (x-1)(x^4 + x^3 + x^2 + x + 1) + 1$.

Die Stammfunktion ist:
\begin{align}
\int \frac{x^5}{x-1}\,dx &= \int \left(x^4 + x^3 + x^2 + x + 1 + \frac{1}{x-1}\right)\,dx \\
&= \frac{x^5}{5} + \frac{x^4}{4} + \frac{x^3}{3} + \frac{x^2}{2} + x + \log|x-1| + C
\end{align}

(b) $\frac{x}{x^3+x^2-x-1}$

Wir faktorisieren den Nenner:
$x^3 + x^2 - x - 1 = x^2(x+1) - (x+1) = (x+1)(x^2-1) = (x+1)(x-1)(x+1) = (x+1)^2(x-1)$

Partialbruchzerlegung:
$$\frac{x}{(x+1)^2(x-1)} = \frac{A}{x+1} + \frac{B}{(x+1)^2} + \frac{C}{x-1}$$

Multiplizieren mit dem Nenner:
$$x = A(x+1)(x-1) + B(x-1) + C(x+1)^2$$

Einsetzen spezieller Werte:
- $x = 1$: $1 = 0 + 0 + C \cdot 4$, also $C = \frac{1}{4}$
- $x = -1$: $-1 = 0 + B \cdot (-2) + 0$, also $B = \frac{1}{2}$
- $x = 0$: $0 = A \cdot 1 \cdot (-1) + B \cdot (-1) + C \cdot 1 = -A - \frac{1}{2} + \frac{1}{4}$, also $A = -\frac{1}{4}$

Die Stammfunktion ist:
\begin{align}
\int \frac{x}{(x+1)^2(x-1)}\,dx &= -\frac{1}{4}\int \frac{1}{x+1}\,dx + \frac{1}{2}\int \frac{1}{(x+1)^2}\,dx + \frac{1}{4}\int \frac{1}{x-1}\,dx \\
&= -\frac{1}{4}\log|x+1| - \frac{1}{2} \cdot \frac{1}{x+1} + \frac{1}{4}\log|x-1| + C \\
&= \frac{1}{4}\log\left|\frac{x-1}{x+1}\right| - \frac{1}{2(x+1)} + C
\end{align}

(c) $\frac{x}{x^3-x^2+x-1}$

Wir faktorisieren den Nenner:
$x^3 - x^2 + x - 1 = x^2(x-1) + (x-1) = (x-1)(x^2+1)$

Da $x^2 + 1$ über $\mathbb{R}$ irreduzibel ist, lautet die Partialbruchzerlegung:
$$\frac{x}{(x-1)(x^2+1)} = \frac{A}{x-1} + \frac{Bx+C}{x^2+1}$$

Multiplizieren mit dem Nenner:
$$x = A(x^2+1) + (Bx+C)(x-1)$$

Einsetzen spezieller Werte:
- $x = 1$: $1 = A \cdot 2 + 0$, also $A = \frac{1}{2}$
- $x = 0$: $0 = A \cdot 1 + C \cdot (-1) = \frac{1}{2} - C$, also $C = \frac{1}{2}$
- $x = i$: $i = 0 + (Bi + \frac{1}{2})(i-1) = Bi(i-1) + \frac{1}{2}(i-1) = -B - Bi + \frac{i-1}{2}$

Koeffizientenvergleich bei $x^2$: $0 = A + B = \frac{1}{2} + B$, also $B = -\frac{1}{2}$

Die Stammfunktion ist:
\begin{align}
\int \frac{x}{(x-1)(x^2+1)}\,dx &= \frac{1}{2}\int \frac{1}{x-1}\,dx + \int \frac{-\frac{1}{2}x + \frac{1}{2}}{x^2+1}\,dx \\
&= \frac{1}{2}\log|x-1| - \frac{1}{2}\int \frac{x}{x^2+1}\,dx + \frac{1}{2}\int \frac{1}{x^2+1}\,dx \\
&= \frac{1}{2}\log|x-1| - \frac{1}{4}\log(x^2+1) + \frac{1}{2}\arctan(x) + C
\end{align}

\textbf{Aufgabe 5:} Berechnung von $\zeta(2)$

Das Integralvergleichskriterium besagt:
$$\int_k^\infty \frac{1}{x^2}\,dx \leq \sum_{n=k}^\infty \frac{1}{n^2} \leq \frac{1}{k^2} + \int_k^\infty \frac{1}{x^2}\,dx$$

Wir berechnen:
$$\int_k^\infty \frac{1}{x^2}\,dx = \Big[-\frac{1}{x}\Big]_k^\infty = \frac{1}{k}$$

Also gilt für die Teilsumme $S_N = \sum_{n=1}^N \frac{1}{n^2}$:
$$S_N + \frac{1}{N+1} \leq \zeta(2) \leq S_N + \frac{1}{N^2} + \frac{1}{N}$$

Wir berechnen einige Teilsummen:
\begin{align}
S_1 &= 1 \\
S_2 &= 1 + \frac{1}{4} = 1.25 \\
S_3 &= 1.25 + \frac{1}{9} \approx 1.361 \\
S_4 &= 1.361 + \frac{1}{16} = 1.4235 \\
S_5 &= 1.4235 + \frac{1}{25} = 1.4635 \\
S_{10} &\approx 1.5498 \\
S_{20} &\approx 1.5962 \\
S_{50} &\approx 1.6251 \\
S_{100} &\approx 1.6350 \\
S_{200} &\approx 1.6399 \\
S_{500} &\approx 1.6429
\end{align}

Mit $N = 500$ erhalten wir:
$$1.6429 + \frac{1}{501} \leq \zeta(2) \leq 1.6429 + \frac{1}{500^2} + \frac{1}{500}$$
$$1.6449 \leq \zeta(2) \leq 1.6449$$

Also ist $\zeta(2) \approx 1.64$ auf zwei Dezimalstellen genau.

(Der exakte Wert ist $\zeta(2) = \frac{\pi^2}{6} \approx 1.6449$.)

\end{document}