\documentclass{article}
\usepackage[utf8]{inputenc}
\usepackage{amsmath}
\usepackage{amssymb}
\usepackage{graphicx}

\newcommand{\R}{\mathbb{R}}

\begin{document}

% Aufgabe
\subsection*{Aufgabe}
Betrachte die Kurve in der Ebene $f:[0,2\pi)\to\R^2$, 
$$
   f(t) := \bigl(\sin(2t)\cos t,\sin(2t)\sin t\bigr).
$$
(a) Zeichne die Kurve $f$ und bestimme ihre singulären Punkte und
Doppelpunkte. 

(b) Berechne die Länge von $f$. 

\subsection*{Lösung}

\textbf{Teil (a): Zeichnung und Analyse der Kurve}

Die Kurve kann in der Form $f(t) = \sin(2t) \cdot (\cos t, \sin t)$ geschrieben werden, was zeigt, dass es sich um eine radiale Modulation der Einheitskreisrichtung handelt.

Die Kurve bildet eine vierblättrige Rosette mit dem Ursprung als Zentrum. Die vier Blätter erstrecken sich in die Richtungen der positiven und negativen $x$- und $y$-Achsen, wobei jedes Blatt die maximale Ausdehnung von $|\sin(2t)| = 1$ erreicht.

\textbf{Bestimmung der singulären Punkte:}

Ein Punkt ist singulär, wenn $f'(t) = 0$. Berechnen wir die Ableitung:

Für $f(t) = (\sin(2t)\cos t, \sin(2t)\sin t)$ erhalten wir mittels Produktregel:
\begin{align}
x'(t) &= \frac{d}{dt}[\sin(2t)\cos t] = 2\cos(2t)\cos t - \sin(2t)\sin t\\
y'(t) &= \frac{d}{dt}[\sin(2t)\sin t] = 2\cos(2t)\sin t + \sin(2t)\cos t
\end{align}

Mit den Additionstheoremen können wir diese umformen zu:
\begin{align}
x'(t) &= 2\cos(2t)\cos t - \sin(2t)\sin t = \cos t + \cos(3t)\\
y'(t) &= 2\cos(2t)\sin t + \sin(2t)\cos t = \sin t + \sin(3t)
\end{align}

Für einen singulären Punkt muss gelten: $x'(t) = 0$ und $y'(t) = 0$, also:
$$\cos t + \cos(3t) = 0 \quad \text{und} \quad \sin t + \sin(3t) = 0$$

Dies bedeutet $\cos t = -\cos(3t)$ und $\sin t = -\sin(3t)$, was äquivalent ist zu:
$$e^{it} = -e^{i3t}$$

Dies führt zu $t = 3t + \pi \pmod{2\pi}$, also $-2t = \pi \pmod{2\pi}$.

Die Lösungen im Intervall $[0, 2\pi)$ sind:
$$t = \frac{\pi}{2} \quad \text{und} \quad t = \frac{3\pi}{2}$$

An diesen Stellen ist $f(\pi/2) = f(3\pi/2) = (0,0)$, der Ursprung.

\textbf{Bestimmung der Doppelpunkte:}

Die Kurve passiert den Ursprung bei $t \in \{0, \pi/2, \pi, 3\pi/2\}$ (wenn $\sin(2t) = 0$).

Der Ursprung ist also ein vierfacher Punkt. Die Tangentenvektoren an diesen Stellen sind:
\begin{align}
f'(0) &= (2, 0)\\
f'(\pi/2) &= (0, -2)\\
f'(\pi) &= (-2, 0)\\
f'(3\pi/2) &= (0, 2)
\end{align}

Zwei dieser Durchgänge ($t = \pi/2$ und $t = 3\pi/2$) haben $f'(t) = 0$, sind also singulär.

\textbf{Zusammenfassung:}
\begin{itemize}
\item Singuläre Punkte: Der Ursprung $(0,0)$ bei $t = \pi/2$ und $t = 3\pi/2$
\item Doppelpunkte: Der Ursprung $(0,0)$ ist ein vierfacher Punkt (bei $t = 0, \pi/2, \pi, 3\pi/2$)
\end{itemize}

\textbf{Teil (b): Berechnung der Kurvenlänge}

Die Länge der Kurve berechnet sich durch:
$$L = \int_0^{2\pi} |f'(t)| \, dt$$

wobei $|f'(t)|^2 = (x'(t))^2 + (y'(t))^2$.

Wir haben bereits gezeigt:
\begin{align}
x'(t) &= \cos t + \cos(3t)\\
y'(t) &= \sin t + \sin(3t)
\end{align}

Berechnen wir $|f'(t)|^2$:
\begin{align}
|f'(t)|^2 &= (\cos t + \cos(3t))^2 + (\sin t + \sin(3t))^2\\
&= \cos^2 t + 2\cos t \cos(3t) + \cos^2(3t) + \sin^2 t + 2\sin t \sin(3t) + \sin^2(3t)\\
&= 1 + 1 + 2(\cos t \cos(3t) + \sin t \sin(3t))\\
&= 2 + 2\cos(3t - t)\\
&= 2 + 2\cos(2t)\\
&= 2(1 + \cos(2t))\\
&= 4\cos^2 t
\end{align}

Daher ist $|f'(t)| = 2|\cos t|$.

Die Länge berechnet sich zu:
$$L = \int_0^{2\pi} 2|\cos t| \, dt$$

Da $\cos t$ das Vorzeichen bei $t = \pi/2$ und $t = 3\pi/2$ wechselt, müssen wir das Integral aufteilen:
\begin{align}
L &= \int_0^{\pi/2} 2\cos t \, dt + \int_{\pi/2}^{\pi} 2(-\cos t) \, dt + \int_{\pi}^{3\pi/2} 2(-\cos t) \, dt + \int_{3\pi/2}^{2\pi} 2\cos t \, dt\\
&= 2\sin t \Big|_0^{\pi/2} - 2\sin t \Big|_{\pi/2}^{\pi} - 2\sin t \Big|_{\pi}^{3\pi/2} + 2\sin t \Big|_{3\pi/2}^{2\pi}\\
&= 2(1 - 0) - 2(0 - 1) - 2(-1 - 0) + 2(0 - (-1))\\
&= 2 + 2 + 2 + 2\\
&= 8
\end{align}

Die Länge der Kurve beträgt also $L = 8$.

\end{document}