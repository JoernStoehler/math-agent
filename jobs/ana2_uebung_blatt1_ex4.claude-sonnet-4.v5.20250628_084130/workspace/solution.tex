\documentclass{article}
\usepackage[utf8]{inputenc}
\usepackage{amsmath}
\usepackage{amssymb}
\usepackage{amsthm}
% No enumerate package needed for basic enumeration

\begin{document}

% Aufgabe
\subsection*{Aufgabe}
Berechne die folgenden Integrale mittels Substitution:
\begin{enumerate}
\item $\int_a^b x\sin(x^2+1)\,dx$ mit dem Ansatz $u=x^2+1$;
\item $\int_a^b \sqrt{1-x^2}\,dx$ für $-1\leq a<b\leq1$ mit dem Ansatz $x=\sin(t)$;
\item $\int_a^b\frac{1}{1-x^2}\,dx$ für $-1<a<b<1$ mit dem Ansatz $x=\tanh(t)$;
\item Zeige mit Hilfe einer Substitution, dass für jede Riemann-integrierbare Funktion $f:[a,b]\to\mathbb{R}$ gilt:
$$
\int_a^b\frac{f'(x)}{f(x)}\,dx = \log(|f(b)|)- \log(|f(a)|).
$$
 und bestimme damit das Integral $\int_a^b\tan(x)\,dx$ für $[a,b]\subset(-\pi/2,\pi/2)$.
\end{enumerate} 

\subsection*{Lösung}

\textbf{(a)} Wir berechnen $\int_a^b x\sin(x^2+1)\,dx$ mit der Substitution $u=x^2+1$.

Für die Substitution gilt:
\begin{align}
u &= x^2 + 1\\
\frac{du}{dx} &= 2x\\
dx &= \frac{du}{2x}
\end{align}

Die Integrationsgrenzen transformieren sich wie folgt:
\begin{align}
x = a &\implies u = a^2 + 1\\
x = b &\implies u = b^2 + 1
\end{align}

Setzen wir die Substitution in das Integral ein:
\begin{align}
\int_a^b x\sin(x^2+1)\,dx &= \int_{a^2+1}^{b^2+1} x\sin(u)\,\frac{du}{2x}\\
&= \int_{a^2+1}^{b^2+1} \frac{1}{2}\sin(u)\,du\\
&= \frac{1}{2}\left[-\cos(u)\right]_{a^2+1}^{b^2+1}\\
&= \frac{1}{2}\left(-\cos(b^2+1) + \cos(a^2+1)\right)\\
&= \frac{1}{2}\left(\cos(a^2+1) - \cos(b^2+1)\right)
\end{align}

\textbf{(b)} Wir berechnen $\int_a^b \sqrt{1-x^2}\,dx$ für $-1\leq a<b\leq1$ mit der Substitution $x=\sin(t)$.

Für die Substitution gilt:
\begin{align}
x &= \sin(t)\\
\frac{dx}{dt} &= \cos(t)\\
dx &= \cos(t)\,dt
\end{align}

Da $x\in[-1,1]$ und $\sin$ bijektiv auf $[-\pi/2,\pi/2]$ ist, wählen wir $t\in[-\pi/2,\pi/2]$.
Die Integrationsgrenzen transformieren sich:
\begin{align}
x = a &\implies t = \arcsin(a)\\
x = b &\implies t = \arcsin(b)
\end{align}

Setzen wir die Substitution ein:
\begin{align}
\int_a^b \sqrt{1-x^2}\,dx &= \int_{\arcsin(a)}^{\arcsin(b)} \sqrt{1-\sin^2(t)}\,\cos(t)\,dt\\
&= \int_{\arcsin(a)}^{\arcsin(b)} \sqrt{\cos^2(t)}\,\cos(t)\,dt
\end{align}

Da $t\in[-\pi/2,\pi/2]$, ist $\cos(t)\geq 0$, also $\sqrt{\cos^2(t)} = \cos(t)$. Damit:
\begin{align}
\int_a^b \sqrt{1-x^2}\,dx &= \int_{\arcsin(a)}^{\arcsin(b)} \cos^2(t)\,dt\\
&= \int_{\arcsin(a)}^{\arcsin(b)} \frac{1+\cos(2t)}{2}\,dt\\
&= \frac{1}{2}\left[t + \frac{\sin(2t)}{2}\right]_{\arcsin(a)}^{\arcsin(b)}\\
&= \frac{1}{2}\left[t + \frac{2\sin(t)\cos(t)}{2}\right]_{\arcsin(a)}^{\arcsin(b)}\\
&= \frac{1}{2}\left[t + \sin(t)\cos(t)\right]_{\arcsin(a)}^{\arcsin(b)}
\end{align}

Mit $\sin(t) = x$ und $\cos(t) = \sqrt{1-x^2}$ erhalten wir:
\begin{align}
\int_a^b \sqrt{1-x^2}\,dx &= \frac{1}{2}\left[\arcsin(x) + x\sqrt{1-x^2}\right]_a^b\\
&= \frac{1}{2}\left(\arcsin(b) + b\sqrt{1-b^2} - \arcsin(a) - a\sqrt{1-a^2}\right)
\end{align}

\textbf{(c)} Wir berechnen $\int_a^b\frac{1}{1-x^2}\,dx$ für $-1<a<b<1$ mit der Substitution $x=\tanh(t)$.

Für die Substitution gilt:
\begin{align}
x &= \tanh(t) = \frac{e^t - e^{-t}}{e^t + e^{-t}}\\
\frac{dx}{dt} &= \frac{1}{\cosh^2(t)} = 1 - \tanh^2(t) = 1 - x^2\\
dx &= (1-x^2)\,dt
\end{align}

Die Integrationsgrenzen transformieren sich:
\begin{align}
x = a &\implies t = \text{artanh}(a) = \frac{1}{2}\log\left(\frac{1+a}{1-a}\right)\\
x = b &\implies t = \text{artanh}(b) = \frac{1}{2}\log\left(\frac{1+b}{1-b}\right)
\end{align}

Setzen wir die Substitution ein:
\begin{align}
\int_a^b\frac{1}{1-x^2}\,dx &= \int_{\text{artanh}(a)}^{\text{artanh}(b)}\frac{1}{1-x^2}\,(1-x^2)\,dt\\
&= \int_{\text{artanh}(a)}^{\text{artanh}(b)} 1\,dt\\
&= \left[t\right]_{\text{artanh}(a)}^{\text{artanh}(b)}\\
&= \text{artanh}(b) - \text{artanh}(a)\\
&= \frac{1}{2}\log\left(\frac{1+b}{1-b}\right) - \frac{1}{2}\log\left(\frac{1+a}{1-a}\right)\\
&= \frac{1}{2}\log\left(\frac{(1+b)(1-a)}{(1-b)(1+a)}\right)
\end{align}

\textbf{(d)} Wir zeigen mit Hilfe einer Substitution, dass für jede Riemann-integrierbare Funktion $f:[a,b]\to\mathbb{R}$ mit $f(x) \neq 0$ auf $[a,b]$ gilt:
$$\int_a^b\frac{f'(x)}{f(x)}\,dx = \log(|f(b)|)- \log(|f(a)|).$$

Wir verwenden die Substitution $u = f(x)$:
\begin{align}
u &= f(x)\\
\frac{du}{dx} &= f'(x)\\
dx &= \frac{du}{f'(x)}
\end{align}

Die Integrationsgrenzen transformieren sich:
\begin{align}
x = a &\implies u = f(a)\\
x = b &\implies u = f(b)
\end{align}

Setzen wir die Substitution ein:
\begin{align}
\int_a^b\frac{f'(x)}{f(x)}\,dx &= \int_{f(a)}^{f(b)}\frac{f'(x)}{u}\,\frac{du}{f'(x)}\\
&= \int_{f(a)}^{f(b)}\frac{1}{u}\,du
\end{align}

Falls $f(x) > 0$ auf $[a,b]$:
$$\int_{f(a)}^{f(b)}\frac{1}{u}\,du = \left[\log(u)\right]_{f(a)}^{f(b)} = \log(f(b)) - \log(f(a)) = \log(|f(b)|) - \log(|f(a)|)$$

Falls $f(x) < 0$ auf $[a,b]$, dann ist $f(a) < 0$ und $f(b) < 0$. Mit der Substitution $v = -u$ erhalten wir:
$$\int_{f(a)}^{f(b)}\frac{1}{u}\,du = -\int_{-f(a)}^{-f(b)}\frac{1}{v}\,dv = -[\log(v)]_{-f(a)}^{-f(b)} = -\log(-f(b)) + \log(-f(a)) = \log(|f(b)|) - \log(|f(a)|)$$

Falls $f$ das Vorzeichen wechselt, müssen wir das Integral an den Nullstellen aufteilen. Die Formel gilt dann für jeden Teilbereich, in dem $f$ das Vorzeichen nicht wechselt.

Nun bestimmen wir $\int_a^b\tan(x)\,dx$ für $[a,b]\subset(-\pi/2,\pi/2)$:

Es gilt $\tan(x) = \frac{\sin(x)}{\cos(x)}$. Wir können schreiben:
$$\tan(x) = \frac{\sin(x)}{\cos(x)} = -\frac{(\cos(x))'}{\cos(x)}$$

da $(\cos(x))' = -\sin(x)$. Mit der obigen Formel und $f(x) = \cos(x)$ erhalten wir:
\begin{align}
\int_a^b\tan(x)\,dx &= -\int_a^b\frac{(\cos(x))'}{\cos(x)}\,dx\\
&= -[\log(|\cos(x)|)]_a^b\\
&= -\log(|\cos(b)|) + \log(|\cos(a)|)\\
&= \log\left(\frac{|\cos(a)|}{|\cos(b)|}\right)
\end{align}

Da $[a,b]\subset(-\pi/2,\pi/2)$, ist $\cos(x) > 0$ auf $[a,b]$, also:
$$\int_a^b\tan(x)\,dx = \log\left(\frac{\cos(a)}{\cos(b)}\right) = \log(\cos(a)) - \log(\cos(b))$$

\end{document}