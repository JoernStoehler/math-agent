\documentclass{article}
\usepackage{amsmath,amsthm,amssymb}
\usepackage{array}

\newtheorem{theorem}{Theorem}
\newtheorem{lemma}[theorem]{Lemma}
\newtheorem{remark}[theorem]{Remark}

\begin{document}

\section*{Solution to Linear Algebra - Sheet 3, Exercise 1}

\textbf{Given:} The matrix
$$A = \begin{pmatrix}
2 & 1 & 0 \\
1 & 2 & 1 \\
0 & 1 & 2
\end{pmatrix}$$

\subsection*{Part (a): Finding all eigenvalues}

To find the eigenvalues, we compute the characteristic polynomial $\det(A - \lambda I) = 0$.

$$A - \lambda I = \begin{pmatrix}
2-\lambda & 1 & 0 \\
1 & 2-\lambda & 1 \\
0 & 1 & 2-\lambda
\end{pmatrix}$$

We calculate the determinant by expanding along the first row:
\begin{align}
\det(A - \lambda I) &= (2-\lambda)\begin{vmatrix}
2-\lambda & 1 \\
1 & 2-\lambda
\end{vmatrix} - 1\begin{vmatrix}
1 & 1 \\
0 & 2-\lambda
\end{vmatrix} + 0\\
&= (2-\lambda)[(2-\lambda)^2 - 1] - 1[1(2-\lambda) - 0]\\
&= (2-\lambda)[(2-\lambda)^2 - 1] - (2-\lambda)\\
&= (2-\lambda)[(2-\lambda)^2 - 1 - 1]\\
&= (2-\lambda)[(2-\lambda)^2 - 2]
\end{align}

Setting this equal to zero:
$$(2-\lambda)[(2-\lambda)^2 - 2] = 0$$

This gives us:
- $\lambda_1 = 2$ (from the first factor)
- $(2-\lambda)^2 = 2$, which gives $2-\lambda = \pm\sqrt{2}$
  - $\lambda_2 = 2 - \sqrt{2}$
  - $\lambda_3 = 2 + \sqrt{2}$

\textbf{The eigenvalues are:} $\lambda_1 = 2$, $\lambda_2 = 2 - \sqrt{2}$, $\lambda_3 = 2 + \sqrt{2}$

\subsection*{Part (b): Finding eigenspaces}

(Work in progress - computing eigenvectors...)

\end{document}