\begin{exercise}[Binomialkoeffzienten]
  Zeige die folgenden Eigenschaften des Binomialkoeffizienten:
  \begin{enumerate}
  \item Berechne die Binomialkoeffizienten \(\binom{n}{k}\) für \(n = 1, \dots,
    4\) und \( k = 0, \dots ,n\).
  \item Wir verallgemeinern den Begriff des Binomialkoeffizienten wie folgt. Für
    $a \in \R$ definieren wir rekursiv $\binom{a}{0} = 1$ und $\binom{a}{n + 1}
    = \frac{a - n}{n + 1} \binom{a}{n}$. Zeige, dass dann immer noch folgende
    Beziehung gilt
    \[
      \binom{a}{n} + \binom{a}{n + 1}
      =
      \binom{a + 1}{n + 1}
    \]
    \item Für natürliche Zahlen $n \leq m$ gilt $\binom{m}{n} = \frac{m!}{n! (m
        - n)!} = \binom{m}{m - n}$.
  \end{enumerate}
  Tipp: Lemma 1.1 aus der Vorlesung: \(\binom{n}{k} + \binom{n}{k-1} =
  \binom{n+1}{k}\).
\end{exercise}
