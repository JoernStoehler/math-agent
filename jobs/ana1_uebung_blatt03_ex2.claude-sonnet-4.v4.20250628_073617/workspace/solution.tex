\documentclass{article}
\usepackage[utf8]{inputenc}
\usepackage{amsmath}
\usepackage{amssymb}
\usepackage{amsthm}

\begin{document}

% Aufgabe
\subsection*{Aufgabe: Binomialkoeffizienten}
Zeige die folgenden Eigenschaften des Binomialkoeffizienten:
\begin{enumerate}
\item Berechne die Binomialkoeffizienten $\binom{n}{k}$ für $n = 1, \dots, 4$ und $k = 0, \dots ,n$.
\item Wir verallgemeinern den Begriff des Binomialkoeffizienten wie folgt. Für
  $a \in \mathbb{R}$ definieren wir rekursiv $\binom{a}{0} = 1$ und $\binom{a}{n + 1}
  = \frac{a - n}{n + 1} \binom{a}{n}$. Zeige, dass dann immer noch folgende
  Beziehung gilt
  \[
    \binom{a}{n} + \binom{a}{n + 1}
    =
    \binom{a + 1}{n + 1}
  \]
  \item Für natürliche Zahlen $n \leq m$ gilt $\binom{m}{n} = \frac{m!}{n! (m
      - n)!} = \binom{m}{m - n}$.
\end{enumerate}
Tipp: Lemma 1.1 aus der Vorlesung: $\binom{n}{k} + \binom{n}{k-1} = \binom{n+1}{k}$.

\subsection*{Lösung}

\textbf{Teil a)} Berechnung der Binomialkoeffizienten für $n = 1, \dots, 4$ und $k = 0, \dots ,n$.

Wir verwenden die Formel $\binom{n}{k} = \frac{n!}{k!(n-k)!}$ zur Berechnung:

Für $n = 1$:
\begin{align}
\binom{1}{0} &= \frac{1!}{0! \cdot 1!} = \frac{1}{1 \cdot 1} = 1 \\
\binom{1}{1} &= \frac{1!}{1! \cdot 0!} = \frac{1}{1 \cdot 1} = 1
\end{align}

Für $n = 2$:
\begin{align}
\binom{2}{0} &= \frac{2!}{0! \cdot 2!} = \frac{2}{1 \cdot 2} = 1 \\
\binom{2}{1} &= \frac{2!}{1! \cdot 1!} = \frac{2}{1 \cdot 1} = 2 \\
\binom{2}{2} &= \frac{2!}{2! \cdot 0!} = \frac{2}{2 \cdot 1} = 1
\end{align}

Für $n = 3$:
\begin{align}
\binom{3}{0} &= \frac{3!}{0! \cdot 3!} = \frac{6}{1 \cdot 6} = 1 \\
\binom{3}{1} &= \frac{3!}{1! \cdot 2!} = \frac{6}{1 \cdot 2} = 3 \\
\binom{3}{2} &= \frac{3!}{2! \cdot 1!} = \frac{6}{2 \cdot 1} = 3 \\
\binom{3}{3} &= \frac{3!}{3! \cdot 0!} = \frac{6}{6 \cdot 1} = 1
\end{align}

Für $n = 4$:
\begin{align}
\binom{4}{0} &= \frac{4!}{0! \cdot 4!} = \frac{24}{1 \cdot 24} = 1 \\
\binom{4}{1} &= \frac{4!}{1! \cdot 3!} = \frac{24}{1 \cdot 6} = 4 \\
\binom{4}{2} &= \frac{4!}{2! \cdot 2!} = \frac{24}{2 \cdot 2} = 6 \\
\binom{4}{3} &= \frac{4!}{3! \cdot 1!} = \frac{24}{6 \cdot 1} = 4 \\
\binom{4}{4} &= \frac{4!}{4! \cdot 0!} = \frac{24}{24 \cdot 1} = 1
\end{align}

Die berechneten Werte bilden die ersten 5 Zeilen des Pascalschen Dreiecks.

\textbf{Teil b)} Beweis der Beziehung $\binom{a}{n} + \binom{a}{n + 1} = \binom{a + 1}{n + 1}$ für die verallgemeinerten Binomialkoeffizienten.

Gegeben ist die rekursive Definition:
\begin{itemize}
\item $\binom{a}{0} = 1$
\item $\binom{a}{n + 1} = \frac{a - n}{n + 1} \binom{a}{n}$
\end{itemize}

Aus der rekursiven Definition können wir die explizite Formel herleiten:
$$\binom{a}{n} = \frac{a(a-1)(a-2)\cdots(a-n+1)}{n!}$$

Dies lässt sich durch vollständige Induktion zeigen. Für $n = 0$ gilt $\binom{a}{0} = 1 = \frac{1}{0!}$. Für den Induktionsschritt:
$$\binom{a}{n+1} = \frac{a-n}{n+1} \binom{a}{n} = \frac{a-n}{n+1} \cdot \frac{a(a-1)\cdots(a-n+1)}{n!} = \frac{a(a-1)\cdots(a-n)}{(n+1)!}$$

Nun beweisen wir die geforderte Beziehung:
\begin{align}
\binom{a}{n} + \binom{a}{n+1} &= \frac{a(a-1)\cdots(a-n+1)}{n!} + \frac{a(a-1)\cdots(a-n)}{(n+1)!} \\
&= \frac{a(a-1)\cdots(a-n+1)}{n!} + \frac{a(a-1)\cdots(a-n+1)(a-n)}{(n+1)!} \\
&= \frac{a(a-1)\cdots(a-n+1)}{n!} \left(1 + \frac{a-n}{n+1}\right) \\
&= \frac{a(a-1)\cdots(a-n+1)}{n!} \cdot \frac{n+1+a-n}{n+1} \\
&= \frac{a(a-1)\cdots(a-n+1)}{n!} \cdot \frac{a+1}{n+1} \\
&= \frac{(a+1)a(a-1)\cdots(a-n+1)}{(n+1)!}
\end{align}

Andererseits gilt nach der expliziten Formel:
$$\binom{a+1}{n+1} = \frac{(a+1)(a)(a-1)\cdots((a+1)-(n+1)+1)}{(n+1)!} = \frac{(a+1)a(a-1)\cdots(a-n+1)}{(n+1)!}$$

Damit ist die Beziehung bewiesen.

\textbf{Teil c)} Beweis der Symmetrie der Binomialkoeffizienten: Für natürliche Zahlen $n \leq m$ gilt $\binom{m}{n} = \frac{m!}{n!(m-n)!} = \binom{m}{m-n}$.

Der erste Teil der Gleichung $\binom{m}{n} = \frac{m!}{n!(m-n)!}$ ist die Definition des Binomialkoeffizienten für natürliche Zahlen.

Für den zweiten Teil berechnen wir:
\begin{align}
\binom{m}{m-n} &= \frac{m!}{(m-n)!(m-(m-n))!} \\
&= \frac{m!}{(m-n)!n!} \\
&= \frac{m!}{n!(m-n)!} \\
&= \binom{m}{n}
\end{align}

Die Gleichheit folgt direkt aus der Kommutativität der Multiplikation im Nenner. Diese Symmetrie-Eigenschaft bedeutet, dass die Anzahl der Möglichkeiten, $n$ Objekte aus $m$ auszuwählen, gleich der Anzahl der Möglichkeiten ist, $m-n$ Objekte aus $m$ auszuwählen (da die Auswahl von $n$ Objekten gleichbedeutend mit dem Nicht-Auswählen von $m-n$ Objekten ist).

\end{document}