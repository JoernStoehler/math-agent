\documentclass{article}
\usepackage[utf8]{inputenc}
\usepackage{amsmath}
\usepackage{amssymb}
\usepackage{enumerate}
% \usepackage[ngerman]{babel}

\begin{document}

% Aufgabe
\subsection*{Aufgabe}
Für ein Dreieck in der Ebene sind die folgenden drei Punkte gleich:

\begin{enumerate}[(a)]
    \item der Schwerpunkt der Ecken mit gleichen Massen;
    \item der Schwerpunkt des Dreiecks mit homogener Massenverteilung;
    \item der Schnittpunkt der Seitenhalbierenden.
\end{enumerate}

\subsection*{Lösung}

Sei das Dreieck durch die Ecken $A$, $B$ und $C$ gegeben. Wir verwenden Ortsvektoren $\vec{a}$, $\vec{b}$ und $\vec{c}$ für die entsprechenden Punkte.

\textbf{Teil (a): Schwerpunkt der Ecken mit gleichen Massen}

Bei gleichen Massen $m$ an jeder Ecke ist der Schwerpunkt:
\begin{align}
S_a = \frac{m\vec{a} + m\vec{b} + m\vec{c}}{m + m + m} = \frac{\vec{a} + \vec{b} + \vec{c}}{3}
\end{align}

\textbf{Teil (c): Schnittpunkt der Seitenhalbierenden}

Eine Seitenhalbierende ist die Verbindungslinie von einer Ecke zum Mittelpunkt der gegenüberliegenden Seite.

Der Mittelpunkt der Seite $BC$ ist:
\begin{align}
M_{BC} = \frac{\vec{b} + \vec{c}}{2}
\end{align}

Die Seitenhalbierende von $A$ nach $M_{BC}$ kann parametrisiert werden als:
\begin{align}
\vec{s}_A(t) = \vec{a} + t(M_{BC} - \vec{a}) = \vec{a} + t\left(\frac{\vec{b} + \vec{c}}{2} - \vec{a}\right) = (1-t)\vec{a} + \frac{t}{2}\vec{b} + \frac{t}{2}\vec{c}
\end{align}

Analog erhalten wir die Seitenhalbierende von $B$ nach $M_{AC} = \frac{\vec{a} + \vec{c}}{2}$:
\begin{align}
\vec{s}_B(s) = \vec{b} + s(M_{AC} - \vec{b}) = \vec{b} + s\left(\frac{\vec{a} + \vec{c}}{2} - \vec{b}\right) = \frac{s}{2}\vec{a} + (1-s)\vec{b} + \frac{s}{2}\vec{c}
\end{align}

Für den Schnittpunkt setzen wir $\vec{s}_A(t) = \vec{s}_B(s)$:
\begin{align}
(1-t)\vec{a} + \frac{t}{2}\vec{b} + \frac{t}{2}\vec{c} = \frac{s}{2}\vec{a} + (1-s)\vec{b} + \frac{s}{2}\vec{c}
\end{align}

Da $\vec{a}$, $\vec{b}$ und $\vec{c}$ die Ecken eines nicht-degenerierten Dreiecks sind, sind die Vektoren $\vec{b} - \vec{a}$ und $\vec{c} - \vec{a}$ linear unabhängig. Daher müssen die Koeffizienten bei jedem Vektor übereinstimmen:

Koeffizientenvergleich:
\begin{align}
\text{bei } \vec{a}: \quad 1-t &= \frac{s}{2}\\
\text{bei } \vec{b}: \quad \frac{t}{2} &= 1-s\\
\text{bei } \vec{c}: \quad \frac{t}{2} &= \frac{s}{2}
\end{align}

Aus der dritten Gleichung folgt $t = s$. Einsetzen in die erste Gleichung:
\begin{align}
1-t = \frac{t}{2} \quad \Rightarrow \quad 2-2t = t \quad \Rightarrow \quad t = \frac{2}{3}
\end{align}

Also ist $t = s = \frac{2}{3}$. Der Schnittpunkt der Seitenhalbierenden ist:
\begin{align}
S_c = \vec{s}_A\left(\frac{2}{3}\right) = \frac{1}{3}\vec{a} + \frac{1}{3}\vec{b} + \frac{1}{3}\vec{c} = \frac{\vec{a} + \vec{b} + \vec{c}}{3}
\end{align}

\textbf{Teil (b): Schwerpunkt des Dreiecks mit homogener Massenverteilung}

Für den Schwerpunkt einer homogenen Fläche verwenden wir die Integralformel:
\begin{align}
S_b = \frac{1}{A} \iint_{\Delta} \vec{r} \, dA
\end{align}

wobei $A$ die Fläche des Dreiecks und $\Delta$ das Dreieck selbst ist.

Wir parametrisieren das Dreieck mit baryzentrischen Koordinaten:
\begin{align}
\vec{r}(u,v) = u\vec{a} + v\vec{b} + (1-u-v)\vec{c}, \quad u \geq 0, v \geq 0, u+v \leq 1
\end{align}

Die Jacobi-Determinante dieser Transformation ist:
\begin{align}
\left|\frac{\partial(x,y)}{\partial(u,v)}\right| = \left|(\vec{a} - \vec{c}) \times (\vec{b} - \vec{c})\right| = 2A
\end{align}

wobei $A$ die Fläche des Dreiecks ist. Damit ergibt sich:
\begin{align}
S_b &= \frac{1}{A} \int_0^1 \int_0^{1-u} \vec{r}(u,v) \cdot 2A \, dv \, du\\
&= 2 \int_0^1 \int_0^{1-u} [u\vec{a} + v\vec{b} + (1-u-v)\vec{c}] \, dv \, du
\end{align}

Wir berechnen die Integrale komponentenweise:
\begin{align}
\int_0^1 \int_0^{1-u} u \, dv \, du &= \int_0^1 u(1-u) \, du = \left[\frac{u^2}{2} - \frac{u^3}{3}\right]_0^1 = \frac{1}{2} - \frac{1}{3} = \frac{1}{6}\\
\int_0^1 \int_0^{1-u} v \, dv \, du &= \int_0^1 \frac{(1-u)^2}{2} \, du = \frac{1}{2}\left[-\frac{(1-u)^3}{3}\right]_0^1 = \frac{1}{6}\\
\int_0^1 \int_0^{1-u} (1-u-v) \, dv \, du &= \int_0^1 \left[(1-u)^2 - \frac{(1-u)^2}{2}\right] \, du = \frac{1}{2} \int_0^1 (1-u)^2 \, du = \frac{1}{6}
\end{align}

Somit erhalten wir:
\begin{align}
S_b = 2\left[\frac{1}{6}\vec{a} + \frac{1}{6}\vec{b} + \frac{1}{6}\vec{c}\right] = \frac{\vec{a} + \vec{b} + \vec{c}}{3}
\end{align}

\textbf{Zusammenfassung:}

Wir haben gezeigt, dass alle drei Punkte durch denselben Ausdruck gegeben sind:
\begin{align}
S_a = S_b = S_c = \frac{\vec{a} + \vec{b} + \vec{c}}{3}
\end{align}

Dies ist der zentrale Schwerpunkt des Dreiecks, der sich genau im Drittel jeder Seitenhalbierenden von der jeweiligen Ecke aus befindet.

\end{document}