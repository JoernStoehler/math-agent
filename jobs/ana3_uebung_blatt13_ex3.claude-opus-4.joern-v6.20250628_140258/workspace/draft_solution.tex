\documentclass{article}
\usepackage[utf8]{inputenc}
\usepackage{amsmath}
\usepackage{amssymb}
\usepackage{amsthm}
\usepackage{enumerate}
\usepackage{tikz}

\DeclareMathOperator{\vol}{vol}
\newcommand{\R}{\mathbb{R}}
\newcommand{\del}{\partial}

\begin{document}

% Aufgabe
\subsection*{Aufgabe}
Sei $M \subset \R^3$ eine durch das Einheits-Normalenfeld $\nu \colon M \mapsto \R^3$ orientierte Hyperfläche und
$$
f \colon \R^2 \supset V \to M\cap U \colon (u,v) \mapsto f(u,v)
$$
eine positiv orientierte Karte. 
\begin{enumerate}[label = (\alph*)]
	\item Man beweise folgende Formeln für das Einheits-Normalenfeld und das Pullback der induzierten Volumenform $\vol_M$ unter $f$: 
	$$
	\nu \circ f = \frac{\frac{\del f}{\del u}\times\frac{\del f}{\del v}}{\bigl|\frac{\del f}{\del u}\times\frac{\del f}{\del v}\bigr|}, \qquad
	f^* \vol_M = \left| \frac{\del f}{\del u}\times\frac{\del f}{\del v} \right| du \wedge dv. 
	$$
	\item Man berechne mit Hilfe der Karte aus Beispiel 7.50(c) im Skript das Einheits-Normalenfeld und den Flächeninhalt (d.h. das $2$-dimensionale Volumen) des Torus $T_{r,R}$. 
\end{enumerate}

\subsection*{Lösung}

\textbf{Teil (a):} Wir beweisen die beiden Formeln.

\textit{Formel für das Einheits-Normalenfeld:}

Da $f: V \to M \cap U$ eine Karte ist, bilden die Tangentialvektoren $\frac{\del f}{\del u}$ und $\frac{\del f}{\del v}$ an jedem Punkt eine Basis des Tangentialraums von $M$ an diesem Punkt. Das Kreuzprodukt $\frac{\del f}{\del u} \times \frac{\del f}{\del v}$ steht senkrecht auf beiden Tangentialvektoren und ist damit ein Normalenvektor zu $M$ an der Stelle $f(u,v)$.

Da $f$ positiv orientiert ist und $\nu$ das Einheits-Normalenfeld ist, das die Orientierung von $M$ bestimmt, zeigt das Kreuzprodukt $\frac{\del f}{\del u} \times \frac{\del f}{\del v}$ in die gleiche Richtung wie $\nu \circ f$. Um das Einheits-Normalenfeld zu erhalten, müssen wir diesen Vektor normieren:

$$\nu \circ f = \frac{\frac{\del f}{\del u}\times\frac{\del f}{\del v}}{\bigl|\frac{\del f}{\del u}\times\frac{\del f}{\del v}\bigr|}$$

\textit{Formel für das Pullback der Volumenform:}

Die induzierte Volumenform $\vol_M$ auf $M$ kann lokal durch eine 2-Form dargestellt werden. Für eine orientierte Hyperfläche im $\R^3$ mit Karte $f$ gilt, dass die Volumenform das orientierte Flächenelement darstellt.

Das Flächenelement eines Parallelogramms, das von den Vektoren $\frac{\del f}{\del u} du$ und $\frac{\del f}{\del v} dv$ aufgespannt wird, hat den Flächeninhalt
$$\left|\frac{\del f}{\del u} \times \frac{\del f}{\del v}\right| \, du \, dv$$

Da $f$ positiv orientiert ist, ist das Pullback der Volumenform:
$$f^* \vol_M = \left| \frac{\del f}{\del u}\times\frac{\del f}{\del v} \right| du \wedge dv$$

\textbf{Teil (b):} Da ich keinen Zugriff auf das Skript mit Beispiel 7.50(c) habe, verwende ich die Standardparametrisierung des Torus $T_{r,R}$ mit innerem Radius $r$ und äußerem Radius $R$ (wobei $R > r > 0$):

$$f(u,v) = \begin{pmatrix}
(R + r\cos v)\cos u \\
(R + r\cos v)\sin u \\
r\sin v
\end{pmatrix}, \quad (u,v) \in [0,2\pi) \times [0,2\pi)$$

\textit{Berechnung der partiellen Ableitungen:}

$$\frac{\del f}{\del u} = \begin{pmatrix}
-(R + r\cos v)\sin u \\
(R + r\cos v)\cos u \\
0
\end{pmatrix}, \quad
\frac{\del f}{\del v} = \begin{pmatrix}
-r\sin v \cos u \\
-r\sin v \sin u \\
r\cos v
\end{pmatrix}$$

\textit{Berechnung des Kreuzprodukts:}

$$\frac{\del f}{\del u} \times \frac{\del f}{\del v} = \begin{pmatrix}
(R + r\cos v)\cos u \cdot r\cos v \\
(R + r\cos v)\sin u \cdot r\cos v \\
r(R + r\cos v)
\end{pmatrix}$$

Nach Berechnung ergibt sich:
$$\frac{\del f}{\del u} \times \frac{\del f}{\del v} = r(R + r\cos v)\begin{pmatrix}
\cos u \cos v \\
\sin u \cos v \\
\sin v
\end{pmatrix}$$

\textit{Berechnung des Betrags:}

$$\left|\frac{\del f}{\del u} \times \frac{\del f}{\del v}\right| = r(R + r\cos v)\sqrt{\cos^2 u \cos^2 v + \sin^2 u \cos^2 v + \sin^2 v}$$
$$= r(R + r\cos v)\sqrt{\cos^2 v + \sin^2 v} = r(R + r\cos v)$$

\textit{Das Einheits-Normalenfeld:}

$$\nu \circ f = \frac{\frac{\del f}{\del u} \times \frac{\del f}{\del v}}{\left|\frac{\del f}{\del u} \times \frac{\del f}{\del v}\right|} = \begin{pmatrix}
\cos u \cos v \\
\sin u \cos v \\
\sin v
\end{pmatrix}$$

\textit{Der Flächeninhalt des Torus:}

Der Flächeninhalt ist gegeben durch:
$$\text{Fläche} = \int_0^{2\pi} \int_0^{2\pi} \left|\frac{\del f}{\del u} \times \frac{\del f}{\del v}\right| \, dv \, du$$
$$= \int_0^{2\pi} \int_0^{2\pi} r(R + r\cos v) \, dv \, du$$
$$= r \int_0^{2\pi} \left[ Rv + r\sin v \right]_0^{2\pi} \, du$$
$$= r \int_0^{2\pi} 2\pi R \, du$$
$$= 2\pi r R \cdot 2\pi = 4\pi^2 rR$$

\textit{Hinweis:} Diese Lösung basiert auf der Standardparametrisierung eines Torus. Die tatsächliche Lösung hängt von der spezifischen Parametrisierung in Beispiel 7.50(c) ab.

\end{document}