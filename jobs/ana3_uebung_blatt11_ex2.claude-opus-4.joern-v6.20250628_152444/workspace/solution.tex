\documentclass{article}
\usepackage[utf8]{inputenc}
% \usepackage[ngerman]{babel}
\usepackage{amsmath,amssymb,amsthm}
\usepackage{enumerate}
\usepackage{mathtools}

\newcommand{\R}{\mathbb{R}}
% \DeclareMathOperator{\det}{det} % Already defined in amsmath

\begin{document}

% Aufgabe
\subsection*{Aufgabe: Orientierungstreue von Diffeomorphismen}

\begin{enumerate}[(a)]
    \item Sei $\Phi \colon U \stackrel{\cong} \longrightarrow V$ ein Diffeomorphismus zwischen offenen Mengen $U, V \subset \R^n$. Man zeige: Ist $U$ wegzusammenhängend, so ist $\Phi$ entweder orientierungserhaltend oder orientierungsumkehrend.
    \item Sei $U=B_1\cup B_2\subset\R^n$ die disjunkte Vereinigung zweier offener Bälle. Man zeige: Es gibt einen Diffeomorphismus $\Phi \colon U \mapsto U$, der weder orientierungserhaltend noch orientierungsumkehrend ist.
\end{enumerate}

\subsection*{Lösung}

\textbf{Teil (a):}

Wir müssen zeigen, dass für einen Diffeomorphismus $\Phi: U \to V$ auf einer wegzusammenhängenden offenen Menge $U \subset \R^n$ gilt: Entweder ist $\det(D\Phi_x) > 0$ für alle $x \in U$ (orientierungserhaltend) oder $\det(D\Phi_x) < 0$ für alle $x \in U$ (orientierungsumkehrend).

\textbf{Beweis:}

Da $\Phi$ ein Diffeomorphismus ist, ist $\Phi$ insbesondere stetig differenzierbar und die Ableitung $D\Phi_x$ ist für jedes $x \in U$ invertierbar. Daraus folgt, dass $\det(D\Phi_x) \neq 0$ für alle $x \in U$.

Betrachten wir die Funktion
\[
f: U \to \R \setminus \{0\}, \quad f(x) = \det(D\Phi_x).
\]

Diese Funktion ist stetig, da die Determinante eine stetige Funktion der Matrixeinträge ist und die partiellen Ableitungen von $\Phi$ nach Voraussetzung stetig sind.

Da $U$ wegzusammenhängend ist und $f$ stetig ist, muss das Bild $f(U)$ wegzusammenhängend in $\R \setminus \{0\}$ sein.

Nun ist aber $\R \setminus \{0\} = (-\infty, 0) \cup (0, \infty)$ die disjunkte Vereinigung zweier offener Intervalle. Diese Menge ist nicht wegzusammenhängend: Es gibt keinen stetigen Weg, der einen Punkt aus $(-\infty, 0)$ mit einem Punkt aus $(0, \infty)$ verbindet, ohne durch $0$ zu gehen.

Da $f(U)$ wegzusammenhängend sein muss, kann $f(U)$ nicht sowohl Punkte aus $(-\infty, 0)$ als auch aus $(0, \infty)$ enthalten. Daher gilt entweder $f(U) \subset (0, \infty)$ oder $f(U) \subset (-\infty, 0)$.

Dies bedeutet:
\begin{itemize}
    \item Falls $f(U) \subset (0, \infty)$, dann ist $\det(D\Phi_x) > 0$ für alle $x \in U$, also ist $\Phi$ orientierungserhaltend.
    \item Falls $f(U) \subset (-\infty, 0)$, dann ist $\det(D\Phi_x) < 0$ für alle $x \in U$, also ist $\Phi$ orientierungsumkehrend.
\end{itemize}

\textbf{Teil (b):}

Wir konstruieren einen Diffeomorphismus $\Phi: U \to U$ auf $U = B_1 \cup B_2$, der weder orientierungserhaltend noch orientierungsumkehrend ist.

\textbf{Konstruktion:}

Seien $B_1 = B((-2, 0, \ldots, 0), 1)$ und $B_2 = B((2, 0, \ldots, 0), 1)$ zwei disjunkte offene Bälle in $\R^n$ mit Radius $1$ und Mittelpunkten bei $(-2, 0, \ldots, 0)$ bzw. $(2, 0, \ldots, 0)$.

Definiere $\Phi: U \to U$ durch:
\[
\Phi(x) = \begin{cases}
    x & \text{falls } x \in B_1, \\
    (4 - x_1, x_2, \ldots, x_n) & \text{falls } x = (x_1, x_2, \ldots, x_n) \in B_2.
\end{cases}
\]

\textbf{Verifikation, dass $\Phi$ ein Diffeomorphismus ist:}

\begin{enumerate}
    \item \textit{$\Phi$ ist wohldefiniert und bijektiv:} 
    
    Auf $B_1$ ist $\Phi$ die Identität, also trivialerweise bijektiv von $B_1$ nach $B_1$.
    
    Auf $B_2$ ist $\Phi$ eine Spiegelung an der Hyperebene $x_1 = 2$. Für $x = (x_1, x_2, \ldots, x_n) \in B_2$ gilt $|x_1 - 2| < 1$, also $1 < x_1 < 3$. Dann ist $1 < 4 - x_1 < 3$, also $|4 - x_1 - 2| = |2 - x_1| < 1$. Da die anderen Koordinaten unverändert bleiben, folgt $\Phi(x) \in B_2$. Die Umkehrabbildung ist $\Phi^{-1}(y) = (4 - y_1, y_2, \ldots, y_n)$ für $y \in B_2$, also $\Phi|_{B_2}$ bijektiv.
    
    \item \textit{$\Phi$ ist glatt:} 
    
    Auf $B_1$ ist $\Phi(x) = x$ offensichtlich glatt.
    
    Auf $B_2$ ist $\Phi(x_1, \ldots, x_n) = (4 - x_1, x_2, \ldots, x_n)$ ebenfalls glatt als affin-lineare Abbildung.
    
    \item \textit{$\Phi^{-1}$ ist glatt:}
    
    Es gilt $\Phi^{-1} = \Phi$, da sowohl die Identität als auch die Spiegelung selbstinvers sind. Daher ist $\Phi^{-1}$ aus denselben Gründen glatt.
\end{enumerate}

\textbf{Berechnung der Jacobi-Matrix:}

Für $x \in B_1$ ist $D\Phi_x = I$ (Einheitsmatrix), also $\det(D\Phi_x) = 1 > 0$.

Für $x \in B_2$ ist 
\[
D\Phi_x = \begin{pmatrix}
    -1 & 0 & \cdots & 0 \\
    0 & 1 & \cdots & 0 \\
    \vdots & \vdots & \ddots & \vdots \\
    0 & 0 & \cdots & 1
\end{pmatrix},
\]
also $\det(D\Phi_x) = -1 < 0$.

\textbf{Schlussfolgerung:}

Der konstruierte Diffeomorphismus $\Phi$ hat die Eigenschaft, dass $\det(D\Phi_x) > 0$ für alle $x \in B_1$ und $\det(D\Phi_x) < 0$ für alle $x \in B_2$. Daher ist $\Phi$ weder orientierungserhaltend (da die Determinante auf $B_2$ negativ ist) noch orientierungsumkehrend (da die Determinante auf $B_1$ positiv ist).

\end{document}