                \begin{prob}[Tr\"agheitstensor]
%
Der {\em Tr\"agheitstensor} (um den Ursprung) der Massenverteilung
$\mu$ ist die lineare Abbildung $\Theta \colon \R^3 \mapsto \R^3$,
$$
\Theta v := \int_{\R^n} \left( |x|^2v - \langle x, v\rangle x \right) d\mu(x), \qquad v \in \R^3,
$$
wobei das Integral wieder komponentenweise definiert ist. Man zeige:

\begin{enumerate}[label = (\alph*)]
	\item F\"ur $|v| = 1$ ist $\langle v, \Theta v \rangle$ das Tr\"agheitsmoment um
	die Gerade durch den Ursprung in Richtung $v$.
	\item Die lineare Abbildung $\Theta$ ist symmetrisch und positiv
	semidefinit; sie ist positiv definit au"ser im Fall, dass die gesamte
	Masse entlang einer Geraden durch den Ursprung konzentriert ist.
	\item Sind $0 \le \lambda_1 \le \lambda_2 \le \lambda_3$ die der Gr\"o\ss e
	nach geordneten Eigenwerte von $\Theta$, so gilt $\lambda_3 \le \lambda_1 + \lambda_2$.
\end{enumerate}

%------------------------------------------------------------------------------------------------------------------------
\newpage
%------------------------------------------------------------------------------------------------------------------------
\vspace{2mm}
                \end{prob}
