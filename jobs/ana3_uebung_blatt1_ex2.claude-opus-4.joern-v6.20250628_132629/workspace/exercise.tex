\begin{prob}[Charakterisierung einer Algebra]\label{ueb:Algebra}
(a) Zeigen Sie: Eine Algebra $\AA\subset\PP(\Om)$ l\"asst sich \"aquivalent
charakterisieren durch die Axiome

\hskip.5cm (i)' $\varnothing\in\AA$;

\hskip.5cm (ii)' $A\in\AA\Longrightarrow A^c\in\AA$; 

\hskip.5cm (iii)' $A,B\in\AA\Longrightarrow A\cup B\in\AA$.

oder

\hskip.5cm (i)'' $\varnothing,\Om\in\AA$;

\hskip.5cm (ii)'' $A,B\in\AA\Longrightarrow A\Delta B\in\AA$; 

\hskip.5cm (iii)'' $A,B\in\AA\Longrightarrow A\cap B\in\AA$.

(b) Eine Algebra $\AA\subset\PP(\Om)$ bildet mit den Operationen $\cup$
und $\cap$ sowie der Negation $A^c$ eine Boolesche Algebra. 

(c) Eine Algebra $\AA\subset\PP(\Om)$ bildet mit der Addition $\Delta$
und der Multiplikation $\cap$ einen kommutativen Ring mit Eins. 
\end{prob}
