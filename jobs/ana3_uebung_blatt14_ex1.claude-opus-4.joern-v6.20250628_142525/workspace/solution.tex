\documentclass{article}
\usepackage[utf8]{inputenc}
\usepackage{amsmath}
\usepackage{amssymb}
\usepackage{amsthm}

\newcommand{\R}{\mathbb{R}}
\newcommand{\del}{\partial}

\begin{document}

% Aufgabe
\subsection*{Aufgabe (Rand-Normalenfeld)}

Sei $M \subset \R^n$ eine $k$-dimensionale Untermannigfaltigkeit mit Rand. Man zeige:

(a) Für $a \in \del M$ ist $T_a(\del M)$ ein $(k-1)$-dimensionaler Untervektorraum von $T_a M$, $T_a^{\perp} M$ ist ein $(n-k)$-dimensionaler Untervektorraum von $T_a^{\perp}(\del M)$, und es gilt
\[
\R^n = T_a^{\perp} M \oplus (T_a M \cap T_a^{\perp}(\del M)) \oplus T_a(\del M).
\]

(b) Man zeige: Es gibt eine eindeutig bestimmte stetige Abbildung $N: \del M \to \R^n$ mit den folgenden Eigenschaften für jedes $x \in \del M$:
\begin{description}
\item[(i)] $\|N(x)\| = 1$;
\item[(ii)] $N(x) \in T_x M \cap T_x^{\perp}(\del M)$;
\item[(iii)] ist $\phi: V \cap H^k \to U \cap M$ eine Karte für $M$ um $x$ mit $\phi(y) = x$ und $w \in \R^k$ der eindeutig bestimmte Vektor mit $D\phi(y)w = N(x)$, so ist die erste Komponente von $w$ positiv.
\end{description}

Die Abbildung $N: \del M \to \R^n$ heißt das \emph{äußere Normalenfeld zu $\del M$ in $M$}.

\subsection*{Lösung}

\textbf{Teil (a):}

Sei $a \in \del M$. Da $M$ eine $k$-dimensionale Untermannigfaltigkeit mit Rand ist und $\del M$ der Rand von $M$ ist, ist $\del M$ eine $(k-1)$-dimensionale Untermannigfaltigkeit ohne Rand von $\R^n$.

\textbf{Behauptung 1:} $T_a(\del M)$ ist ein $(k-1)$-dimensionaler Untervektorraum von $T_a M$.

\textit{Beweis:} Da $\del M \subset M$ gilt, folgt $T_a(\del M) \subset T_a M$. Da $\del M$ eine $(k-1)$-dimensionale Untermannigfaltigkeit ist, gilt $\dim T_a(\del M) = k-1$.

\textbf{Behauptung 2:} $T_a^{\perp} M$ ist ein $(n-k)$-dimensionaler Untervektorraum von $T_a^{\perp}(\del M)$.

\textit{Beweis:} Da $T_a M$ ein $k$-dimensionaler Untervektorraum von $\R^n$ ist, gilt $\dim T_a^{\perp} M = n-k$. Da $\del M \subset M$ gilt, folgt $T_a(\del M) \subset T_a M$, und somit $T_a^{\perp} M \subset T_a^{\perp}(\del M)$.

\textbf{Behauptung 3:} $\R^n = T_a^{\perp} M \oplus (T_a M \cap T_a^{\perp}(\del M)) \oplus T_a(\del M)$.

\textit{Beweis:} Wir zeigen zuerst, dass die drei Räume paarweise orthogonal sind und dann, dass ihre direkte Summe ganz $\R^n$ ist.

\textit{Schritt 1: Paarweise Orthogonalität.}
\begin{itemize}
\item $T_a^{\perp} M \perp T_a(\del M)$: Da $T_a(\del M) \subset T_a M$ und $T_a^{\perp} M \perp T_a M$, folgt $T_a^{\perp} M \perp T_a(\del M)$.

\item $T_a^{\perp} M \perp (T_a M \cap T_a^{\perp}(\del M))$: Da $T_a M \cap T_a^{\perp}(\del M) \subset T_a M$ und $T_a^{\perp} M \perp T_a M$, folgt die Orthogonalität.

\item $(T_a M \cap T_a^{\perp}(\del M)) \perp T_a(\del M)$: Da $T_a M \cap T_a^{\perp}(\del M) \subset T_a^{\perp}(\del M)$ und $T_a(\del M) \perp T_a^{\perp}(\del M)$, folgt die Orthogonalität.
\end{itemize}

\textit{Schritt 2: Dimensionsargument.}
Wir berechnen die Dimensionen:
\begin{itemize}
\item $\dim T_a^{\perp} M = n-k$
\item $\dim T_a(\del M) = k-1$
\item Für $\dim(T_a M \cap T_a^{\perp}(\del M))$ nutzen wir: Da $\dim T_a M = k$ und $\dim T_a^{\perp}(\del M) = n-(k-1) = n-k+1$, und da $T_a M + T_a^{\perp}(\del M) = \R^n$ (da ihre orthogonalen Komplemente $T_a^{\perp} M$ und $T_a(\del M)$ sich zu einem $(n-k) + (k-1) = n-1$ dimensionalen Raum addieren), folgt aus der Dimensionsformel:
\[
\dim(T_a M \cap T_a^{\perp}(\del M)) = \dim T_a M + \dim T_a^{\perp}(\del M) - \dim(T_a M + T_a^{\perp}(\del M)) = k + (n-k+1) - n = 1.
\]
\end{itemize}

Die Summe der Dimensionen ist $(n-k) + 1 + (k-1) = n$, und da die Räume paarweise orthogonal sind, folgt
\[
\R^n = T_a^{\perp} M \oplus (T_a M \cap T_a^{\perp}(\del M)) \oplus T_a(\del M).
\]

\textbf{Teil (b):}

Wir zeigen die Existenz und Eindeutigkeit der Abbildung $N: \del M \to \R^n$.

\textbf{Eindeutigkeit:} 
Angenommen, es gibt eine solche Abbildung $N$. Für jedes $x \in \del M$ muss $N(x)$ folgende Bedingungen erfüllen:
\begin{itemize}
\item $N(x) \in T_x M \cap T_x^{\perp}(\del M)$, welches nach Teil (a) ein 1-dimensionaler Raum ist.
\item $\|N(x)\| = 1$.
\end{itemize}

Daher gibt es nur zwei Kandidaten für $N(x)$: die beiden Einheitsvektoren in $T_x M \cap T_x^{\perp}(\del M)$. Bedingung (iii) legt fest, welcher der beiden zu wählen ist. Somit ist $N(x)$ eindeutig bestimmt.

\textbf{Existenz:}
Wir konstruieren $N$ lokal und zeigen dann, dass die lokalen Definitionen zusammenpassen.

Sei $x \in \del M$. Wähle eine Karte $\phi: V \cap H^k \to U \cap M$ für $M$ um $x$ mit $\phi(y) = x$ für ein $y \in V \cap \partial H^k$. Hier ist $H^k = \{z \in \R^k : z_1 \geq 0\}$ der obere Halbraum.

Das Differential $D\phi(y): \R^k \to \R^n$ ist injektiv, und das Bild ist $T_x M$. Der Vektor $e_1 = (1,0,\ldots,0) \in \R^k$ zeigt nach außen von $H^k$ bei $y$. Betrachte den Vektor
\[
v = D\phi(y)(e_1) \in T_x M.
\]

Da $\phi$ den Rand $\partial H^k$ auf $\del M$ abbildet, bildet $D\phi(y)$ den Tangentialraum $T_y(\partial H^k) = \{z \in \R^k : z_1 = 0\}$ auf $T_x(\del M)$ ab. Daher ist $v = D\phi(y)(e_1) \perp T_x(\del M)$, also $v \in T_x M \cap T_x^{\perp}(\del M)$.

Setze
\[
N(x) = \frac{v}{\|v\|}.
\]

Dies erfüllt alle drei Bedingungen:
\begin{itemize}
\item[(i)] $\|N(x)\| = 1$ nach Konstruktion.
\item[(ii)] $N(x) \in T_x M \cap T_x^{\perp}(\del M)$ wie oben gezeigt.
\item[(iii)] Der Vektor $w$ mit $D\phi(y)w = N(x)$ ist $w = \frac{e_1}{\|v\|}$, dessen erste Komponente $\frac{1}{\|v\|} > 0$ ist.
\end{itemize}

\textbf{Unabhängigkeit von der Kartenwahl:}
Seien $\phi_1: V_1 \cap H^k \to U_1 \cap M$ und $\phi_2: V_2 \cap H^k \to U_2 \cap M$ zwei Karten um $x$ mit $\phi_1(y_1) = x = \phi_2(y_2)$. Der Kartenwechsel
\[
\psi = \phi_2^{-1} \circ \phi_1: \phi_1^{-1}(U_1 \cap U_2) \cap H^k \to \phi_2^{-1}(U_1 \cap U_2) \cap H^k
\]
ist ein Diffeomorphismus, der $\partial H^k$ auf $\partial H^k$ abbildet. Das Differential $D\psi(y_1)$ bildet daher den nach außen zeigenden Vektor bei $y_1$ auf einen nach außen zeigenden Vektor bei $y_2$ ab. Dies zeigt, dass beide Karten dasselbe $N(x)$ liefern.

\textbf{Stetigkeit:}
Die Stetigkeit von $N$ folgt aus der lokalen Konstruktion: In einer Kartenumgebung hängt $N$ stetig von den Koordinaten ab, da die Normierung und das Differential stetig sind.

Damit ist die Existenz und Eindeutigkeit von $N$ bewiesen.

\end{document}