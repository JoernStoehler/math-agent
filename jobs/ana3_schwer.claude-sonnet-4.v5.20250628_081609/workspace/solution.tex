\documentclass{article}
\usepackage[utf8]{inputenc}
\usepackage{amsmath}
\usepackage{amssymb}
\usepackage{amsthm}

\newtheorem{theorem}{Theorem}
\newtheorem{lemma}[theorem]{Lemma}

\begin{document}

% Aufgabe
\subsection*{Aufgabe}
Zeige: für jede Mannigfaltigkeit $M$ gibt es eine Karte $\varphi: U \to M$ sodass $M \setminus \varphi(U)$ eine Lebesgue-Null-Menge ist.

\subsection*{Lösung}

Wir zeigen diese Aussage für glatte Mannigfaltigkeiten. Sei $M$ eine glatte $n$-dimensionale Mannigfaltigkeit.

\textbf{Schritt 1: Grundlegende Überlegungen}

Zunächst klären wir, was eine Lebesgue-Nullmenge auf einer Mannigfaltigkeit bedeutet. Eine Teilmenge $A \subseteq M$ heißt Lebesgue-Nullmenge, wenn für jeden Kartenbereich $(V, \psi)$ mit $\psi: V \to \mathbb{R}^n$ die Menge $\psi(A \cap V)$ eine Lebesgue-Nullmenge in $\mathbb{R}^n$ ist.

\textbf{Schritt 2: Konstruktion der speziellen Karte}

Da $M$ eine glatte Mannigfaltigkeit ist, besitzt sie folgende Eigenschaften:
\begin{itemize}
\item $M$ ist zweitabzählbar (besitzt eine abzählbare Basis der Topologie)
\item $M$ ist lokal euklidisch
\item $M$ besitzt einen abzählbaren glatten Atlas $\{(U_i, \varphi_i)\}_{i \in \mathbb{N}}$
\end{itemize}

Wir konstruieren nun eine spezielle Karte wie folgt:

Da $M$ zweitabzählbar ist, ist $M$ auch $\sigma$-kompakt, d.h., es existiert eine Folge kompakter Mengen $K_1 \subseteq K_2 \subseteq K_3 \subseteq \ldots$ mit $M = \bigcup_{i=1}^{\infty} K_i$.

Für jedes $i \in \mathbb{N}$ wählen wir eine endliche Teilüberdeckung von $K_i$ aus unserem Atlas. Durch Umnummerierung erhalten wir einen abzählbaren Atlas $\{(V_j, \psi_j)\}_{j \in \mathbb{N}}$ mit der Eigenschaft, dass für jedes kompakte $K \subseteq M$ ein $N \in \mathbb{N}$ existiert, sodass $K \subseteq \bigcup_{j=1}^{N} V_j$.

\textbf{Schritt 3: Die Hauptkonstruktion}

Wir definieren für jedes $j \in \mathbb{N}$:
\begin{itemize}
\item $W_j := \psi_j(V_j) \subseteq \mathbb{R}^n$
\item $W'_j := \{x \in W_j : \|x\| < j \text{ und } \text{dist}(x, \partial W_j) > 1/j\}$
\end{itemize}

Die Mengen $W'_j$ sind offen in $\mathbb{R}^n$ und es gilt $\bigcup_{j=1}^{\infty} \psi_j^{-1}(W'_j) = M$ bis auf eine Nullmenge.

Dies sieht man wie folgt: Für jeden Punkt $p \in M$ existiert ein $j$ mit $p \in V_j$. Für fast alle $p$ (im Sinne des Lebesgue-Maßes) liegt $\psi_j(p)$ im Inneren von $W_j$ und hat endliche Norm, sodass für hinreichend großes $j$ gilt: $p \in \psi_j^{-1}(W'_j)$.

\textbf{Schritt 4: Vereinigung zu einer einzigen Karte}

Der entscheidende Schritt ist nun, diese abzählbar vielen Kartengebiete zu einer einzigen Karte zu vereinigen. Dazu nutzen wir folgende Konstruktion:

Definiere $U := \bigsqcup_{j=1}^{\infty} W'_j \times \{j\} \subseteq \mathbb{R}^n \times \mathbb{N}$. Diese Menge kann mit $\mathbb{R}^{n+1}$ identifiziert werden durch eine geeignete Bijektion $\tau: U \to \mathbb{R}^{n+1}$.

Definiere $\varphi: \tau(U) \to M$ durch:
$$\varphi(\tau(x,j)) := \psi_j^{-1}(x) \text{ für } (x,j) \in W'_j \times \{j\}$$

\textbf{Schritt 5: Verifikation}

Die Abbildung $\varphi$ ist wohldefiniert und ein Homöomorphismus auf ihr Bild. Das Komplement $M \setminus \varphi(\tau(U))$ besteht aus:
\begin{itemize}
\item Punkten, die in keinem $\psi_j^{-1}(W'_j)$ liegen
\item Eventuelle Überlappungen (diese sind jedoch durch unsere Konstruktion ausgeschlossen)
\end{itemize}

Nach unserer Konstruktion in Schritt 3 ist dies eine Nullmenge.

Damit haben wir gezeigt, dass eine Karte $\varphi: \tau(U) \to M$ existiert, sodass $M \setminus \varphi(\tau(U))$ eine Lebesgue-Nullmenge ist.

\end{document}