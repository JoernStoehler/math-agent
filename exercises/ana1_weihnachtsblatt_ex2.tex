\begin{prob}[Schiefe Türme]
  Eine Architektin möchte gerne von Mathematikern, wie euch, eine verzwickte
  Aufgabe lösen lassen. Die Architektin würde für ihren Auftraggeber König
  Cesàro gerne eine Turm aus gleichgroßen Ziegelsteinen bauen, ohne dabei Zement
  oder ähnliche Hilfmittel zu verwenden. Dieser Turm soll zusätzlich möglichst
  schief sein und zwar soll jede \enquote{Etage} je aus genau einem Ziegelstein
  bestehen. Da König Cesàro sehr vermögend ist, müsst ihr euch um Kosten von
  eventuell sehr vielen Steinen keine Gedanken machen. Um der Architektin aus
  der Patsche zu helfen, konstruiert einen Turm, der möglichst weit übersteht.
  Wie weit kann er maximal überstehen?
  \begin{center}
    \includestandalone[width=\linewidth/2]{tikz/bricks_tower}
  \end{center}
  Zur Konstruktion sei das Folgende gesagt:
  \begin{itemize}
  \item Der Turm bricht nicht zusammen, wenn jeweils der
    Schwerpunkt aller oberhalb eines Steines $S$ liegenden Steine über dem Stein
    $S$ liegt.
  \item Aus der Physik wissen wir: Sind $K_1$ und $K_2$ zwei Körper vom Gewicht
    $m_1$ bzw.~$m_2$, und sind die $x$-Koordinaten ihrer Schwerpunkte bezüglich
    eines gegebenen (kartesischen) Koordinatensystems durch $s_1$ und $s_2$
    gegeben, so berechnet sich die $x$-Koordinate des Schwerpunktes von
    $K_1 \cup K_2$ nach der Formel
    $\displaystyle{\frac{m_1 s_1 + m_2 s_2}{m_1 + m_2}}$.
  \item Beginne die Konstruktion (in Gedanken) mit den beiden obersten
    Ziegelsteinen; berechne den Schwerpunkt des Körpers, der aus diesen beiden
    Steinen besteht und überlege, wie weit der vorletze Stein über den
    vorvorletzten herausragen kann \dots \end{itemize}
\end{prob}
