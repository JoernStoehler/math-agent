\begin{prob}
  Beweise mit Hilfe des Zwischenwertsatzes:
  \begin{enumerate}[label=(\alph*)]
  \item Sei $f \colon \interval{0}{1} \to \R$ eine stetige Funktion mit
    $f(0) = f(1)$. Dann gibt es für jede natürliche Zahl $n \in \N$ ein
    $p \in \interval{0}{1 - \frac{1}{n}}$, so dass $f(p) = f(p + \frac{1}{n})$.
  \item Jedes Polynom ungeraden Grades mit reellen Koeffizienten hat eine reelle
    Nullstelle.
  \end{enumerate}
\end{prob}
