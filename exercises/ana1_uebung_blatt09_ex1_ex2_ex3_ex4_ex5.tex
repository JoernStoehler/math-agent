%% WARNING: This file contains multiple exercises and should be split into separate files
\begin{exercise}[Indizierung von Folgen und Reihen]
    Man berechne:
  \begin{enumerate}[label=(\alph*)]
  \item $\displaystyle \sum_{k=1}^n\frac{1}{n}$
  \item $\displaystyle \sum_{n=1}^\infty \frac{1 + (-1)^n n}{n^2}$ (Diese Reihe
    ist bereits aus der Übung bekannt)
  \item
    $\displaystyle \sum_{n = 1}^\infty \left( \frac{1}{n} \sum_{k = 1}^n
      (-1)^{k+1} \cdot (2k-1) \right)$ (vgl. Tutorien-Blatt 7)
  \item Sind die Reihen $\displaystyle \sum_{k=1}^{\infty} a_k$,
    $\displaystyle \sum_{n=1}^{\infty} a_n$ und
    $\displaystyle \sum_{n=1}^{\infty}\sum_{k=1}^{n} a_k$ die selben? Beweise
    oder widerlege. Dies ist keine schwere Übungsaufgabe, soll aber nochmal die
    Denkweise \enquote{Der Indexbuchstabe spielt keine Rolle} schärfen.
  \end{enumerate}
\end{exercise}


\begin{exercise}[$\liminf$ und $\limsup$]
  Man zeige die folgenden Aussagen:
  \begin{enumerate}[label=(\alph*)]
  \item $\limsup_{n \to \infty} a_n$ (bzw. $\liminf_{n \to \infty} a_n$) ist der
    größte (bzw. kleinste) Häufungspunkt der Folge $a_n$.

  \item Man mache sich den Zusammenhang von $\limsup$, $\liminf$, $\sup$ und
    $\inf$ einer Folge $a_n$ klar, d.h. zeige zunächst, dass folgende
    Ungleichung für jede Folge $a_n$ gilt:
    \begin{equation*}
      \inf \Set{a_n | n \in \N}
      \leq
      \liminf_{n \to \infty} a_n
      \leq
      \limsup_{n \to \infty} a_n
      \leq
      \sup \Set{a_n | n \in \N}.
    \end{equation*}
    Mache dir nun Gedanken, in welchen Fällen Gleichheit gilt (\emph{Tipp}:
    Monotonie der Folge $a_n$).

  \item (Quotientenkriterium mit $\limsup$) Beweise, dass das
    Quotientenkriterium mit $\limsup$ korrekt ist, d.h. es gilt: Gilt
    $\limsup_{n \to \infty} \abs{\frac{a_{n+1}}{a_n}} < 1$, so konvergiert die
    Reihe $\displaystyle \sum_{n=0}^{\infty}a_n$ absolut.

  \item (Quotientenkriterium mit $\liminf$) Zeige nun, dass es nicht ausreicht,
    wenn $\liminf_{n \to \infty} \abs{\frac{a_{n+1}}{a_n}} < 1$ gilt, um
    Konvergenz der Reihe $\displaystyle \sum_{n=0}^{\infty}a_n$ zu zeigen.
    Genauer noch: Über die Konvergenz lässt sich gar keine Aussage mehr treffen,
    wenn
    \begin{equation*}
      \liminf_{n \to \infty} \abs{\frac{a_{n+1}}{a_n}}
      \leq
      1
      \leq
      \limsup_{n \to \infty} \abs{\frac{a_{n+1}}{a_n}}
    \end{equation*}
    gilt.

  \item(Divergenz mit Quotientenkriterium und $\liminf$) Zu guter Letzt zeige
    man nun noch, dass das \enquote{umgekehrte} Quotientenkriterium mit
    $\liminf$ funktioniert, dass also folgendes gilt: Gilt
    $\liminf_{n \to \infty} \abs{\frac{a_{n+1}}{a_n}} > 1$, so divergiert die Reihe
    $\displaystyle \sum_{n=0}^{\infty}a_n$.
  \end{enumerate}
  \textbf{Achtung:} Es sei an dieser Stelle nochmal erwähnt, dass man durch die
  Betrachtung von $\liminf$ bzw. $\limsup$ die Aussage für \enquote{$=1$}
  verliert, man also nichts mehr über Konvergenz aussagen kann, falls
  $\liminf_{n \to \infty} \abs{\frac{a_{n+1}}{a_n}} = 1$ bzw.
  $\limsup_{n \to \infty} \abs{\frac{a_{n+1}}{a_n}} = 1$ gilt!
\end{exercise}


\begin{exercise}
  Da der Beweis des \emph{Doppelreihensatzes} in der Vorlesung etwas kurz
  ausgefallen ist, wollen wir hier nochmal einen genauen Blick darauf werfen.
  Beweise den Doppelreihensatz:
  \begin{theorem}[Doppelreihensatz]
    Sei $\displaystyle \sum_{i,j=0}^{\infty}a_{ij}$ absolut konvergent. Dann
    gilt:
    \begin{equation*}
      \sum_{i,j=0}^{\infty} a_{ij}
      =
      \sum_{i=0}^{\infty}\left( \sum_{j=0}^{\infty}a_{ij}\right)
      =
      \sum_{j=0}^{\infty}\left( \sum_{i=0}^{\infty}a_{ij}\right)
      =
      \sum_{k=0}^{\infty}\left( \sum_{i=0}^{k}a_{i,k-i}\right).
    \end{equation*}
  \end{theorem}
\end{exercise}

\newpage

\subsection*{Ein Aufgabensammelsurium zu komplexen Zahlen}
Diese Aufgabensammlung soll als Hilfe für all diejenigen dienen, die sich mit
komplexen Zahlen sehr schwer tun. Die ein oder andere Aufgabe sollte sicherlich
im Tutorium bearbeitet werden und für viele sind diese Aufgaben vielleicht auch
sehr leicht.

\begin{exercise}[Einfache komplexe Rechnungen]
  Es seien $z_1 \coloneqq 5 + 3\im$ und $z_2 \coloneqq 1 - \im$. Berechne:
  \begin{enumerate}[label=(\alph*)]  \item $z_1 + z_2$
  \item $z_1 - z_2$
  \item $z_1 \cdot z_2$
  \item $ z_1^2 $ und $z_2^2$
  \end{enumerate}
\end{exercise}

\begin{exercise}
  \label{ex:complex-calculations}
  Berechne den Real- und Imaginärteil, das komplex Konjugierte, sowie den Betrag
  von:
  \begin{enumerate}[label=(\alph*)]
  \item $4 + 3\im$
  \item $\displaystyle 7 \cdot \exp\left(3 + \frac{3 \pi}{4} \im\right)$
  \item $\displaystyle\frac{7 - 2\im}{5 + \im}$
  \item $2^\im$
  \item $\im^\im$
  \end{enumerate}
\end{exercise}
