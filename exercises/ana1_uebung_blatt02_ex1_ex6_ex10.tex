%% WARNING: This file contains multiple exercises and should be split into separate files
\begin{exercise}[Vollständigen Induktion]
  Zeige die folgenden Aussagen mit Hilfe von vollständiger Induktion:
  \begin{enumerate}
  \item für alle $n \in \N_0$ gilt:
    \begin{equation*}
      \begin{split}
        \sum\limits_{ i=1 }^{ n}{ \frac{1}{i(i+1)} } = \frac{n}{n+1}
      \end{split}
    \end{equation*}
  \item für alle ungeraden $n \in \N$ gilt:
    \begin{itemize}
    \item $2^{2n + 1}$ ist durch $8$ teilbar
    \item $2^{n} + 1$ ist durch $3$ teilbar
    \end{itemize}
  \end{enumerate}
\end{exercise}

\begin{exercise}[Türme von Hanoi]
  Die \emph{Türme von Hanoi} sind ein bekanntes Spiel mit drei Stäben und
  Scheibenringen unterschiedlicher Größen die darauf aufgesteckt werden können.
  Wie der Name vermuten lässt sind aus den Scheiben Türme zu bauen. Ziel des
  Spiels ist, einen Turm der Größe $N$ von dem linken Stab auf den rechten
  umzuwälzen. Die Regeln dabei sind:
  \begin{itemize}
  \item Pro Zug darf nur eine Scheibe bewegt werden
  \item Ein Zug besteht daraus die oberste Scheibe von einem der Stäbe zu nehmen
    und auf einen der anderen Stäbe zu legen
  \item Scheiben dürfen nur auf Scheiben platziert werden, die größer sind als
    sie selbst, es werden also keine \enquote{Überhänge} gebaut
  \end{itemize}
  Aufgabe ist nun zu zeigen, dass für jedes $n \in \N$ der Turm von Hanoi bewegt
  werden kann, also das Spiel immer eine Lösung hat.
  \begin{itemize}
  \item[$\ast$] Bonusaufgabe: Zeige dass immer eine Lösung mit $2^n - 1$ Zügen
    möglich ist
  \item[$\ast\ast$] Bonusaufgabe 2: Zeige dass die Lösung in $2^n - 1$ Zügen auch
    die optimale Lösung ist.
  \end{itemize}
\end{exercise}

\begin{exercise}[Fibonacci-Zahlen]
  Die Fibonnaci Zahlen sind definiert durch $f_0 = f_1 = 1$ und rekursiv
  weiter mit $f_{n+1} = f_n + f_{n-1}$. Zeige induktiv für alle $n \in \N$:
  \begin{enumerate}
  \item $\sum\limits_{ i= 0 }^{ n }{ f_i } = f_{n+2} - 1$
  \item $f_{2n}$ ist teilbar durch $f_n$
  \item $f_n$ und $f_{n+1}$ sind relativ prim, das heißt es gibt keine Zahl
    $a \in \N$, die sowohl $f_n$ als auch $f_{n+1}$ teilt
  \end{enumerate}
\end{exercise}

\begin{exercise}[Trinominos]
  Ein \emph{Trinomino} ist wie ein Dominostein, nur aus drei Quadraten aufgebaut
  und von der Form \includestandalone[scale=0.15]{graphics/trinominos/trinom1}.
  Trinominos dürfen beliebig gedreht werden, und sollen im Folgenden benützt
  werden, um \enquote{Schachbretter} lückenlos zu pflastern. Dabei stimmt die
  Größe der Quadrate aus denen die Schachbretter aufgebaut sind überein mit der
  Größe der Quadrate aus denen die Trinominos gemacht sind. Ein Trinomino
  bedeckt so immer genau drei Quadrate.\\
  Zu zeigen ist nun, dass es immer (für alle $n \in \N$) möglich ist, folgende
  \enquote{Schachbretter} vollständig ohne Überlappen mit Trinominos zu
  pflastern:
  \begin{itemize}
  \item $\downarrow$ Ein $2^n \times 2^n$ Schachbrett dem die rechte untere
    Ecke fehlt
    \begin{figure}[!h]
      \centering
      \begin{minipage}[b]{0.38\textwidth}
        \includestandalone[width=\textwidth]{graphics/trinominos/trinom3}
      \end{minipage}
      \hspace{1cm}
      \begin{minipage}[b]{0.38\textwidth}
        \includestandalone[width=\textwidth]{graphics/trinominos/trinom4}
      \end{minipage}
    \end{figure}
  \item Ein $2^n \times 2^n$ Schachbrett dessen rechtes unteres Viertel fehlt
    $\uparrow$
  \end{itemize}
\end{exercise}

\begin{exercise}[Fehlgeschlagene Induktionen]
  Versuche eine vollständige Induktion bei den offensichtlich falschen Aussagen
  \begin{enumerate}
  \item $\fa{k \in \N} k > k + 1$
  \item $\fa{k \in \N} k^2 > k^3$
  \item $\fa{k \in \N} 5k = 0$
  \end{enumerate}
  Was geht hier schief? Was funktioniert dennoch?
\end{exercise}

\begin{exercise}[Mächtigkeit der Potenzmenge]
  Die Potenzmenge $\mathcal{P}(M)$ ist bekanntlich definiert als die Menge aller
  Teilmengen einer Menge $M$. Zeige, dass für endliche Mengen $M$ gilt
  \begin{equation*}
    \begin{split}
      \abs{ \mathcal{P}(M) } > \abs{ M }
    \end{split}
  \end{equation*}
  wobei $\abs{ M }$ wie normal die Mächtigkeit der Menge (also die Anzahl
  Elemente in der Menge) bezeichne.
\end{exercise}

\begin{exercise}[Boolsche Algebra und Mengen]
  Überlege ob für Mengen alle Axiome einer Booleschen Algebra erfüllt
  sind, und was das bedeuten würde. Sind zum Beispiel die De-Morgan'schen
  Gesetze noch einmal zu zeigen?
\end{exercise}

\begin{exercise}[Graph einer Abbildung]
  \label{ex:graph-of-map}
  Der Graph einer Abbildung $f\colon X \rightarrow Y$ ist bekanntlich definiert
  durch
  \begin{equation*}
    \begin{split}
      \graph(f) \eqdef \left\{ (a,b) \in X\times Y \:\arrowvert\:
        f(a)=b\right\} \subset X\times Y
    \end{split}
  \end{equation*}
  Zeige dass die Angabe eines Graphen eine Abbildung schon eindeutig bestimmt
  und andersherum.
\end{exercise}

\begin{exercise}[Raum aller Abbildungen]
  Überlege anhand von endlichen Mengen, warum es sinnvoll ist, den Raum der
  Abbildungen $f$ mit
  \begin{center}
    \begin{tikzcd}[row sep=tiny]
      X
      \arrow [r, "f"]
      & Y \\
    \end{tikzcd}
  \end{center}
  Auch mit $Y^X$ zu betiteln.
\end{exercise}

\begin{exercise}[Symmetrische Mengendifferenz]
  Es sei $X$ eine Menge, und $A,B \subset X$ Teilmengen. Wir definieren die
  \emph{Symmetrische Differenz} als
  \begin{center}
    \begin{tikzcd}[row sep=tiny]
      \mathcal{P}(X)\times\mathcal{P}(X)
      \arrow [r, "\vartriangle"]
      & \mathcal{P}(X) \\
      (A,B)
      \arrow [r, mapsto]
      & A\vartriangle B \eqdef (A\cup B)\setminus (A\cap B) \\
    \end{tikzcd}
  \end{center}
  Weise nach, dass
  \begin{itemize}
  \item $\vartriangle$ kommutativ ist und
  \item $\vartriangle$ assoziativ ist und dass
  \item $A\vartriangle B = (A\setminus B)\cup (B\setminus A)$.
  \end{itemize}
\end{exercise}
