\begin{prob} 
Sei $U\subset\R^n$ offen und $f:U\to\R$ eine differenzierbare
Funktion. Zeige:

(a) Der Gradient von $f$ steht senkrecht auf den Niveaumengen
$f^{-1}(c)$ im folgenden Sinne: Ist $\gamma:(-\eps,\eps)\to f^{-1}(c)$
eine differenzierbare Kurve, so gilt $\la \grad
f\bigl(\gamma(0)\bigr),\gamma'(0)\ra=0$.

(b) Sei $U$ {\it wegzusammenh\"angend}, d.h.~je zwei Punkte in $U$
lassen sich durch eine differenzierbare Kurve in $U$ verbinden. Ist $\grad
f(x)=0$ f\"ur alle $x\in U$, so ist $f$ auf $U$ konstant.
\end{prob}
