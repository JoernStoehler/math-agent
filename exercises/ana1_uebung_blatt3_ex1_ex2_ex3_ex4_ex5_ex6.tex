%% WARNING: This file contains multiple exercises and should be split into separate files
\begin{prob}
  Folgere aus den Axiomen der Multiplikation in einem Körper:
  \begin{enumerate}[label=(\alph*)]
  \item Die Eins ist eindeutig bestimmt.
  \item Für $a \neq 0$ ist das inverse Element $a^{-1}$ eindeutig bestimmt.
  \item Für $a \neq 0$ gilt $(a^{-1})^{-1} = a$
  \item Für $a,\ b\neq 0$ gilt $(ab)^{-1} = a^{-1}b^{-1}$
  \end{enumerate}
\end{prob}

\begin{prob}
  Die \textit{Charakteristik} eines Körpers $\K$, geschrieben $\chi(\K)$, ist
  das kleinste $n \in \N$ so, dass $n \cdot 1 = 0$. Wenn es kein solches $n$
  gibt, definieren wir $\chi(\K) := 0$. \\
  Zeige, dass...
  \begin{enumerate}[label=(\alph*)]
  \item $\chi(\K) \neq 0 \iff \chi(\K)=p$ für eine Primzahl $p$;
  \item $\chi(\K)= 0 \iff \K$ enthält $\Q$ als Teilkörper;
  \item $\fa{a \in \K} \chi(\K) \cdot a = 0$;
  \item falls die Anzahl der Elemente von $\K$ endlich ist, so gilt
    $\chi(\K) > 0$ und $\chi(\K)$ ist ein Teiler der Anzahl der Elemente von $\K$
  \end{enumerate}
\end{prob}

\begin{prob}Konstruiere einen Körper...
  \begin{enumerate}[label=(\alph*)]
  \item ...mit 3 Elementen;
  \item ...mit 4 Elementen.
  \end{enumerate}
  Zeige außerdem, dass es keinen Körper mit 6 Elementen gibt.
\end{prob}

\begin{prob}
  Folgere aus den Körperaxiomen und den Anordnungsaxiomen in einem angeordneten
  Körper $\K$:
  \begin{enumerate}[label=(\alph*)]
  \item $a < b$ und $c > 0 \implies ac < bc$; $a < b$ und
    $c < 0 \implies ac > bc$.
  \item $a \neq 0 \implies a^2 > 0$. Insbesondere $1 > 0$.
  \item $0 < a < b \implies a^{-1} > b^{-1} >0$.
  \item Drei Elemente $a, b, c$ von $\K$ sind genau dann alle positiv, wenn die
    folgenden Ungleichungen alle erfüllt sind:
    \begin{equation*}
      \begin{split}
        a + b + c    & > 0 \\
        ab + bc + ca & > 0 \\
        abc          & > 0
      \end{split}
    \end{equation*}
  \end{enumerate}
\end{prob}

\begin{prob}
  Betrachte die Menge der gebrochenrationalen Funktionen
  \begin{equation*}
    \mathbb{Q}(t)
    \coloneqq
    \Set{\frac{P}{Q} | \textit{P, Q Polynome in t mit Koeffizienten in } \Q, \ Q
      \neq 0}
  \end{equation*}
  mit der in der Vorlesung definierten Addition und Multiplikation. Zeige:
  \begin{enumerate}[label=(\alph*)]
  \item $\Q(t)$ ist ein Körper.
  \item $\Q(t)$ ist ein angeordneter Körper mit
    $\frac{P}{Q}>0 \iff a_n\cdot b_m>0$ für $P = a_nt^n + \ldots + a_0$,
    $Q = b_mt^m + \ldots + b_0$ mit $a_n \neq 0$ und $b_m \neq 0$.
  \item $\Q(t)$ mit dieser Anordnung ist kein archimedischer Körper.
  \end{enumerate}
\end{prob}

\begin{prob}
  \textit{Wiederholungsaufgabe. Diese Aufgabe ist 2 Punkte wert, die nur
    verteilt werden, wenn du die für die entsprechende Aufgabe auf Blatt 1 2
    Punkte oder weniger erhalten hast.} \\
  Für Aussagen $A$ und $B$, leite folgende Gleichheiten aus den Gesetzen (1)-(7)
  her, ohne Wahrheitstafeln zu verwenden:
  \begin{enumerate}
  \item Wenn $A \land B = 0$ und $A \lor B = 1$, dann ist $B = \neg A$.
  \item $\neg (A \lor B) = (\neg A) \land (\neg B)$, $\neg (A \land B) = (\neg
    A) \lor (\neg B)$
  \end{enumerate}
\end{prob}
