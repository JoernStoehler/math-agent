\begin{exercise}[Türme von Hanoi]
  Die \emph{Türme von Hanoi} sind ein bekanntes Spiel mit drei Stäben und
  Scheibenringen unterschiedlicher Größen die darauf aufgesteckt werden können.
  Wie der Name vermuten lässt sind aus den Scheiben Türme zu bauen. Ziel des
  Spiels ist, einen Turm der Größe $N$ von dem linken Stab auf den rechten
  umzuwälzen. Die Regeln dabei sind:
  \begin{itemize}
  \item Pro Zug darf nur eine Scheibe bewegt werden
  \item Ein Zug besteht daraus die oberste Scheibe von einem der Stäbe zu nehmen
    und auf einen der anderen Stäbe zu legen
  \item Scheiben dürfen nur auf Scheiben platziert werden, die größer sind als
    sie selbst, es werden also keine \enquote{Überhänge} gebaut
  \end{itemize}
  Aufgabe ist nun zu zeigen, dass für jedes $n \in \N$ der Turm von Hanoi bewegt
  werden kann, also das Spiel immer eine Lösung hat.
  \begin{itemize}
  \item[$\ast$] Bonusaufgabe: Zeige dass immer eine Lösung mit $2^n - 1$ Zügen
    möglich ist
  \item[$\ast\ast$] Bonusaufgabe 2: Zeige dass die Lösung in $2^n - 1$ Zügen auch
    die optimale Lösung ist.
  \end{itemize}
\end{exercise}
