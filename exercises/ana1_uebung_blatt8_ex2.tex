\begin{prob}
  Für $s \in \R$ und $n \in \N_0$ definieren wir den
  \textit{Binomialkoeffizienten} rekursiv durch
  \begin{equation*}
    \binom{s}{0} \coloneqq 1 \quad \text{und} \quad
    \binom{s}{n + 1} \coloneqq \frac{s - n}{n + 1} \cdot \binom{s}{n}
  \end{equation*}
  Sei weiter $x \in \R$. Wir definieren die \textit{Binomialreihe} als
  \begin{equation*}
    B(s,x):=\sum_{k=0}^\infty \binom{s}{k}x^k.
  \end{equation*}
  \begin{enumerate}[label=(\alph*)]
  \item Beweise das \emph{Additionstheorem} für alle $s, t \in \R$:
    \begin{equation*}
      \sum_{k=0}^n \binom{s}{k}\binom{t}{n - k}
      =
      \binom{s+t}{n}.
    \end{equation*}
  \item Zeige, dass die Binomialreihe $B(s,x)$ für $\abs{x} < 1$ und beliebiges
    $s \in \R$ absolut konvergiert.
  \item Beweise die \textit{Funktionalgleichung} für $s, t \in \R$ und
    $\abs{x} < 1$:
    \begin{equation*}
      B(s, x) \cdot B(t,x) = B(s + t, x).
    \end{equation*}
  \item Zeige: Für jedes $x \in \R$ ist
    \begin{equation*}
      \lim_{n\to\infty} B\left( n, \frac{x}{n} \right) = e^x,
    \end{equation*}
    wobei $n\in\mathbb{N}$
  \end{enumerate}
\end{prob}
