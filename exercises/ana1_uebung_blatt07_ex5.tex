\begin{exercise} Abzählbare Mengen
  \begin{enumerate}[label=(\alph*)]
  \item
    Wir definieren die Menge aller \enquote{abbrechenden} rationalen Folgen als
    \begin{equation*}
      \begin{split}
        d \eqdef \Set{a \colon \N \to \Q | \ex{N \in \N} \fa{n \geqslant N} a(n) = 0}
      \end{split}
    \end{equation*}
    Man zeige nun, dass $d$ abzählbar ist.

  \item
    Wir definieren den Polynomring über $\Z$ als
    \begin{equation*}
      \begin{split}
        \Z[X] \eqdef \left\{a_k X^k + \cdots + a_0 \:\arrowvert\: a_i \in \Z\right\}
      \end{split}
    \end{equation*}
    Man zeige, dass auch dieser als Menge abzählbar ist.

  \item
    Wir definieren die algebraischen Zahlen
    \begin{equation*}
      \begin{split}
        \mathbb{A} \eqdef \Set{x \in \R | \ex{p \in \adjunction{\Z}{X}} x
          \textnormal{ ist Nullstelle von } p}
      \end{split}
    \end{equation*}
    Man vergewissere sich kurz, dass
    \begin{equation*}
      \begin{split}
        \Z \subsetneq \Q \subsetneq \mathbb{A} \subsetneq \C
      \end{split}
    \end{equation*}
    gilt und zeige dann, dass auch die algebraischen Zahlen abzählbar sind.

  \item \label{itm:subset-of-countable-set}Wir haben in der Vorlesung gezeigt,
    dass unendliche Teilmengen von $\N$ abzählbar sind. Man wiederhole den
    Beweis und passe ihn an, um die folgende, stärkere Behauptung zu beweisen:
    \begin{theorem}
      \normalfont\itshape Ist $M$ eine abzählbare Menge, so ist jede unendliche
      Teilmenge $ S \subset M$ auch abzählbar
    \end{theorem}

  \item Wir wollen mal die Sprache betrachten
    \begin{equation*}
      \begin{split}
        D = \Set{s | s \textnormal{ ist ein grammatikalisch korrekter deutscher Satz}}
      \end{split}
    \end{equation*}
    Wie immer zeige man nun, dass $D$ abzählbar ist. Es könnte dabei hilfreich
    sein, sich der Teilaufgabe \ref{itm:subset-of-countable-set} zu bedienen.
    Mit dieser können wir dann auch direkt folgern dass die Menge aller
    Mathematischen Sätze Abzählbar ist.

  \item Man überlege sich, ob und wie man mit Induktion Aussagen zeigen kann die
    für alle Elemente einer abzählbaren Menge gelten sollen. Kommt man zum
    Schluss, dass das geht, so modifiziere man die Definition der Induktion
    entsprechend, und anders gebe man ein Gegebeispiel.
  \end{enumerate}
\end{exercise}
