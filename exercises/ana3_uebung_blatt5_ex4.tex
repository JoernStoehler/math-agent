                    \begin{prob}[Volumina von Kegeln und Simplizes] \\
Man zeige:
\begin{enumerate}[label = (\alph*)]
	\item F�r eine kompakte Teilmenge $B \subset \R^{n-1}$ und $h > 0$ sei $C_h(B) \subset \R^n$ ein Kegel der H\"ohe $h$ \"uber $B$. Dann
	$$
	\lambda_n( C_h(B) ) = \frac{1}{n} \lambda_{n-1}(B) \cdot h.
	$$
	\item Sei $S(a_0,\dots,a_n)\subset\R^n$ der $n$-Simplex mit Ecken $a_0, \dots , a_n \in \R^n$. Dann
	$$
	\lambda_n(S(a_0,\dots,a_n)) = \frac{1}{n!}|\det(a_1-a_0,a_2-a_0,\dots,a_n-a_0)|.
	$$
\end{enumerate}
%------------------------------------------------------------------------------------------------------------------------
\vspace{2mm}
                    \end{prob}
