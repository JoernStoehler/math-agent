\begin{prob}
F\"ur Teilmengen $A,B$ eines metrischen Raumes $X$
%und einen Punkt $x\in X$
definieren wir
$$
%   {\rm dist}(x,A) := \inf\{d(x,a)\mid a\in A\}\quad\text{und}\quad
   {\rm dist}(A,B) := \inf\{d(a,b)\mid a\in A,\;b\in B\}.
$$
(a) Zeige: Wenn $A$ und $B$ kompakt sind, so existieren $a\in A$ und $b\in
B$ mit $d(a,b)={\rm dist}(A,B)$. 

(b) Finde ein Beispiel, in dem die Punkte $a,b$ nicht eindeutig sind. 

(c) Finde ein Gegenbeispiel im Fall, dass die Menge $A$ nicht kompakt
ist. 
\end{prob}
