\begin{prob}
  In dieser Aufgabe sollst du untersuchen, an welchen Stellen der Vorlesung es
  entscheidend war in $\R$ zu arbeiten, und an welchen auch $\C$ genügt hätte.
  Genauer: Finde alle Sätze aus den Kapiteln 4, 5 und 7, bei denen die
  Forderung, dass $(a_n)$ eine Folge reeller Zahlen ist, nicht automatisch durch
  die Forderung, dass $(a_n)$ eine Folge komplexer Zahlen ist ersetzt werden
  kann. Kannst du diese Aussagen trotzdem leicht modifiziert retten? Wenn ja,
  gib eine neue gültige Formulierung, f"ur die der fragliche Satz auch in $\C$
  noch wahr ist, an.
\end{prob}
