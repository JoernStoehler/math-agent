    \begin{prob}[Orientierung durch Einheits-Normalenfelder]
%
(a) Sei $M\subset\R^3$ eine orientierte $2$-dimensionale Untermannigfaltigkeit mit Rand, orientiert durch ein Einheitsnormalenfeld $\nu:M\to\R^3$. Sei $N:\del M\to\R^3$ das \"aussere Einheitsnormalenfeld zu $\del M$ in $M$ aus Aufgabe 1. Man zeige: F\"ur jedes $x \in \del M$ definiert $\nu(x) \times N(x)$ die Orientierung der Randkurve $\del M$, d.h. die Orientierung von $T_x(\del M)$.

(b) Man zeichne f\"ur die Hemisph\"aren $S^2_\pm$ und die Halbtori $T_{r,R}^\pm$ aus �bung 8.6 (c-d) die Einheitsnormalenfelder $\nu$ und $N$ sowie die induzierten Orientierungen der Randkurven. Was f\"allt auf?
%------------------------------------------------------------------------------------------------------------------------
\newpage
%------------------------------------------------------------------------------------------------------------------------
\vspace{2mm}
    \end{prob}
