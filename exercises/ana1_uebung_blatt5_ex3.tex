\begin{prob}
  \begin{enumerate}[label=(\alph*)]
  \item Es sei $p > 1$ eine natürliche Zahl. Zeige: Jede positive reelle Zahl $a$
    besitzt eine $p$-adische Entwicklung, das heißt, es gibt eine ganze Zahl
    $k_0$ und eine Folge $(a_k)_{k \geq k_0}$ ganzer Zahlen mit $0 \leq a_k < p$,
    so dass
    \begin{equation*}
      a = \sum_{k=k_0}^{\infty}\frac{a_k}{p^k}.
    \end{equation*}
  \item Zwei $p$-adische Darstellungen
    $\displaystyle\sum_{k=k_0}^{\infty} \frac{a_k}{p^k}$ und
    $\displaystyle\sum_{k=k_1}^{\infty} \frac{b_k}{p^k}$ repräsentieren genau dann
    dieselbe Zahl $a \in \R_+$, wenn entweder $a_k = b_k$ für alle $k \in \Z$
    gilt, oder ein $\ell \in \Z$ existiert, sodass
    \begin{equation*}
      \begin{cases}
        a_k = b_k\ \text{für alle } k < \ell, \\
        a_\ell = b_\ell + 1,                  \\
        a_k = 0\ \text{für alle } k > \ell,   \\
        b_k = p-1\ \text{für alle } k > \ell,
      \end{cases}
    \end{equation*}
    gilt oder dasselbe,  mit den Rollen der $a_k$ und $b_k$ vertauscht, gilt.
  \end{enumerate}
\end{prob}
