\begin{exercise}[Begrenzte Differenzierbarkeit]
  Manchmal denkt man sich wie komisch es ist, dass sich manche Funktionen nur
  genau ein paar Mal ableiten lassen. Solche Funktionen zu finden und zu
  verstehen ist hier das Ziel
  \begin{enumerate}[label=(\alph*)]
  \item Finde für jedes $n \in \N$ eine Funktion, die genau $n$-Mal
    differenzierbar ist.
  \item Auf den Übungsblättern haben wir gezeigt, dass die Funktion
    \begin{equation*}
      \begin{split}
        f(x) = x \sin \left( \frac{1}{x} \right)
      \end{split}
    \end{equation*}
    stetig bei $x_0 = 0$ ist. Ist sie auch differenzierbar? Was müsste man an
    der Funktion ändern damit sie es wäre, und warum ist dem anschaulich so?
  \item Betrachte den Glättungskern
    % Someone please fix this mess and teach my how to do TeX partially defined
    % Functions beautifully and properly. Thanks.
    \begin{equation*}
      f(x) =
      \begin{cases}
        \exp \left( \frac{1}{1 - \abs{x}^2} \right) & \abs{x} < 1  \\
        0                                           & \text{sonst} \\
      \end{cases}
    \end{equation*}
    Zeige dass dieser Glättungskern überraschenderweise unendlich oft
    differenzierbar ist. Dies wird uns vermutlich in der Analysis 3 einmal
    irgendwann gute Dienste leisten.
  \end{enumerate}
\end{exercise}
