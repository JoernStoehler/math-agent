\begin{exercise}[Mächtigkeit der Potenzmenge]
  Die Potenzmenge $\mathcal{P}(M)$ ist bekanntlich definiert als die Menge aller
  Teilmengen einer Menge $M$. Zeige, dass für endliche Mengen $M$ gilt
  \begin{equation*}
    \begin{split}
      \abs{ \mathcal{P}(M) } > \abs{ M }
    \end{split}
  \end{equation*}
  wobei $\abs{ M }$ wie normal die Mächtigkeit der Menge (also die Anzahl
  Elemente in der Menge) bezeichne.
\end{exercise}
