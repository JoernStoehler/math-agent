\begin{exercise}[Grundwerkzeuge für Reihen]
  \begin{enumerate}[label=(\alph*)]
  \item Zur Wiederholung erinnere man sich an die folgenden Konvergenzkriterien
    für Reihen und beweise sie:
    \begin{itemize}
    \item Leibnitzkriterium
    \item Majorantenkriterium
    \item Quotientenkriterium
    \end{itemize}

  \item Hier ein Vorschlag für ein weiteres Kriterium, diesmal allerdings für
    Divergenz:
    \begin{theorem}[\emph{Minorantenkriterium für Divergenz}]\normalfont\itshape
      Sei $\sum \limits_{ n=0 }^{ \infty } b_n $ eine divergente Reihe und $a_n$
      eine Folge mit $a_n \geqslant b_n \geqslant 0$ für alle $n\in \N$, so gilt
      \begin{equation*}
        \begin{split}
          \sum\limits_{ n=1 }^{ N }{ a_n } \overset{N \to \infty}{\longrightarrow} \infty
        \end{split}
      \end{equation*}
    \end{theorem}
    Stimmt dieser Satz so? Beweise oder widerlege!

  \item In der Vorlesung wurde der Begriff der absoluten Konvergenz eingeführt.
    Wiederhole die Definition und zeige, dass absolute Konvergenz schon
    Konvergenz impliziert.

  \item Bei Folgen haben wir gesehen dass die ersten endlich vielen
    Folgenglieder an der Konvergenz nichts ändern. Man zeige das Analoge nun für
    Reihen:
    \begin{theorem}
      \normalfont\itshape Seien $\left( a_n \right)_{n \in N}$ und
      $\left( b_n \right)_{n\in \N}$ Folgen in $\R$, sodass ein $K \in \N$
      existiert mit
      \begin{equation*}
        \begin{split}
          \fa{n \geqslant K} a_n = b_n
        \end{split}
      \end{equation*}
      Dann konvergiert die Reihe über die $a_n$ genau dann, wenn die Reihe über
      die $b_n$ konvergiert.
    \end{theorem}
  \end{enumerate}
\end{exercise}
