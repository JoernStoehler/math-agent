\begin{exercise}[Die Cantor-Menge]
  Wir definieren für reelle Zahlen $a$ und Mengen $M\subset \R$ die Mengen
  \begin{equation*}
    \begin{split}
      aM  & \eqdef \Set{am | m \in M} \\
      a+M & \eqdef \Set{a + m | m \in M}
    \end{split}
  \end{equation*}
  Die $n$-te Cantoriterierte ist rekursiv definiert als
  \begin{equation*}
    \begin{split}
      C_0 &= (0,1)\\
      C_n &= \frac{1}{3} \cdot C_{n-1} \dot\cup \left( \frac{2}{3} + \frac{1}{3} C_{n-1} \right)
    \end{split}
  \end{equation*}
  Nun definieren wir die Cantormenge als
  \begin{equation*}
    \begin{split}
      C \eqdef \bigcap_{n\in \N} C_n
    \end{split}
  \end{equation*}
  Nach etwas Überlegung und einer Skizze ist man sich nichteinmal sicher ob
  diese überhaupt Punkte enthält. Es ist jedoch etwas schier unglaubliches der
  Fall: Die Cantormenge ist sogar überabzählbar! Dies wollen wir im Verlauf
  dieser Aufgabe zeigen.
  \begin{enumerate}[label=(\alph*)]
  \item Man fertige eine hübsche Skizze der Cantoriterierten an und vergewissere
    sich die Definition gut verdaut zu haben indem man eine nicht-rekursive
    äquivalente Darstellung der $C_n$ herleite.
  \item Zum Aufwärmen überlege man sich nun
    \begin{equation*}
      \begin{split}
        C \neq \emptyset
      \end{split}
    \end{equation*}
    \emph{Tipp}: Man betrachte dabei die $3$-adische Darstellung der Zahlen im
    Intervall $(0,1)$ und überlege sich was dann die Cantoriteration bedeutet .

  \item Nun ans Eingemachte: Man zeige, dass
    \begin{equation*}
      \begin{split}
        C \cong (0,1)
      \end{split}
    \end{equation*}
    \emph{Tipp}: Es könnte für die Bijektion hilfreich sein, auch die
    $2$-adische Darstellung nocheinmal zu wiederholen.
  \end{enumerate}
\end{exercise}
