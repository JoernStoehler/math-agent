\begin{prob}

(a) Zeige, dass die ``Metrik des franz\"osischen Eisenbahnnetzes''
($d(x,y)=d_0(x,y)$, falls $x,y$ auf einer Strecke nach P(aris) liegen,
und $d_0(x,P)+d_0(P,y)$ sonst; $d_0$ = Abstand in Luftlinie)
tats\"achlich eine Metrik ist. 

(b) Sei $p$ eine Primzahl. F\"ur $x\in\Z$ sei $n_p(x)$ der Exponent,
mit dem $p$ in der Primfaktorzerlegung von $|x|$ vorkommt. Zeige: Die
``$p$-adische Metrik'' 
$$
   d(x,y):=\begin{cases} p^{-n_p(x-y)} & \text{falls $x\neq y$,}\cr 0
   & \text{falls $x=y$.} \end{cases}
$$
ist eine Metrik auf $\Z$. {\it Bemerkung: }
  Diese Metrik spielt eine Rolle in der Zahlentheorie. 
\end{prob}
