            \begin{prob}[Windungszahl]
%\textit{Vorbemerkung: Die hier benutzten Notationen werden in Abschnitt 8.4 des Skriptums im Digicampus eingef�hrt. Insbesondere wird dort die Form $\sigma$ definiert.}
F\"ur $n\in\N$ betrachten wir die in Abschnitt 8.4 des Skripts definierte geschlossene $(n-1)$-Form
$$
   \sigma := i_w\vol_n \in \Omega^{n-1}(\R^n\setminus\{0\})\quad\text{mit}\quad
   w(x) := \frac{x}{\|x\|^n}.
$$
Sei $M\subset\R^n$ eine kompakte $n$-dimensionale Untermannigfaltigkeit mit Rand, so dass $0\notin\del M$. Man zeige:
$$
\int_{\del M}\sigma = \begin{cases}
	\vol(S^{n-1}) & 0\in \mathring{M}, \cr
	0 & 0\notin M.
\end{cases}
$$
{\em Hinweis: Im Fall $0\in\mathring{M}$ wende man den Satz von Stokes auf
	$M\setminus B(0,\eps)$ f\"ur einen offenen Ball $B(0,\eps)$ vom Radius
	$\eps$ mit $\overline{B(0,\eps)}\subset\mathring{M}$ an.} 
%------------------------------------------------------------------------------------------------------------------------
\vspace{2mm}
            \end{prob}
