    \begin{prob}[Aufgabe 5:]{\bf Pathologische Eigenschaften der Teufelstreppe} \vspace{2mm} \\
Sei $C \subset [0, 1]$ die Cantormenge und $f \colon [0, 1] \mapsto [0, 1]$ die Cantor-Funktion aus der vorigen Aufgabe. Man zeige: 
\begin{enumerate}[label = (\alph*)]
\item Die Funktion
%
\[ g \colon [0, 1] \mapsto [0, 1] \colon x \mapsto \frac{f(x) + x}{2} \]
%
ist ein Hom\"oomorphismus, d.h. stetig mit stetiger Umkehrfunktion.
\item F\"ur jedes $E \subset \R$ mit $\lambda^*(E) > 0$ gibt es eine nicht Lebesgue-messbare Teilmenge $W \subset E$.

$Hinweis$: Betrachten Sie die in der Vorlesung konstruierte Vitali-Menge $V$. Zeigen Sie zun�chst, dass jede messbare Teilmenge $A$ von $p + V,  p \in \Q$, eine Nullmenge ist. Folgern Sie daraus, dass $W = (p + V ) \cap E$ f�r ein $p \in \Q$ gew�hlt werden kann.
\item Die Menge $g(C)$ ist eine Borel-Menge mit $\lambda(g(C) ) = 1/2$ und enth\"alt eine nicht Lebesgue-messbare Teilmenge $W \subset g(C)$. Die Urbildmenge $g^{-1}(W)$ ist Lebesgue-messbar aber keine Borel-Menge, somit ist der Hom\"oomorphismus $g^{-1}$ nicht $\mathcal{L}_1$-$\mathcal{L}_1$-messbar.
\end{enumerate}
\end{prob}
