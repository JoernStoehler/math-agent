\begin{exercise}[Alternative Differenzierbarkeit, übernommen von Prof.
  Oldenburg]
  Hier stellen wir einige andere Definitionen von Differenzierbarkeit vor. Zu
  überlegen ist sich, ob diese äquivalent, schwächer oder stärker als die
  übliche Definition mit der $h$-Methode.
  \begin{enumerate}[label=(\alph*)]
  \item Wir nennen eine Funktion $f$ symmetrisch differenzierbar bei $x$, wenn
    \begin{equation*}
      \begin{split}
        f^\sim(x) \eqdef \lim_{h \to 0}{\frac{f(x + h) - f(x-h)}{2h}}
      \end{split}
    \end{equation*}
    existiert.
    \begin{enumerate}[label=(\roman*)]
    \item Berechne die symmetrische Ableitung von $f(x) = x^2$
    \item Untersuche ob aus symmetrischer Differenzierbarkeit normale
      Differenzierbarkeit folgt und andersherum (Tipp: $\abs{x}$)
    \end{enumerate}

  \item Wir nennen eine Funktion $f$ skalar differenzierbar bei $x \neq 0$, wenn
    \begin{equation*}
      \begin{split}
        f^\circledast(x) \eqdef \lim_{q \to 1}{\frac{f(qx) - f(x)}{qx - x}}
      \end{split}
    \end{equation*}
    existiert.
    \begin{enumerate}[label=(\roman*)]
    \item Berechne die symmetrische Ableitung von $f(x) = x^2$ und
      $g(x) = \log (x)$
    \item Untersuche ob aus skalarer Differenzierbarkeit normale
      Differenzierbarkeit folgt und andersherum.
    \end{enumerate}
  \end{enumerate}
\end{exercise}
