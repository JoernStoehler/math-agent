%% WARNING: This file contains multiple exercises and should be split into separate files
\newtheorem{prob}{Aufgabe}
\newtheorem{wrongsol}{Falsche L\"osung}[prob]
\newtheorem{sol}{L\"osung}

\includecomment{solution}
\includecomment{tutoren}

\begin{document}
\pagestyle{empty}
\textbf{Universit"at Augsburg} \hfill \textbf{Kai Cieliebak} \\
\textbf{Sommersemester 2020} \hfill \textbf{Julius Natrup} \\
\textbf{Analysis II} \hfill \textbf{Kathrin Helmsauer}
\vspace{0.5cm}
\begin{center} \textbf{\Large \"Ubungsblatt 1} \end{center}
\vspace{1cm}


\begin{prob} 
Berechne die folgenden Integrale f\"ur $a<b$ mittels partieller Integration:

\begin{enumerate}[label=(\alph*)]
\item $\int_a^b x\sin(x)\,dx$;
\item $\int_a^b x^2\log(x)\,dx$, wobei $0<a$;
\item $\int_a^b x^n e^x\,dx$, wobei $n\in\N$;
\item $\int_a^b \sin(x)\cos(x)\,dx$;
\item $\int_a^b e^x\sin(x)\,dx$;
\item $\int_a^b \arctan(x)\,dx$.
\end{enumerate}
\end{prob}

\vspace{0.3cm}

\begin{prob}
Betrachte die Integrale
$$
   I_m := \int_0^{\pi/2}\sin^mx\,dx,\qquad m\in\N_0.
$$
\begin{enumerate}[label=(\alph*)]
\item Beweise durch partielle Integration die Rekursionsformel
$$
   I_m = \frac{m-1}{m}I_{m-2},\qquad m\geq 2.
$$
\item Folgere aus (a) folgende Formeln f\"ur $n\in\N_0$:
\begin{align*}
   I_{2n} &= \frac{(2n-1)(2n-3)\cdots 3\cdot 1}{(2n-0)(2n-2)\cdots 4\cdot 2}\cdot\frac{\pi}{2},\qquad
   I_{2n+1} = \frac{(2n-0)(2n-2)\cdots 4\cdot 2}{(2n+1)(2n-1)\cdots 5\cdot 3}\cdot 1.
\end{align*}
\item Beweise: 
%direkt aus der Definition von $I_m$: 
Die Folge $(I_m)$ ist streng monoton fallend und $\lim_{m\to\infty}I_m=0$.
\item Zeige $\lim_{n\to\infty}\frac{I_{2n+1}}{I_{2n}}=1$ und folgere
daraus die {\em Wallis'sche Produktdarstellung}
$$
   \frac{\pi}{2} = \prod_{n=1}^\infty \frac{4n^2}{4n^2-1} = \frac{2}{1} \cdot \frac{2}{3} \cdot \frac{4}{3} \cdot \frac{4}{5} \cdot \frac{6}{5} \cdot \frac{6}{7} \cdot \cdots
$$
\end{enumerate}
%Wir definieren $I(n):=\int_0^\pi\sin(x)^n\,dx$.
%\begin{enumerate}[label=(\alph*)]
%\item Berechne rekursiv mittels partieller Integration die Werte von $I(n)$ für alle $n\in\N$
%\item Zeige, dass $\lim_{n\to\infty} \frac{I(n)}{I(n+1)} = 1$
%\item Zeige die \textit{Wallis-Produktformel}:
%$$
%\frac{\pi}{2} = \frac{2}{1} \cdot \frac{2}{3} \cdot \frac{4}{3} \cdot \frac{4}{5} \cdot \frac{6}{5} \cdot \frac{6}{7} \cdot \dots
%$$
%\end{enumerate} 
\end{prob}

\vspace{0.3cm}

\begin{prob}
Berechne die folgenden Integrale mittels Substitution:
\begin{enumerate}[label=(\alph*)]
\item $\int_a^b x\sin(x^2+1)\,dx$ mit dem Ansatz $u=x^2+1$;
\item $\int_a^b \sqrt{1-x^2}\,dx$ f\"ur $-1\leq a<b\leq1$ mit dem Ansatz $x=\sin(t)$;
\item $\int_a^b\frac{1}{1-x^2}\,dx$ f\"ur $-1<a<b<1$ mit dem Ansatz $x=\tanh(t)$;
\item Zeige mit Hilfe einer Substitution, dass für jede Riemann-integrierbare Funktion $f:[a,b]\to\R$ gilt:
$$
\int_a^b\frac{f'(x)}{f(x)}\,dx = \log(|f(b)|)- \log(|f(a)|).
$$
 und bestimme damit das Integral $\int_a^b\tan(x)\,dx$ für $[a,b]\subset(-\pi/2,\pi/2)$.
\end{enumerate} 
\end{prob}

\vspace{0.3cm}

\begin{prob}
Bestimme mittels Polynomdivision und Partialbruchzerlegung Stammfunktionen der folgenden Funktionen:
$$
%\begin{enumerate}[label=(\alph*)] 
(a)\ \frac{x^5}{x-1};\quad
(b)\ \frac{x}{x^3+x^2-x-1};\quad
(c)\ \frac{x}{x^3-x^2+x-1}.
%\end{enumerate} 
$$

\textit{Vorsicht!} Die Nenner bei (b) und (c) sind verschieden!
\end{prob}

\vspace{0.3cm}

\begin{prob}
Berechne den Wert von $\zeta(2)=\sum_{n=1}^\infty n^{-2}$ mit dem
Integralvergleichskriterium bis auf zwei Stellen hinter dem Komma genau.
%Gegeben sei $n\in\N$ mit $2\leq n$. Finde mit Hilfe des Integralvergleichskriteriums obere und untere Schranken für die konvergenten Reihen
%$$
%\sum_{k=1}^\infty k^{-n}
%$$
\end{prob}
