\begin{prob}
(a) Seien $f,g:[a,b]\to\R$ zwei stetige Funktionen mit $g\not\equiv 0$
(d.h.~$\exists t\in[a,b]$ mit $g(t)\neq 0$). Es gelte f\"ur jede
$C^\infty$-Funktion $h:[a,b]\to\R$ mit $h(a)=h(b)=0$ die Implikation
$$
  \int_a^bg(t)h(t)dt=0 \Longrightarrow \int_a^bf(t)h(t)dt=0.
$$
Zeige, dass dann $f=\lambda g$ f\"ur ein $\lambda\in\R$ ist. 

(b) Gegeben seien zwei $C^2$-Lagrangefunktionen $L(t,x,\dot x)$ und
$F(t,x,\dot x)$. Die $C^2$-Funktion $x:[a,b]\to\R^n$ sei ein lokales
Extremum des Wirkungsfunktionals
$$
   S(x) = \int_a^bL(t,x,\dot x)dt
$$
unter Variationen mit festen Endpunkten unter der Nebenbedingung
$$
   \int_a^bF(t,x,\dot x)dt = 0. 
$$
Mache plausibel (ohne rigorosen Beweis): Ist $\frac{\p F}{\p x}-\frac{d}{dt}\frac{\p F}{\p\dot
  x}\not\equiv 0$, so gibt es einen ``Lagrange-Multiplikator''
$\lambda\in\R$, so dass
$$
   \frac{\p L}{\p x}-\frac{d}{dt}\frac{\p L}{\p\dot x}
   = \lambda\Bigl(\frac{\p F}{\p x}-\frac{d}{dt}\frac{\p F}{\p\dot x}\Bigr). 
$$
\end{prob}
