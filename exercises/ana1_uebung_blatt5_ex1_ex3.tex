%% WARNING: This file contains multiple exercises and should be split into separate files
\begin{prob}
  Eine Schnecke startet morgens am linken Ende eines 1 Meter langen Gummibandes.
  Jeden Tag kriecht sie einen Zentimeter nach rechts das Band entlang. Jede
  Nacht (während die Schnecke schläft) wird das Gummiband gleichmäßig um 1 Meter
  gedehnt. Erreicht die Schnecke das rechte Ende des Gummibands?
\end{prob}

\begin{prob}
  Gegeben sei die rekursiv definierte Folge $(f_n)$ mit $f_1 = f_2 = 1$ und
  $f_n = f_{n-1} + f_{n-2}$ für $n \geq 3$.
  \begin{enumerate}[label=(\alph*)]
  \item Untersuche $(f_n)$ auf Konvergenz.
  \item Zeige, dass für alle $n \geq 2$ gilt, dass
    $\frac{3}{2} \leq \frac{f_{n+1}}{f_n} \leq 2$.
  \item Untersuche die Folge $a_n \coloneqq \frac{f_{n+1}}{f_n}$ auf Monotonie.
  \item Untersuche $(a_n)$ auf Konvergenz und berechne den Grenzwert, wenn einer
    existiert.
  \end{enumerate}
\end{prob}

\begin{prob}
  \begin{enumerate}[label=(\alph*)]
  \item Es sei $p > 1$ eine natürliche Zahl. Zeige: Jede positive reelle Zahl $a$
    besitzt eine $p$-adische Entwicklung, das heißt, es gibt eine ganze Zahl
    $k_0$ und eine Folge $(a_k)_{k \geq k_0}$ ganzer Zahlen mit $0 \leq a_k < p$,
    so dass
    \begin{equation*}
      a = \sum_{k=k_0}^{\infty}\frac{a_k}{p^k}.
    \end{equation*}
  \item Zwei $p$-adische Darstellungen
    $\displaystyle\sum_{k=k_0}^{\infty} \frac{a_k}{p^k}$ und
    $\displaystyle\sum_{k=k_1}^{\infty} \frac{b_k}{p^k}$ repräsentieren genau dann
    dieselbe Zahl $a \in \R_+$, wenn entweder $a_k = b_k$ für alle $k \in \Z$
    gilt, oder ein $\ell \in \Z$ existiert, sodass
    \begin{equation*}
      \begin{cases}
        a_k = b_k\ \text{für alle } k < \ell, \\
        a_\ell = b_\ell + 1,                  \\
        a_k = 0\ \text{für alle } k > \ell,   \\
        b_k = p-1\ \text{für alle } k > \ell,
      \end{cases}
    \end{equation*}
    gilt oder dasselbe,  mit den Rollen der $a_k$ und $b_k$ vertauscht, gilt.
  \end{enumerate}
\end{prob}
