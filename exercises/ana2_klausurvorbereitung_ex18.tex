\begin{prob}
Die Fläche eines Dreiecks aufgespannt von zwei Vektoren $a,b \in \R^2$ ist gegeben durch
$$\operatorname{Area}\left(\begin{pmatrix}a_1 \\ a_2\end{pmatrix},\begin{pmatrix}b_1 \\ b_2\end{pmatrix}\right) = \frac{1}{2} \left\rVert \begin{pmatrix}a_1 \\ a_2 \\ 0\end{pmatrix} \times \begin{pmatrix}b_1 \\ b_2 \\ 0\end{pmatrix}\right\rVert$$
Wir erhalten also eine Funktion 
\begin{equation*}
        \begin{split}
            \mathcal{A}:\R^4&\to\R\\
            (a_1, a_2, b_1, b_2) &\mapsto \operatorname{Area}\left(\begin{pmatrix}a_1 \\ a_2 \end{pmatrix}, \begin{pmatrix}b_1 \\ b_2 \end{pmatrix}\right)
        \end{split}
    \end{equation*}
Finde die Maxima von $\mathcal{A}$ unter den Nebenbedingungen $a_1^2+a_2^2=1$, $b_1^2+b_2^2=1$, $a_i,b_i>0$ f"ur $i\in\{1,2\}$.
Interpretiere das Ergebnis geometrisch.
\end{prob}
