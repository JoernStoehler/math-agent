                \begin{prob}[Jordan-messbare Mengen]
%
F�r eine beschr�nkte Menge $A \subset \R^n$ definieren wir die Mengensysteme
%
\[ \mathcal{U}(A) = \left\{ U \subset \R^n  \, | \, U \subset A , \, U \text{ ist endliche Vereinigung von halboffenen Quadern} \right\} \]
%
und
%
\[ \mathcal{O}(A) = \left\{ O \subset \R^n  \, | \, A \subset O , \, O \text{ ist endliche Vereinigung von halboffenen Quadern} \right\}. \]
%
Sei $\lambda$ das Lebesgue-Ma� auf $\R^n$. Die beschr�nkte Menge $A$ hei�t {\em Jordan-messbar}, wenn
%
\[ \forall \eps > 0 \, \exists U \in \mathcal{U}(A) , O \in \mathcal{O}(A) \colon \lambda(O \setminus U) < \eps. \]
%
Ein $\setminus$-stabiles, $\cup$-stabiles Mengensystem, das $\emptyset$ umfasst, hei�t ein {\em Ring}.\footnote{Ein Mengen-Ring ist also bis auf die wom�glich fehlende Eins der Multiplikation $\cap$ eine Mengen-Algebra, was den Namen 'Ring' erkl�rt.} Beweisen Sie:
\begin{enumerate}[label = (\alph*)]
\item Das System $\mathcal{J}(\R^n)$ der Jordan-messbaren Mengen bildet einen Ring.
\item Die Mengenfunktion
	%
	\[ \mathfrak{J} \colon \mathcal{J}(\R^n) \mapsto \left[ 0 , \i \right) \colon A \mapsto \sup_{U \in \mathcal{U}(A) } \lambda(U) = \inf_{O \in \mathcal{O}(A) } \lambda(O) \]
	%
	ist ein wohldefinierter Inhalt.
\item Jede Jordan-messbare Menge ist Lebesgue-messbar und $\mathfrak{J} = \left. \lambda \right|_{ \mathcal{J}(\R^n) }$, d.h. $\mathfrak{J}$ stimmt mit der Einschr�nkung des Lebesgue-Ma�es auf $\mathcal{J}(\R^n)$ �berein.
\end{enumerate}
% Es gibt Jordan-messbare Mengen, die nicht Borel-messbar sind, vgl. Korollar 7.8 bei Elstrodt.
% $Bemerkung$: Jordan-messbare Mengen verhalten sich zum mehrdimensionalen Riemann-Integral wie Borel-Mengen zum Lebesgue-Integral. Man beachte die Analogie zu Unter- und Obersummen.
%------------------------------------------------------------------------------------------------------------------------
\vspace{2mm}
                \end{prob}
