                        \begin{prob}[ Charakterisierung Lebesgue-messbarer Mengen}
Sei $\lambda$ das Lebesgue-Ma{\ss} auf dem $\R^n$ und $\lambda^*$ das zugeh\"orige \"au\ss ere Ma\ss. Man zeige: F\"ur $C\subset\R^n$ sind \"aquivalent:
\begin{enumerate}[label = (\alph*)]
	\item $C$ ist Lebesgue-messbar, d.h. $C \in \mathcal{L}(\R^n)$.
	\item F\"ur jedes $\eps > 0$ gibt es eine offene Menge $B \subset \R^n$ mit $C \subset B$ und $\lambda^*(B \setminus C) < \eps$.
	\item Es gibt eine {\em $G_\delta$-Menge} (d.h. ein abz\"ahlbarer Durchschnitt offener Mengen) $B \subset \R^n$ mit $C \subset B$ und $\lambda^*(B \setminus C) = 0$.
	\item F\"ur jedes $\eps > 0$ gibt es eine abgeschlossene Menge $B \subset \R^n$ mit $B \subset C$ und $\lambda^*(C \setminus B) < \eps$.
	\item Es gibt eine {\em $F_\sigma$-Menge} (d.h. eine abz\"ahlbare Vereinigung abgeschlossener Mengen) $B \subset \R^n$ mit $B \subset C$ und $\lambda^*(C \setminus B ) = 0$. 
	\item Es gibt eine Borel-Menge $B \subset \R^n$ mit $\lambda^*( B \Delta C) = 0$.
\end{enumerate}

                        \end{prob}
