                    \begin{prob}[Parametrisierte Integrale] \\
%
Das Lebesgue-Integral liefert sehr starke hinreichende Bedingungen f�r Stetigkeit und Differenzierbarkeit parametrisierter Integrale. Diese Resultat n�tzen auf vielen Gebieten, z.B. in den Partiellen Differentialgleichungen oder der Variationsrechnung.

Sei $\left( \Omega , \mathcal{A}, \mu \right)$ ein Ma�raum. Zeigen Sie:
\begin{enumerate}[label = (\alph*)]
	\item Sei $\left( X , d \right)$ ein metrischer Raum, $x_0 \in X$ und $f \colon \Omega \times X \mapsto \R$ eine Funktion mit folgenden Eigenschaft:
	\begin{enumerate}[label = (\roman*)]
		\item $\omega \mapsto f( \omega , x )$ ist integrabel f�r alle $x \in X$.
		\item $x \mapsto f( \omega , x )$ ist stetig bei $x_0$ f�r alle $\omega \in \Omega$.
		\item Es existiert eine integrable Funktion $g$ auf $\Omega$ mit
		%
		\[ \left| f( \omega , x ) \right| \le g(\omega) \quad \forall \omega \in \Omega, x \in X. \]
		%
	\end{enumerate}
	Dann ist die Abbildung $F \colon X \mapsto \R$, gegeben durch
	%
	\[ F(x) = \int_\Omega f( \omega , x ) \, d \mu, \]
	%
	in $x_0$ stetig.
	\item Sei $U \subset \R^n$ offen und $f \colon \Omega \times U \mapsto \R$ eine Funktion mit folgenden Eigenschaften:
	\begin{enumerate}[label = (\roman*)]
		\item $\omega \mapsto f( \omega , x )$ ist integrabel f�r alle $x \in X$.
		\item $x \mapsto f( \omega , x )$ ist partiell nach $x_i$ differenzierbar f�r alle $\omega \in \Omega$.
		\item Es existiert eine integrable Funktion $g$ auf $\Omega$ mit
		%
		\[ \left| \del_{x_i} f( \omega , x ) \right| \le g(\omega) \quad \forall \omega \in \Omega, x \in U. \]
		%
	\end{enumerate}
	Dann ist die Abbildung $F \colon U \mapsto \R$, gegeben durch
	%
	\[ F(x) = \int_\Omega f( \omega , x ) \, d \mu, \]
	%
	partiell nach $x_i$ differenzierbar mit
	%
	\[ \del_{x_i} F(x) = \int_\Omega \del_{x_i} f( \omega , x ) \, d \mu.\]
	%
	$Hinweis$: Nutzen Sie entlang passender Folgen $h_k \in \R$ mit $h_k \to 0$ den Mittelwertsatz f�r den Differenzenquotienten
	%
	\[ f_k(\omega) := \frac{f(\omega , x + h_k e_i) - f(\omega , x) }{h_k}. \]
	%
\end{enumerate}

\end{prob}
