\begin{exercise}[Fingerübungen zum Ableiten]
  Zum Aufwärmen ein paar leichte Fingerübungen, die vorallem Schulstoff
  wiederholen sollen.
  \begin{enumerate}[label=(\alph*)]
  \item
    \begin{enumerate}
    \item $f(x) = 3x^2 + 5x + 10$
    \item $g(x) = e^{x^2}+ e^x$
    \item $h(x) = 2^{x}+ x^2$
    \item $i(x) = 2x\log(x) + 2x$
    \item $j(x) = x^{\log(2)} - \frac{1}{x} $
    \item $k(x) = \frac{1+ x^2}{x^2 - 1}$
    \end{enumerate}
  \item Zeige, dass bei einem Polynom vom Grad $m$, also einer Funktion der Form
    \begin{equation*}
      \begin{split}
        p(x) = a_m x^{m} + \cdots + a_1 x + a_0
      \end{split}
    \end{equation*}
    Alle Ableitungen nach der $m+1$-sten konstant $0$ sind. Insbesondere ist
    damit jedes Polynom unendlich oft differenzierbar.
  \end{enumerate}
\end{exercise}
