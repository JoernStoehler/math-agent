\begin{prob}[Rand-Normalenfeld]
%
Sei $M\subset\R^n$ eine $k$-dimensionale Untermannigfaltigkeiten mit Rand. Man zeige:

(a) F\"ur $a \in \del M$ ist $T_a(\del M)$ ein $(k-1)$-dimensionaler Untervektorraum von $T_a M$, $T_a^\perp M$ ist ein $(n-k)$-dimensionaler Untervektorraum von $T_a^\perp(\del M)$, und es gilt
$$
\R^n = T_a^\perp M \oplus (T_aM \cap T_a^\perp(\del M) ) \oplus T_a(\del M).
$$
(b) Man zeige: Es gibt eine eindeutig bestimmte stetige Abbildung $N:\del M\to\R^n$ mit den folgenden Eigenschaften f\"ur jedes $x \in \del M$:
\begin{description}%[label = (i)]
	\item[(i)] $\|N(x)\|=1$;
	\item[(ii)] $N(x)\in T_xM\cap T_x^\perp(\del M)$;
	\item[(iii)] ist $\phi:V\cap H^k\to U\cap M$ eine Karte f\"ur $M$ um $x$ mit
	$\phi(y)=x$ und $w\in\R^k$ der eindeutig bestimmte Vektor mit
	$D\phi(y)w=N(x)$, so ist die erste Komponente von $w$ positiv. 
\end{description}
Die Abbildung $N:\del M\to\R^n$ hei\ss t das {\em \"au\ss ere Normalenfeld zu $\del M$ in $M$}. 
%------------------------------------------------------------------------------------------------------------------------
\vspace{2mm}
\end{prob}
