\begin{prob}
Eine {\em Rotationsfl\"ache} wird in Zylinderkoordinaten
$(r,\varphi,z)$ durch eine Gleichung $r=r(z)$ f\"ur eine Funktion
$r:[-a,a]\to(0,\infty)$ beschrieben, d.h.~die Fl\"ache entsteht durch
Rotation des Graphen der Function $r(z)$ um die $z$-Achse. Ihr
Fl\"acheninhalt ist dann
$$
   A(r) = 2\pi\int_{-a}^ar(z)\sqrt{1+r'(z)^2}\,dz.
$$
L\"ose die Euler-Lagrange-Gleichungen f\"ur das Funktional $A(r)$ auf
$C^2$-Funktionen $r:[-a,a]\to(0,\infty)$ mit festen Endpunkten $r(\pm a)=R$.
Zeige dazu: Es gibt ein $\mu>0$, so dass es f\"ur $R/a>\mu$ genau eine
L\"osung und f\"ur $R/a<\mu$ keine L\"osung gibt. Interpretiere das
Ergebnis geometrisch. 
\end{prob}
