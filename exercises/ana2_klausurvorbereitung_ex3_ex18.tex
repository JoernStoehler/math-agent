%% WARNING: This file contains multiple exercises and should be split into separate files
\begin{prob}
Überprüfe die Funktionenfolge
\begin{equation*}
\begin{split}
f_n:[0,\pi]&\to\R\\
x&\mapsto \frac{\sin(nx)}{n}
\end{split}
\end{equation*}
sowie die Funktionenfolge $f'_n$ auf gleichm"a"sige Konvergenz.
\end{prob}

\vspace{0.3cm}

\begin{prob}
Bestimme alle $x\in \mathbb{R}$, für die die folgende Potenzreihe konvergiert
$$\sum_{n=1}^\infty \frac{x^n}{\sqrt{n}}.$$ 
\end{prob}

\vspace{0.3cm}

\begin{prob}
Berechne das Taylorpolynom  zweiten Grades $T_0^2f(x)$ der Funktion 
\begin{equation*}
\begin{split}
f:\R&\to\R\\
x&\mapsto \log(x^2+1)
\end{split}
\end{equation*}
an der Stelle $x_0:=0$ und schätze den Fehler
$$|f(1)-T_0^2f(1)|$$
mit Hilfe des Lagrangerestglieds nach oben ab. 
\end{prob}

\vspace{0.3cm}

\begin{prob}
(a) Zeige: Der Raum $\R^{m\times n}$ der $m\times n$-Matrizen ist mit der Funktion $d(A,B)=\text{rang}(A-B)$ ein metrischer Raum.

(b) Bestimme f"ur $k\in\N$ die offenen B"alle $B(0,k)$ in $(\R^{m\times n},d)$.

(c) Zeige, dass es keine Norm auf $\R^{m\times n}$ gibt, so dass $\|A-B\|=d(A,B)$.
\end{prob}

\vspace{0.3cm}

\begin{prob}
Es seien $f,g\in\mathcal{C}([0,1])$ und
$$\la f,g\ra:=\integral{0}{1}{f(x)g(x)}{x}.$$
Zeige: $(\mathcal{C}([0,1]),\la\cdot,\cdot\ra)$ ist ein Euklidischer Vektorraum.
\end{prob}

\vspace{0.3cm}

\begin{prob}
(a) Bestimme das Innere, den Abschluss und den Rand der Menge
$$X=\{(x,y)\in\R^2\;\bigl|\;|x|\leq 1,\ |y|<1\}\subset\R^2.$$

(b) Bestimme das Innere, den Abschluss und den Rand desselben $X\subset X$.
\end{prob}

\vspace{0.3cm}

\begin{prob}
(a) Es sei $(X,d)$ ein vollst"andiger nicht-leerer metrischer Raum, $n\in\N$ eine nat"urliche Zahl und
$$f:X\to X$$
eine Abbildung, so dass $d(f^n(x),f^n(y))\leq \theta d(x,y))$ f"ur ein $0<\theta<1$ und alle $x,y\in X$. Zeige, dass dann $f$ einen eindeutigen Fixpunkt hat.

(b) Zeige, dass die Funktion
\begin{equation*}
\begin{split}
f:(0,\infty)&\to(0,\infty)\\
x&\mapsto 1+\frac{1}{x}
\end{split}
\end{equation*}
 genau einen Fixpunkt hat und bestimme diesen.
\end{prob}

\vspace{0.3cm}

\begin{prob}
(a) Untersuche die Menge 
$$B:=\{x\in\R^n\;\bigl|\;0<|x|\leq 1\} \subset \R^n$$ auf Kompaktheit.

(b) Zeige: Die Funktion 
\begin{equation*}
\begin{split}
f:B&\to\R\\
x&\mapsto \frac{1}{\|x\|^2}
\end{split}
\end{equation*}
nimmt auf $B$ ihr Minimum an.
\end{prob}

\vspace{0.3cm}

\begin{prob}
Zeichne schematisch die Kurve
\begin{equation*}
\begin{split}
\gamma:[0,2]&\to\R^2\\
t&\mapsto \begin{pmatrix}
t\cos(2\pi t)\\
t\sin(2\pi t)\\
\end{pmatrix}
\end{split}
\end{equation*}
und zeige, dass sie einfach und regul"ar ist.
\end{prob}

\vspace{0.3cm}

\begin{prob}
Bestimme die Divergenz und Rotation des Vektorfelds 
    \begin{equation*}
        \begin{split}
            f:\R^3&\to\R^3\\
            (x,y,z)&\mapsto 
            \begin{pmatrix}
            x-y+z\\
            x+y+z\\
            z
            \end{pmatrix}.
        \end{split}
    \end{equation*}
\end{prob}

\vspace{0.3cm}

\begin{prob}
Berechne die Jacobimatrix der Funktion 
    \begin{equation*}
        \begin{split}
            f:\R^4&\to\R^2\\
            (w,x,y,z)&\mapsto 
            \begin{pmatrix}
            \sin(w)\sin(z)\\
            e^{x-y}
            \end{pmatrix}
        \end{split}
    \end{equation*}
    und bestimme ihren Rang in Abh"angigkeit von $(w,x,y,z)$.
\end{prob}


\vspace{0.3cm}

\begin{prob}
Untersuche die Funktion 
\begin{equation*}
        \begin{split}
            f:\R^2&\to\R\\
            (x,y)&\mapsto \begin{cases}
						x^3y^2\sin\left(\frac{1}{x^2+y}\right),& x^2+y\neq 0\\
						0,& x^2+y=0
						\end{cases}
        \end{split}
    \end{equation*}
 am Punkt $(0,0)$ auf Stetigkeit, partielle Differenzierbarkeit und totale Differenzierbarkeit. Berechne f"ur alle $v\in\R^2$ mit $\|v\|=1$ die Richtungsableitungen $D_vf(0,0)$ (falls sie existieren).
\end{prob}

\vspace{0.3cm}

\begin{prob}
Untersuche die Funktion
\begin{equation*}
        \begin{split}
            f:\R^3&\to\R^3\\
            (x,y,z)&\mapsto (yz, -xz, xy)
        \end{split}
    \end{equation*}
 auf Lipschitz-Stetigkeit.
\end{prob}

\vspace{0.3cm}

\begin{prob}
Berechne das Taylorpolynom zweiten Grades $T_0^2f(x)$ am Punkt $x_0=0$ der Funktion
\begin{equation*}
        \begin{split}
            f:\R^4&\to\R\\
            (w,x,y,z) &\mapsto wy-x^2z
        \end{split}
    \end{equation*}
\end{prob}

\vspace{0.3cm}

\begin{prob}
Finde die lokalen und globalen Minima und Maxima der Funktion
\begin{equation*}
        \begin{split}
            f:\R^2&\to\R\\
            (x,y) &\mapsto \cos(x)+\sin(y)
        \end{split}
    \end{equation*}
und interpretiere das Ergebnis geometrisch.
\end{prob}

\vspace{0.3cm}

\begin{prob}
Die Fläche eines Dreiecks aufgespannt von zwei Vektoren $a,b \in \R^2$ ist gegeben durch
$$\operatorname{Area}\left(\begin{pmatrix}a_1 \\ a_2\end{pmatrix},\begin{pmatrix}b_1 \\ b_2\end{pmatrix}\right) = \frac{1}{2} \left\rVert \begin{pmatrix}a_1 \\ a_2 \\ 0\end{pmatrix} \times \begin{pmatrix}b_1 \\ b_2 \\ 0\end{pmatrix}\right\rVert$$
Wir erhalten also eine Funktion 
\begin{equation*}
        \begin{split}
            \mathcal{A}:\R^4&\to\R\\
            (a_1, a_2, b_1, b_2) &\mapsto \operatorname{Area}\left(\begin{pmatrix}a_1 \\ a_2 \end{pmatrix}, \begin{pmatrix}b_1 \\ b_2 \end{pmatrix}\right)
        \end{split}
    \end{equation*}
Finde die Maxima von $\mathcal{A}$ unter den Nebenbedingungen $a_1^2+a_2^2=1$, $b_1^2+b_2^2=1$, $a_i,b_i>0$ f"ur $i\in\{1,2\}$.
Interpretiere das Ergebnis geometrisch.
\end{prob}
