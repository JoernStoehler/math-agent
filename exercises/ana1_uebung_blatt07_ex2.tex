\begin{exercise}[Anwendung der Werkzeuge]
  Die folgenden Reihen sind auf Konvergenz zu untersuchen:
  \begin{enumerate}[label=(\alph*)]
  \item $\displaystyle \sum_{n = 1}^\infty \frac{1}{n^\alpha}$ für $\alpha \in \Q$
  \item $\displaystyle \sum_{n = 1}^\infty \frac{1}{1 + \sqrt{n}}$
  \item $\displaystyle \sum_{n = 1}^\infty \frac{1+n}{n+n^2}$
  \item $\displaystyle \sum_{k = 1}^\infty \frac{a^k}{k!}$ für $a \in \R$
  \item $\displaystyle \sum_{n = 1}^\infty \frac{1}{(n+1)2^n}$
  \item $\displaystyle \sum_{k = 1}^\infty \frac{k!}{\left( 2k \right)!}$
  \item $\displaystyle \sum_{n = 1}^\infty \frac{n! 2^n}{n^n}$
  \item $\displaystyle \sum_{n \in \Z} (-1)^n \cdot \frac{n }{n^2 + 1}$
  \item $\displaystyle \sum_{n = 1}^\infty \left(^n - \left( 2+\frac{1}{n} \right)^{n-1} \right)$
  \item $\displaystyle \sum_{n = 1}^\infty \left( \frac{1}{n} \sum_{k = 1}^n (-1)^{k+1} \cdot (2k-1) \right)$
  \item $\displaystyle \sum_{n = 1}^\infty (-1)^{2n+1}\cdot \frac{1}{2n+1}$
  \end{enumerate}
  Dies sollte man einmal in seinem Leben gemacht haben, dann kann man die
  Konvergenzkriterien anwenden.
\end{exercise}
