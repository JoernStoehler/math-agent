\begin{exercise}[Mächtigkeit von Mengen]
  Wir wollen einige (auch für die Übungsaufgaben sehr relevante)
  Gleichmächtigkeitsaussagen zeigen. Wir schreiben dabei für \enquote{$M$ ist
    gleichmächtig zu $N$} kurz \enquote{$M \cong N$}.
  \begin{enumerate}[label=(\alph*)]
  \item Zum Aufwärmen wollen wir uns überlegen dass für jede unendliche Menge
    $M$ und jedes $a \not\in M$ gilt, dass
    \begin{equation*}
      \begin{split}
        M \cong M\cup\{a\}
      \end{split}
    \end{equation*}
  \item Man überlege sich nun die einfache Folgerung, dass für $a < b$ gilt,
    dass
    \begin{equation*}
      \begin{split}
        [a,b] \cong (a,b) \cong [a,b) \cong (a,b]
      \end{split}
    \end{equation*}
  \item Man zeige nun, dass jedes Intervall gleichmächtig zu jedem anderen ist,
    also seien $a < b$ gegeben, zeige man
    \begin{equation*}
      \begin{split}
        [a,b] \cong [0,1]
      \end{split}
    \end{equation*}
    mittels der affin linearen Transformation
    \begin{center}
      \begin{tikzcd}[row sep=tiny]
        [0,1]
        \arrow [r, "\psi"]
        & {[a,b]} \\
        x
        \arrow [r, mapsto]
        & a+x(b-a) \\
      \end{tikzcd}
    \end{center}
    Man zeige nun Transitivität von $\cong$ und folgere nun mit (b) die
    geforderte Aussage.
  \item Nun wollen wir noch zeigen, dass
    \begin{equation*}
      \begin{split}
        (0,1) \cong \R^+\setminus \{0\}
      \end{split}
    \end{equation*}
    gilt. Man überlege sich dazu eine geeignete Bijektion und zeige alles
    Nötige.
  \end{enumerate}
\end{exercise}
