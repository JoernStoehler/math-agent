\begin{prob}
  \begin{enumerate}[label=(\alph*)]
  \item Zeige für $z, w \in \C$: $\Re(z\conj{w})$ ist das aus der Schule
    bekannte Skalarprodukt von $z$ und $w$, interpretiert als Vektoren in der
    Ebene. Insbesondere ist $\Re(z\conj{w}) = 0$ genau dann, wenn $z \perp w$.
  \item Formuliere und beweise den Satz des Pythagoras für rechtwinklige
    Dreiecke in $\mathbb{C}$.
  \item Formuliere und beweise den Satz des Thales in $\C$.
  \item Seien $a, b, c$ drei komplexe Zahlen mit $\abs{a} = \abs{b} = \abs{c}$.
    Zeige: $a$, $b$ und $c$ bilden die Ecken eines gleichseitigen Dreiecks genau
    dann, wenn $a+b+c = 0$ gilt .
  \end{enumerate}
\end{prob}
