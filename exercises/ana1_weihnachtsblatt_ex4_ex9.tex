%% WARNING: This file contains multiple exercises and should be split into separate files
\begin{prob}[Injektive Abbildungen]
  Ziel dieser Aufgabe ist der Beweis des folgenden Satzes:
  \begin{theorem*}
    Sind $A, B$ zwei Mengen und gibt es zwei injektive Abbildungen
    $f \colon A \to B$ und $g \colon B \to A$, so gibt es auch eine bijektive
    Funktion $h \colon A \to B$. Insbesondere haben damit $A$ und $B$ die
    gleiche Mächtigkeit.
  \end{theorem*}
  Für endliche Mengen $A, B$ kennst Du diese Aussage bereits, wir wollen diese
  Aussage nun für beliebige Mengen zeigen. Wir nennen ein Element $b \in B$
  \emph{einsam}, wenn es kein $a \in A$ mit $f(a) = b$ gibt. Weiter nennen wir
  ein Element $b_1 \in B$ einen \emph{Nachfahren} von $b_0 \in B$, falls
  $b_1 = (f \circ g)^n (b_0)$ für ein $n \in \N_0$ gilt. Wir definieren nun eine
  Abbildung $h \colon A \to B$ durch
  \[
    a \mapsto
    \begin{cases}
      g^{-1}(a) & f(a) \text{ ist der Nachfahre eines einsamen Elements} \\
      f(a)      & \text{sonst}                                           \\
    \end{cases}
  \]
  Zeige nun das Folgende:
  \begin{enumerate}[label=(\roman*)]
  \item $h$ ist wohldefiniert
  \item $h$ ist surjektiv
  \item $h$ ist injektiv
  \end{enumerate}
  Zeige nun davon ausgehend, dass wir auch die Aussage aus den folgenden
  Voraussetzungen schließen können:
  \begin{enumerate}
  \item Es gibt surjektive Abbildungen $f \colon A \to B$ und $g \colon B
    \to A$
  \item Es gibt eine surjektive Abbildung $f \colon A \to B$ und eine injektive
    Abbildung $g \colon A \to B$
  \end{enumerate}
\end{prob}

\begin{prob}
  Sei $\mathbf{K}_p$ ein Körper mit Charakteristik $p>0$. Wir betrachten im
  folgenden die Abbildung
  \begin{center}
    \includestandalone{tikz/cd_koerper2}
  \end{center}
  etwas näher. Dabei werden wir viele schöne Eigenschaften zeigen können, die
  einem hoffentlich im weiteren mathematischen Leben noch das eine oder andere Mal
  begegnen.

  \begin{enumerate}[label=(\alph*)]
  \item \label{itm:fieldhom}Lasst uns mit einigen Überlegungen zur
    Wohldefiniertheit der Abbildung beginnen: Zu zeigen ist
    \begin{enumerate}[label=\roman*.]
    \item $f(1) = 1$
    \item $f(0) = 0$
    \item $\fa{x, y \in \mathbf{K_p}} f(x \cdot y) = f(x) \cdot f(y)$
    \item $\fa{x, y \in \mathbf{K_p}} f(x + y) = f(x) + f(y)$
    \end{enumerate}
    Hiermit haben wir eigentlich vor allem gezeigt, dass $f_p$ ein
    \emph{Körperhomomorphismus} ist, falls du dir das später im Studium noch mal
    anschaust.

  \item Lasst uns jetzt zeigen, dass $f_p$ natürlich ist, soll heißen es ist zu
    beweisen, dass das folgende Diagramm kommutiert, wobei $\mathbf{K}$ und
    $\widetilde{\mathbf{K}}$ zwei Körper der Charakteristik $p$ sind und
    $\phi \colon \mathbf{K} \to \widetilde{\mathbf{K}}$ auch ein
    Körperhomomorphismus ist, also die Eigenschaften in \ref{itm:fieldhom}
    erfüllt:
    \begin{center}
      \includestandalone{tikz/cd_koerper1}
    \end{center}
    Zu zeigen ist also, dass für alle Körperhomomorphismen $\phi$ und für alle
    $x \in \mathbf{K}$ schon die Gleichung
    \begin{equation*}
      \begin{split}
        \left( \phi \circ f_p \right)(x)= \left( f_p \circ \phi \right)(x)
      \end{split}
    \end{equation*}
    erfüllt ist.

  \item Nun kommen wir zum Eingemachten: Im Folgenden ist $f_p$ auf Injektivität
    und Surjektivität zu untersuchen. Wir nennen einen Körper \emph{perfekt},
    wenn $f_p$ ein Ringisomorphismus -- also surjektiv und injektiv -- ist. Gebe
    nun zusätzlich Beispiele an (oder beweise dass es keine gibt) für
    \begin{itemize}
    \item einen nichtperfekten Körper mit Charakteristik $p > 0$
    \item einen perfekten Körper mit Charakteristik $p > 0$
    \end{itemize}

  \item Untersuche die Fixpunktmenge von $f_p$, also die Menge
    \begin{equation*}
      \begin{split}
        M \defn \Set{x \in \mathbf{K}_p | x^p = x}
      \end{split}
    \end{equation*}
    \enquote{Überlege dir}, wann $M$ selbst wieder ein Körper wird.
  \end{enumerate}
\end{prob}

\begin{prob}
  In dieser Aufgabe sollst du untersuchen, an welchen Stellen der Vorlesung es
  entscheidend war in $\R$ zu arbeiten, und an welchen auch $\C$ genügt hätte.
  Genauer: Finde alle Sätze aus den Kapiteln 4, 5 und 7, bei denen die
  Forderung, dass $(a_n)$ eine Folge reeller Zahlen ist, nicht automatisch durch
  die Forderung, dass $(a_n)$ eine Folge komplexer Zahlen ist ersetzt werden
  kann. Kannst du diese Aussagen trotzdem leicht modifiziert retten? Wenn ja,
  gib eine neue gültige Formulierung, f"ur die der fragliche Satz auch in $\C$
  noch wahr ist, an.
\end{prob}

\begin{prob}
  Wiederhole die aus der Vorlesung bekannte Definition vom Häufungspunkt einer
  Folge $\left(a_n\right)_{n \in \N} \subset \R$. Untersuche, ob die folgenden
  alternativen Definitionen äquivalent, schwächer, oder stärker sind, oder gar
  etwas ganz anderes beschreiben.
  \begin{enumerate}[label=(\alph*)]
  \item Wir nennen einen Punkt $x$ HHäufungspunkt der Folge
    $\left(a_n\right)_{n\in\N}$, wenn es für jedes $N \in \N$ und jedes
    $\varepsilon > 0$ ein $n \geqslant N$ gibt mit
    \begin{equation*}
      \abs{a_n - x} < \varepsilon
    \end{equation*}
  \item Wir nennen einen Punkt $x$ HHHäufungspunkt der
    Folge$\left(a_n\right)_{n\in\N}$, wenn für jedes $\eps > 0$ unendlich viele
    $n$ existieren mit
    \begin{equation*}
      \abs{a_n - x} < \eps
    \end{equation*}
  \end{enumerate}
\end{prob}

\begin{prob}
  In dieser Aufgabe sollen noch einige, vielleicht für den ein oder anderen
  interessante Lemmata über Folgen bewiesen werden.
  \begin{enumerate}[label=(\alph*)]
  \item Es sei $\left( a_n \right)_{ n \in \N}$ eine beschränkte Folge und es
    gelte zudem
    \begin{equation*}
      \begin{split}
        \lim_{n \to \infty} \left( a_n - a_{n-1} \right) = 0
      \end{split}
    \end{equation*}
    Dann ist die Menge der Häufungspunkte von $(a_n)_{n \in \N}$ ein
    abgeschlossenes Intervall der Form $\interval{a}{b}$ (wobei wir hier
    außnahmsweise triviale Intervalle der Form $\interval{a}{a}$ zulassen
    wollen). Gilt die analoge Aussage auch, wenn wir auf Beschränktheit
    verzichten?
  \item Seien $\left( a_n \right)_{ n \in \N}$ und
    $\left( b_n \right)_{ n \in \N}$ zwei beschränkte Folgen und es gelte zudem
    \begin{equation*}
      \begin{split}
        \lim_{n \to \infty} (a_n - b_n) = 0
      \end{split}
    \end{equation*}
    Dann folgt schon, dass $\left( a_n \right)_{n \in \N}$ und
    $\left( b_n \right)_{n \in \N}$ die selben Häufungspunkte haben.
  \item Es sei $\left( a_n \right)_{n \in \N}$ eine nicht negative Folge reeller
    Zahlen, deren Cesaro-Limes gegen Null geht, also
    \begin{equation*}
      \begin{split}
        \lim_{n \to \infty}{\left( \frac{1}{n} \sum_{ k=1 }^{ n } a_k \right)}
        =
        0
      \end{split}
    \end{equation*}
    Dann gibt es eine Teilfolge $\left( a_{n_k} \right)_{n_k \in \N}$ mit
    \begin{equation*}
      \begin{split}
        \lim_{k \to \infty} a_{n_k} = 0
      \end{split}
    \end{equation*}
    Stimmt die Aussage auch, wenn $\left( { a }_{ n } \right)_{ n \in \N}$
    negative Werte annehmen darf?
  \item Es sei wieder $\left( a_n \right)_{n \in \N}$ eine Folge reeller Zahlen.
    Konvergiert $\left( a_n \right)_{n \in \N}$ gegen ein $m$ und divergiert
    jedoch $\left( \floor{a_n} \right)_{n \in \N}$, so ist $m \in \Z$.
  \item Es sei $\left( a_n \right)_{n \in \N}$ eine beschränkte Folge reeller
    Zahlen mit abzählbar unendlich vielen Häufungspunkten. Sei nun
    $\left( b_k \right)_{k \in \N}$ eine Folge von solcher Häufungspunkte, die
    gegen ein $b \in \R$ konvergiert. Dann folgt, dass $b$ auch schon ein
    Häufungspunkt der Folge $\left( a_n \right)_{n \in \N}$ war.
  \item Finde eine divergente Folge $\left( a_n \right)_{n \in \N}$, sodass für
    alle $k > 1$ die Teilfolgen
    \begin{equation*}
      \begin{split}
        \left( a_{kn} \right)_{n \in \N}
      \end{split}
    \end{equation*}
    konvergieren und beweise die Richtigkeit deiner Aussage. Wenn man analog
    fordert, dass für alle streng monotonen Funktionen $\phi \colon \N \to \N$
    mit
    \begin{equation*}
      \begin{split}
        \frac{1}{n} \phi(n) > \theta > 1
      \end{split}
    \end{equation*}
    die Teilfolge $\left( a_{\phi(n)} \right)_{n \in \N}$ konvergiert, folgt
    dann, dass $\left( a_n \right)_{n \in \N}$ konvergiert? Wenn nicht, was muss
    man für die Funktionen $\phi$ fordern, damit die Folgerung gültig ist?
  \end{enumerate}
\end{prob}

\begin{prob}
  Wir wollen rekursive Folgen studieren und in speziellen Fällen eine direkte
  Darstellung angeben.
  \begin{enumerate}[label=(\alph*)]
  \item Sei dazu $(a_n)_{n \in \N}$ eine Folge und $d \in \N$, so dass die
    ersten $d$ Folgenglieder $a_0, \dots, a_{d - 1}$ bereits gegeben sind und es
    $c_1, \dots, c_d \in \R$ gibt, so dass für alle weiteren Folgenglieder mit
    $n \geq d$ gilt
    \begin{equation*}
      a_n
      =
      \sum_{i = 1}^d c_i a_{n - i}.
    \end{equation*}
    Wir betrachten ferner das Polynom
    \begin{equation*}
      p(x)
      \coloneqq
      x^d - \sum_{i = 1}^d c_i x^{d - i}.
    \end{equation*}
    Nach dem \emph{Fundamentalsatz der Algebra} hat dieses Polynom $d$
    Nullstellen $r_1, \dots, r_d \in \C$. Beweise: Gibt es nun weiter
    $k_1, \dots, k_d \in \C$, so dass
    \begin{equation*}
      \fa{i \in \{0, \dots, d - 1\}} a_i = \sum_{j = 1}^d k_j r_j^i
    \end{equation*}
    gilt, dann ist schon für alle $n \in \N$:
    \begin{equation*}
      a_n = \sum_{j = 1}^d k_j r_j^n
    \end{equation*}

  \item Die \emph{Lukaszahlen} sind rekursiv über $f_0 = 2, \ f_1 = 1$ und
    $f_n = f_{n - 1} + f_{n - 2}$ für $n \geq 2$ definiert. Finde nun eine
    explizite Formel für die $n$-te Lukaszahl.

  \item Finde eine explizite Formel für die $n$-te Fibonaccizahl.
  \end{enumerate}
\end{prob}
