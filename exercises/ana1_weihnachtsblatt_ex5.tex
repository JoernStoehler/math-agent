\begin{prob}
  Sei $\mathbf{K}_p$ ein Körper mit Charakteristik $p>0$. Wir betrachten im
  folgenden die Abbildung
  \begin{center}
    \includestandalone{tikz/cd_koerper2}
  \end{center}
  etwas näher. Dabei werden wir viele schöne Eigenschaften zeigen können, die
  einem hoffentlich im weiteren mathematischen Leben noch das eine oder andere Mal
  begegnen.

  \begin{enumerate}[label=(\alph*)]
  \item \label{itm:fieldhom}Lasst uns mit einigen Überlegungen zur
    Wohldefiniertheit der Abbildung beginnen: Zu zeigen ist
    \begin{enumerate}[label=\roman*.]
    \item $f(1) = 1$
    \item $f(0) = 0$
    \item $\fa{x, y \in \mathbf{K_p}} f(x \cdot y) = f(x) \cdot f(y)$
    \item $\fa{x, y \in \mathbf{K_p}} f(x + y) = f(x) + f(y)$
    \end{enumerate}
    Hiermit haben wir eigentlich vor allem gezeigt, dass $f_p$ ein
    \emph{Körperhomomorphismus} ist, falls du dir das später im Studium noch mal
    anschaust.

  \item Lasst uns jetzt zeigen, dass $f_p$ natürlich ist, soll heißen es ist zu
    beweisen, dass das folgende Diagramm kommutiert, wobei $\mathbf{K}$ und
    $\widetilde{\mathbf{K}}$ zwei Körper der Charakteristik $p$ sind und
    $\phi \colon \mathbf{K} \to \widetilde{\mathbf{K}}$ auch ein
    Körperhomomorphismus ist, also die Eigenschaften in \ref{itm:fieldhom}
    erfüllt:
    \begin{center}
      \includestandalone{tikz/cd_koerper1}
    \end{center}
    Zu zeigen ist also, dass für alle Körperhomomorphismen $\phi$ und für alle
    $x \in \mathbf{K}$ schon die Gleichung
    \begin{equation*}
      \begin{split}
        \left( \phi \circ f_p \right)(x)= \left( f_p \circ \phi \right)(x)
      \end{split}
    \end{equation*}
    erfüllt ist.

  \item Nun kommen wir zum Eingemachten: Im Folgenden ist $f_p$ auf Injektivität
    und Surjektivität zu untersuchen. Wir nennen einen Körper \emph{perfekt},
    wenn $f_p$ ein Ringisomorphismus -- also surjektiv und injektiv -- ist. Gebe
    nun zusätzlich Beispiele an (oder beweise dass es keine gibt) für
    \begin{itemize}
    \item einen nichtperfekten Körper mit Charakteristik $p > 0$
    \item einen perfekten Körper mit Charakteristik $p > 0$
    \end{itemize}

  \item Untersuche die Fixpunktmenge von $f_p$, also die Menge
    \begin{equation*}
      \begin{split}
        M \defn \Set{x \in \mathbf{K}_p | x^p = x}
      \end{split}
    \end{equation*}
    \enquote{Überlege dir}, wann $M$ selbst wieder ein Körper wird.
  \end{enumerate}
\end{prob}
