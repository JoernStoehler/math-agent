\begin{exercise} % Aufgabe 6
  Auf dem aktuellen Übungsblatt befindet sich folgende Aufgabe: Beweise die
  \emph{Ungleichungen zwischen harmonischem, geometrischem und arithmetischen
    Mittel}: Für alle $\lambda_1, \ldots, \lambda_n > 0$ mit
  $\sum_{i=1}^n \lambda_i = 1$ und alle $x_1, \ldots, x_n > 0$ gilt:
  \begin{equation*}
    \left( \sum_{i = 1}^n \frac{\lambda_i}{x_i} \right)^{-1}
    \leq
    \prod_{i = 1}^n x_i^{\lambda_i}
    \leq
    \sum_{i = 1}^n \lambda_i x_i
  \end{equation*}
  Wir wollen nun in der Übung einen Spezialfall dieser Aufgabe nachweisen.
  Hierfür sei $\lambda_1 = \lambda_2 = \frac{1}{2},$ und $x_1, x_2>0.$ Zeige
  demnach:
  \begin{equation*}
    \left(\frac{0,5}{x_1} + \frac{0,5}{x_2}\right)^{-1}
    \leq
    \sqrt{x_1} \cdot \sqrt{x_2}
    \leq
    \frac{1}{2} \cdot x_1 + \frac{1}{2} \cdot x_2
  \end{equation*}
\end{exercise}
