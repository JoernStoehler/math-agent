\begin{prob}
  Die \textit{Charakteristik} eines Körpers $\K$, geschrieben $\chi(\K)$, ist
  das kleinste $n \in \N$ so, dass $n \cdot 1 = 0$. Wenn es kein solches $n$
  gibt, definieren wir $\chi(\K) := 0$. \\
  Zeige, dass...
  \begin{enumerate}[label=(\alph*)]
  \item $\chi(\K) \neq 0 \iff \chi(\K)=p$ für eine Primzahl $p$;
  \item $\chi(\K)= 0 \iff \K$ enthält $\Q$ als Teilkörper;
  \item $\fa{a \in \K} \chi(\K) \cdot a = 0$;
  \item falls die Anzahl der Elemente von $\K$ endlich ist, so gilt
    $\chi(\K) > 0$ und $\chi(\K)$ ist ein Teiler der Anzahl der Elemente von $\K$
  \end{enumerate}
\end{prob}
