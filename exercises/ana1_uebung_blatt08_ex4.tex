\begin{exercise}[Umordnung von Reihen]
  Wir wollen zeigen, dass man Reihen nicht ohne Weiteres umsortieren kann. Dazu
  betrachten wir die alternierende harmonische Reihe. Diese konvergiert
  bekanntermaßen, jedoch kann mit geschicktem Anordnen erreicht werden, dass
  diese divergiert. Finde eine Umordnung, die dies erreicht.\\
  \textit{Hinweis}: Es darf vorrausgesetzt werden, dass
  \[
    \fa{k \in \N} \sum_{n=k}^{2k} \frac{1}{2n+1} \geq \frac{1}{k+1}
  \]
  gilt, diese Aussage ist äquivalent zu
  \begin{align}
    \label{eq:harmonic-series}
    \sum_{n=k}^{2k} \frac{1}{2n+1} - \frac{1}{2(k+1)} \geq \frac{1}{2(k+1)}.
  \end{align}
  Das Berechnen der ersten 4 Werte macht die Nützlichkeit dieser Formel klar.
  Mithilfe von \Cref{eq:harmonic-series} kann die Umsortierung der
  alternierenden harmonischen Reihe gegen
  \[
    \frac{1}{2} \sum_{n=1}^\infty \frac{1}{n}
  \]
  abgeschätzt werden.
\end{exercise}
