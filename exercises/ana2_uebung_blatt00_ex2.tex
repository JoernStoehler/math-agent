\begin{prob} 
F\"ur $a<b$ bezeichne $\mathcal{T}[a,b]$ die Menge der Treppenfunktionen und $\mathcal{R}[a,b]$ die Menge der Riemann-integrierbaren Funktionen $[a,b]\to\R$. Zeige:

(a) $\mathcal{R}[a,b]$ ist ein $\R$-Vektorraum und $\mathcal{T}[a,b]\subset\mathcal{R}[a,b]$ ist ein Untervektorraum. 

(b) Das Riemann-Integral definiert eine lineare Abbildung $\mathcal{R}[a,b]\to\R$.

(c) Die Dimension von $\mathcal{T}[a,b]$ (und damit auch von $\mathcal{R}[a,b]$) ist \"uberabz\"ahlbar unendlich. 
\textit{Hinweis:}  Nimm per Widerspruch an, dass der Vektorraum abzählbar unendlich ist, und untersuche dann die Menge der möglichen Sprungstellen.
\end{prob}
