\begin{prob}
  In dieser Aufgabe sollen noch einige, vielleicht für den ein oder anderen
  interessante Lemmata über Folgen bewiesen werden.
  \begin{enumerate}[label=(\alph*)]
  \item Es sei $\left( a_n \right)_{ n \in \N}$ eine beschränkte Folge und es
    gelte zudem
    \begin{equation*}
      \begin{split}
        \lim_{n \to \infty} \left( a_n - a_{n-1} \right) = 0
      \end{split}
    \end{equation*}
    Dann ist die Menge der Häufungspunkte von $(a_n)_{n \in \N}$ ein
    abgeschlossenes Intervall der Form $\interval{a}{b}$ (wobei wir hier
    außnahmsweise triviale Intervalle der Form $\interval{a}{a}$ zulassen
    wollen). Gilt die analoge Aussage auch, wenn wir auf Beschränktheit
    verzichten?
  \item Seien $\left( a_n \right)_{ n \in \N}$ und
    $\left( b_n \right)_{ n \in \N}$ zwei beschränkte Folgen und es gelte zudem
    \begin{equation*}
      \begin{split}
        \lim_{n \to \infty} (a_n - b_n) = 0
      \end{split}
    \end{equation*}
    Dann folgt schon, dass $\left( a_n \right)_{n \in \N}$ und
    $\left( b_n \right)_{n \in \N}$ die selben Häufungspunkte haben.
  \item Es sei $\left( a_n \right)_{n \in \N}$ eine nicht negative Folge reeller
    Zahlen, deren Cesaro-Limes gegen Null geht, also
    \begin{equation*}
      \begin{split}
        \lim_{n \to \infty}{\left( \frac{1}{n} \sum_{ k=1 }^{ n } a_k \right)}
        =
        0
      \end{split}
    \end{equation*}
    Dann gibt es eine Teilfolge $\left( a_{n_k} \right)_{n_k \in \N}$ mit
    \begin{equation*}
      \begin{split}
        \lim_{k \to \infty} a_{n_k} = 0
      \end{split}
    \end{equation*}
    Stimmt die Aussage auch, wenn $\left( { a }_{ n } \right)_{ n \in \N}$
    negative Werte annehmen darf?
  \item Es sei wieder $\left( a_n \right)_{n \in \N}$ eine Folge reeller Zahlen.
    Konvergiert $\left( a_n \right)_{n \in \N}$ gegen ein $m$ und divergiert
    jedoch $\left( \floor{a_n} \right)_{n \in \N}$, so ist $m \in \Z$.
  \item Es sei $\left( a_n \right)_{n \in \N}$ eine beschränkte Folge reeller
    Zahlen mit abzählbar unendlich vielen Häufungspunkten. Sei nun
    $\left( b_k \right)_{k \in \N}$ eine Folge von solcher Häufungspunkte, die
    gegen ein $b \in \R$ konvergiert. Dann folgt, dass $b$ auch schon ein
    Häufungspunkt der Folge $\left( a_n \right)_{n \in \N}$ war.
  \item Finde eine divergente Folge $\left( a_n \right)_{n \in \N}$, sodass für
    alle $k > 1$ die Teilfolgen
    \begin{equation*}
      \begin{split}
        \left( a_{kn} \right)_{n \in \N}
      \end{split}
    \end{equation*}
    konvergieren und beweise die Richtigkeit deiner Aussage. Wenn man analog
    fordert, dass für alle streng monotonen Funktionen $\phi \colon \N \to \N$
    mit
    \begin{equation*}
      \begin{split}
        \frac{1}{n} \phi(n) > \theta > 1
      \end{split}
    \end{equation*}
    die Teilfolge $\left( a_{\phi(n)} \right)_{n \in \N}$ konvergiert, folgt
    dann, dass $\left( a_n \right)_{n \in \N}$ konvergiert? Wenn nicht, was muss
    man für die Funktionen $\phi$ fordern, damit die Folgerung gültig ist?
  \end{enumerate}
\end{prob}
