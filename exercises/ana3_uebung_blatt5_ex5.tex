                        \begin{prob}[Volumen der Kugel] \\
F�r $r > 0$ sei $B^n_r \subset \R^n$ die abgeschlossene Kugel vom Radius $r$ um $0$. Man zeige:
$$
\lambda_n(B^n_r) = r^n\begin{cases}
	\frac{1}{k!}\pi^k & n = 2k \text{ gerade}, \\
	\frac{2^{k+1}}{1 \cdot 3 \cdot \ldots \cdot (2k+1)}\pi^k & n = 2k+1 \text{ ungerade}.
\end{cases}
$$
\textit{Hinweis}: Der Schnitt einer Hyperebene mit einer Kugel ist ebenfalls eine Kugel. Kombinieren Sie das mit dem Cavalierischen Prinzip-

                        \end{prob}
