\begin{prob}
  Wir wollen rekursive Folgen studieren und in speziellen Fällen eine direkte
  Darstellung angeben.
  \begin{enumerate}[label=(\alph*)]
  \item Sei dazu $(a_n)_{n \in \N}$ eine Folge und $d \in \N$, so dass die
    ersten $d$ Folgenglieder $a_0, \dots, a_{d - 1}$ bereits gegeben sind und es
    $c_1, \dots, c_d \in \R$ gibt, so dass für alle weiteren Folgenglieder mit
    $n \geq d$ gilt
    \begin{equation*}
      a_n
      =
      \sum_{i = 1}^d c_i a_{n - i}.
    \end{equation*}
    Wir betrachten ferner das Polynom
    \begin{equation*}
      p(x)
      \coloneqq
      x^d - \sum_{i = 1}^d c_i x^{d - i}.
    \end{equation*}
    Nach dem \emph{Fundamentalsatz der Algebra} hat dieses Polynom $d$
    Nullstellen $r_1, \dots, r_d \in \C$. Beweise: Gibt es nun weiter
    $k_1, \dots, k_d \in \C$, so dass
    \begin{equation*}
      \fa{i \in \{0, \dots, d - 1\}} a_i = \sum_{j = 1}^d k_j r_j^i
    \end{equation*}
    gilt, dann ist schon für alle $n \in \N$:
    \begin{equation*}
      a_n = \sum_{j = 1}^d k_j r_j^n
    \end{equation*}

  \item Die \emph{Lukaszahlen} sind rekursiv über $f_0 = 2, \ f_1 = 1$ und
    $f_n = f_{n - 1} + f_{n - 2}$ für $n \geq 2$ definiert. Finde nun eine
    explizite Formel für die $n$-te Lukaszahl.

  \item Finde eine explizite Formel für die $n$-te Fibonaccizahl.
  \end{enumerate}
\end{prob}
