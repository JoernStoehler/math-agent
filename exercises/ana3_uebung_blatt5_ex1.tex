        \begin{prob}[Dirac-Folgen]  \\
%
Eine Folge messbarer Funktionen $\varphi_n \colon \R^d \mapsto \R$ hei�t eine Dirac-Folge, falls
\begin{enumerate}[label = (\roman*)]
	\item $\varphi_n \ge 0$,
	\item $\int_{\R^d} \varphi_n \x = 1$,
	\item $\int_{U^c} \varphi_n \x \to 0$ \text{ f�r jede Nullumgebung} $U$.
\end{enumerate}
Wir definieren den Raum der stetigen Testfunktionen als
%
\[ C_c(\R^d) := \left\{ f \mid f \colon \R^d \mapsto \R \text{ ist stetig und hat kompakten Tr�ger} \right\}. \]
%
Sei $\varphi_n$ eine Dirac-Folge. Zeigen Sie:
\begin{enumerate}[label = (\alph*)]
	\item Jede Teilfolge von $\varphi_n$ hat eine Teilteilfolge, die fast �berall gegen Null konvergiert, aber $\varphi_n$ konvergiert nicht in $L^1(\R^d)$;
	\\
	\textit{Hinweis}: Was bedeutet die erste Aussage f�r einen m�glichen $L^1$-Grenzwert? F�hren Sie das mit den �brigen Eigenschaften von $\varphi_n$ zu einem Widerspruch.
	\item F�r $f \in C_c(\R^d)$ gilt $\int_{\R^d} \varphi_n f \x \to f(0)$.
	\item Geben Sie ein Beispiel einer Dirac-Folge an.
\end{enumerate}

%------------------------------------------------------------------------------------------------------------------------
\vspace{2mm}
        \end{prob}
