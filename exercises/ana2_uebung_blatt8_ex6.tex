\begin{prob}
Wir betrachten hier den $\R^2$ in Polarkoordinaten und den $\R^3$ in sph"arischen beziehungsweise zylindrischen Koordinaten. Genauer: Wir definieren Abbildungen $\psi_{pol}:\R_+\times[0,2\pi)\to\R^2\backslash\{0\}$, $\psi_{sph}:\R_+\times(0,\pi)\times[0,2\pi)\to \R^3\backslash\{(x,y,z):x=y=0\}$ und $\psi_{zyl}:\R_+\times[0,2\pi)\times\R\to\R^3\backslash\{(x,y,z):x=y=0\}$ gegeben durch
\begin{equation*}
\begin{split}
\psi_{pol}(r,\varphi):=&
\begin{pmatrix}
r\cos(\varphi)\\ 
r\sin(\varphi)
\end{pmatrix};\\
\psi_{sph}(r,\vartheta,\varphi):=
&\begin{pmatrix} 
r\sin(\vartheta)\cos(\varphi)\\
r\sin(\vartheta)\sin(\varphi)\\
r\cos(\vartheta)
\end{pmatrix};\\
\psi_{zyl}(r,\varphi,z):=&
\begin{pmatrix}
r\cos(\varphi)\\
r\sin(\varphi)\\
z
\end{pmatrix}.
\end{split}
\end{equation*}

Zeige:

(a) Die obigen Abbildungen sind bijektiv und stetig differenzierbar.

(b) Die Ableitungen dieser Abbildungen sind an jedem Punkt invertierbar.

Im Folgenden sei eine Funktion $f:\R^2\setminus\{0\}\to\R$ gegeben als $f(r,\phi)$
in Polarkoordinaten. Zeige:

(c)
$$
\Delta f = \frac{\p^2f}{\p r^2}+\frac{1}{r}\frac{\p f}{\p
  r} + \frac{1}{r^2}\frac{\p^2f}{\p\phi^2}.
$$
(d) Jede Linearkombination der Funktionen $r^n\cos(n\phi)$,
$r^n\sin(n\phi)$ mit
$n\in\Z$ l\"ost die Laplace-Gleichung $\Delta f=0$.
\end{prob}
