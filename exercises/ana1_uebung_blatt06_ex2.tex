\begin{exercise}[Darstellung von reeellen Zahlen]
  Sei $x \in \R_+$ eine positive reelle Zahl. Sei weiter die Folge
  $(m_n)_{n \geq k}$ eine Dezimaldarstellung der Zahl $x$. Das heißt für eine
  ganze Zahl $k \in \Z$ gilt:
  \[
    x = \sum_{i = k}^\infty \frac{m_i}{10^i}
  \]
  Definieren wir weiter die Folge $(a_n)_{n \geq k}$ durch
  \[
    a_n \coloneqq \sum_{i = k}^n \frac{m_i}{10^i}.
  \]
  Zeige, dass mit dieser Setzung gilt:
  \[
    \{x\}
    =
    \bigcap_{i = k}^\infty \left[ a_i - 10^{-i}, a_i + 10^{-i} \right]
  \]
\end{exercise}
