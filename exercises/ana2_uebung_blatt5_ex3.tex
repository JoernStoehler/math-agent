\begin{prob}
  (a) Gegeben sei ein metrischer Raum $X$. Zeige: Wenn $X$ kompakt ist, dann ist $X$ vollst"andig.

(b) Wir definieren jetzt den {\em Durchmesser} einer Teilmenge $A\subset X$ durch
$$
   {\rm diam}(A) := \sup\{d(x,y)\mid x,y\in A\}.
$$
Zeige: $X$ ist genau dann vollst\"andig, wenn das
{\em Schachtelungsprinzip} gilt: Ist $A_1\supset A_2\supset
A_3\supset\dots$ eine Schachtelung nichtleerer abgeschlossener
Teilmengen mit $\lim_{k\to\infty}{\rm diam}(A_k)=0$, so gibt es genau
einen Punkt $x\in X$, der in allen $A_k$ liegt.

(c) Ist $A_1\supset A_2\supset\dots$ eine Schachtelung nichtleerer
kompakter Teilmengen von $X$, so ist $\bigcap_{n=1}^\infty A_n$ nichtleer und kompakt. 
\end{prob}
