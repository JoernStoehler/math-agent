\begin{prob}
  Es sei $a_{ij}$ die Zahl in der $i$-ten Zeile und $j$-en Spalte des Schemas
  \begin{equation*}
    \begin{matrix}
      -1          & 0           & 0           & 0      & \cdots \\[3pt]
      \frac{1}{2} & -1          & 0           & 0      & \cdots \\[3pt]
      \frac{1}{4} & \frac{1}{2} & -1          & 0      & \cdots \\[3pt]
      \frac{1}{8} & \frac{1}{4} & \frac{1}{2} & -1     & \cdots \\
      \vdots      & \vdots      & \vdots      & \vdots & \ddots
    \end{matrix}
  \end{equation*}
  Anders ausgedrück betrachten wir die Abbildung $a \colon \N \times \N \to \R$
  mit:
  \begin{align*}
    a_{ij}
    =
    \begin{cases}
      -1                 & i = j \\
      0                  & j > i \\
     \frac{1}{2^{i - j}} & i > j \\
    \end{cases}
  \end{align*}
  Berechne $\sum_{i=1}^\infty (\sum_{j=1}^{\infty} a_{ij})$ und
  $\sum_{j=1}^{\infty} (\sum_{i=1}^\infty a_{ij})$. Wie verträgt sich das
  Ergebnis mit dem Doppelreihensatz?
\end{prob}
