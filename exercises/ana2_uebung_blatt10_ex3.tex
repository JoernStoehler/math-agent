\begin{prob}
Es sei $p:\C\to\C$ ein komplexes Polynom von Grad $n>0$, also von der Form $p(z):=a_nz^n+a_{n-1}z^{n-1}+\cdots+a_1z+a_0$, wobei $a_k\in\C$ und $a_n\neq 0$. Zeige:
\begin{enumerate}[label=(\alph*)]
\item es gibt ein $R>0$, so dass $|p(z)|>|p(0)|$ f"ur alle $z\in \C\backslash B(0,R)$;
\item die Funktion $f(z):=|p(z)|^2$ ist stetig differenzierbar;
\item $f$ nimmt sein Minimum an einem $z_0\in B(0,R)$ an;
\item $f(z_0)=0$ ({\em Tipp: Taylorentwicklung um $z_0$});
\item den \emph{Fundamentalsatz der Algebra:} $p$ hat mindestens eine Nullstelle in $\C$.
\end{enumerate}
\end{prob}
