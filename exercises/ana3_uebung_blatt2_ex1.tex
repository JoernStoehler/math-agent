            \begin{prob}[Smith-Volterra-Cantormengen]
%
Sei $0 < \alpha \le \tfrac{1}{3}$. Wir konstruieren die {\em Smith-Volterra-Cantormenge} $C_\alpha$ wie folgt: Aus der Mitte des Intervalls $\left[ 0, 1 \right]$ entnehmen wir ein Intervall der L�nge $\alpha$, sodass wir die Intervalle
%
\[ \left[0, \tfrac{1}{2} - \tfrac{\alpha}{2} \right] \cup \left[ \tfrac{\alpha}{2} + \tfrac{1}{2} , 1 \right] \]
%
erhalten. Aus der Mitte jedes dieser Intervalle entnehmen wir wiederum Intervalle der L�nge $\alpha^2$, sodass wir die Intervalle 
%
\[ \left[ 0 , \tfrac{1}{4} - \tfrac{\alpha}{4} - \tfrac{\alpha^2}{2} \right] \cup \left[ \tfrac{1}{4} - \tfrac{\alpha}{4} + \tfrac{\alpha^2}{2}, \tfrac{1}{2} - \tfrac{\alpha}{2} \right] \cup \left[ \tfrac{\alpha}{2} + \tfrac{1}{2} , \tfrac{\alpha}{4} + \tfrac{3}{4} - \tfrac{\alpha^2}{2} \right] \cup \left[ \tfrac{\alpha}{4} + \tfrac{3}{4} + \tfrac{\alpha^2}{2} , 1 \right] \]
%
erhalten und immer so weiter. Formal hei�t das
%
\[ E_0 := \left[ 0 , 1 \right], \quad E_n := \bigcup_{k = 0}^{2^{n - 1} - 1} \left( \left[ a_k , \frac{a_k + b_k}{2} - \frac{\alpha^n}{2} \right] \cup \left[ \frac{\alpha^n}{2} + \frac{a_k + b_k}{2} , b_k \right] \right), \]
%
wobei die $0 = a_1 < b_1 < a_2 < b_2 < \cdots < a_{2^{n - 1} } < b_{2^{n - 1} } = 1$ durch
%
\[ E_{n - 1} = \bigcup_{k = 1}^{2^n - 1} \left[ a_k , b_k \right]\]
%
eindeutig gegeben sind. Nun
%
\[ C_\alpha :=  \bigcap_{n = 1}^\i E_n. \]
%
\begin{enumerate}[label = (\alph*)]
	\item Zeigen Sie: $C_\alpha \in \mathcal{B}(\R)$ und $C_\alpha$ hat leeres Inneres.
	\item Berechnen Sie das Lebesgue-Ma� von $C_\alpha$.
\end{enumerate}
%------------------------------------------------------------------------------------------------------------------------
\newpage
%------------------------------------------------------------------------------------------------------------------------
\vspace{2mm}
            \end{prob}
