\begin{exercise}[Folgerungen aus Stetigkeit]
  Man zeige die folgenden Aussagen:
  \begin{enumerate}[label=(\alph*)]
  \item Sei f eine stetige Funktion und $f(x_0) > 0$, dann existiert ein
    $\eps > 0$, sodass $f(x) > 0$ für $x \in \interval{x_0 - \eps}{x_0 + \eps}$.
    Man mache sich klar, warum die enstprechende Aussage dann auch für $< 0$
    folgt.

  \item Sei f eine stetige Funktion mit
    $f \colon \interval{a}{b} \to \interval{c}{d}$ mit
    $\interval{c}{d} \subsetneq \interval{a}{b}$, so existiert ein Punkt
    $x_0 \in \interval{a}{b}$, sodass $f(x_0)=x_0$.

  \item Seien f und g zwei stetige Funktionen mit $f(a) \leq g(a)$ und
    $g(b) \leq f(b)$ mit $a, b \in \R$ , dann existiert ein $x_0$ mit
    $f(x_0) = g(x_0)$.
  \end{enumerate}
\end{exercise}
