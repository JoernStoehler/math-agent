        \end{prob}
            \begin{prob}[Lebesgue-R�ume]
Sei $\left( \Omega , \mathcal{A} , \mu \right)$ ein Ma�raum. F�r $1 < p < \i$ und $u \colon \Omega \left[ - \i , \i \right]$ messbar definieren wir die $L^p(\mu)$-Norm von $u$ durch
$$
\| u \|_p := \left( \int_\Omega | u |^p \, d \mu \right)^{1 / p}.
$$
Die Endlichkeitsmenge von $\| \cdot \|_{L^p(\mu)}$ notieren wir als
$$
\mathcal{L}^p(\mu) = \left\{ u \in \mathcal{L}^0(\Omega) \, \middle| \, \| u \|_{L^p(\mu) } < \i \right\}.
$$
Man nennt die Elemente von $\mathcal{L}^p(\mu)$ die im $p$-ten Mittel integrierbaren Funktionen. Wie im Fall $p = 1$ notieren wir mit $L^p(\mu)$ den Raum der �quivalenzklassen von Funktionen in $\mathcal{L}^p(\mu)$, die f.�. �bereinstimmen. Zeigen Sie: 
\begin{enumerate}[label = (\alph*)]
	\item F�r $u \in \mathcal{L}^p(\mu)$ und $v \in \mathcal{L}^q(\mu)$ mit $\frac{1}{p} + \frac{1}{q} = 1$ gilt die \textit{H�ldersche Ungleichung}
	$$
	\| u v \|_1 \le \| u \|_p \| v \|_q.
	$$
	Insbesondere folgt aus $u \in \mathcal{L}^p(\mu)$ und $v \in \mathcal{L}^q(\mu)$, dass $uv \in \mathcal{L}^1(\mu)$.
	\\
	\textit{Hinweis}: Beweisen Sie zun�chst die elementare Youngsche Ungleichung
	$$a b \le \frac{a^p}{p} + \frac{b^q}{q} \text{ f�r } a, b \in \left[ 0 , \i \right],$$ indem Sie die Konkavit�t des Logarithmus nutzen. �berlegen Sie sich dann, warum es nicht einschr�nkt $\| u \|_p = \| v \|_q = 1$ anzunehmen, und folgern Sie mittels der Youngschen Ungleichung.
	\item Der Raum $L^p(\mu)$ ist normiert.
	\\
	\textit{Hinweis}: Betrachten Sie f�r die Dreiecksungleichung die $p$-te Potenz $\| u + v \|_p^p$ und faktorisieren Sie deren Integranden zu $| u + v | | u + v |^{p - 1}$, um die H�ldersche Ungleichung anzuwenden. %Es hilft zu bemerken, dass $q = \frac{p}{p - 1}$.
%	\item Der Raum $L^p(\mu)$ ist vollst�ndig. Jede in $L^p(\mu)$ konvergente Folge enth�lt eine Teilfolge, die f.�. gegen ihren Grenzwert konvergiert.
	%\\
	%\textit{Hinweis}: Studieren Sie im Skript den Fall $p = 1$.
	
\end{enumerate}

\vspace{2mm}
            \end{prob}
