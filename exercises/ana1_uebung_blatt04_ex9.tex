\begin{exercise}[Achilles und die Schildkröte]
  Zenon von Elea, Philosoph aus Elea, ließ Achilles, den schnellsten Läufer des
  Altertums (10 Sekunden auf 100 Meter), in Gedanken gegen eine Schildkröte zu
  einem Wettlauf antreten. Da Achilles hundertmal so schnell wie die Schildkröte
  lief, wurde dieser ein Vorsprung von 100 Metern gewährt. Beide liefen nach dem
  Startschuss gleichzeitig los. Trotz seiner Schnelligkeit konnte Achilles die
  Schildkröte merkwürdigerweise nicht erreichen: Denn als er 100 Meter
  zurückgelegt hatte, war auch die Schildkröte ein Stück weitergelaufen; bei
  Erreichen dieses Punktes war die Schildkröte schon wieder ein Stück voraus,
  ...\\
  Löse dieses Paradoxon des Zenon von Elea durch Argumentation mit unendlichen
  Reihen auf und überprüfe das Resultat mit physikalischen Argumenten über die
  Geschwindigkeit (insbesondere den Zeitpunkt des Erreichens).
\end{exercise}
