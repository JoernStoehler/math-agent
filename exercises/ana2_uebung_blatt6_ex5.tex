\begin{prob} 
Sei $f:[a,b]\to\R^n$ eine stetige Kurve. Zeige:

(a) Ist $f$ rektifizierbar, so sind auch f\"ur jedes $c\in(a,b)$ sind
die Restriktionen $f|_{[a,c]}$ und $f|_{[c,b]}$ rektifizierbar und es gilt
$$
   L(f) = L(f|_{[a,c]}) + L(f|_{[c,b]}).
$$
(b) Ist $f$ rektifizierbar, so gilt
\begin{equation}\label{eq:sup}
   L(f) = \sup\{P_f(t_0,\dots,t_k)\mid a=t_0<t_1<\cdots<t_k=b \text{
     Unterteilung von }[a,b]\}.
\end{equation}
(c) Ist umgekehrt das Supremum auf der rechten Seite
von~\eqref{eq:sup} endlich, so ist $f$ rektifizierbar und die
Formel~\eqref{eq:sup} gilt.    
\end{prob}
