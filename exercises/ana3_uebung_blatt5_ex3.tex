                \begin{prob}[Dynkin-Systeme und ihre Eigenschaften]  \\
%
Sei $\Omega$ eine Menge. Eine Teilmenge $\mathcal{D} \subset \mathcal{P}(\Omega)$ hei�t \textit{Dynkin-System}, wenn gilt:
\begin{enumerate}[label = (\roman*)]
	\item $\Omega \in \mathcal{D}$;
	\item $D, E \in \mathcal{D}$ mit $D \subset E \implies E \setminus D \in \mathcal{D}$;
	\item $D_n \in \mathcal{D}$, $n \in \N$ disjunkt $\implies \bigcup_{n \in \N} D_n \in \mathcal{D}$.
\end{enumerate}
F�r $S \subset \mathcal{P}(\Omega)$ sei
$$
\delta(S) := \bigcap \left\{ \mathcal{D} \mid \mathcal{D} \subset \mathcal{P}(\Omega) \text{
	Dynkin-System mit } \mathcal{D} \supset S \right\}
$$
das von $S$ erzeugte Dynkin-System. Zeigen Sie: 
\begin{enumerate}[label = (\alph*)]
	\item F\"ur $\mathcal{A}\subset\mathcal{P}(\Omega)$ gilt: $\mathcal{A}$ $\sigma$-Algebra
	$\iff$ $\mathcal{A}$ Dynkin-System und $\cap$-stabil.
	\item $S \subset \mathcal{P}(\Omega)$ $\cap$-stabil $\implies$
	$\delta(S) = \sigma(S)$. 
\end{enumerate}
%------------------------------------------------------------------------------------------------------------------------
\newpage
%------------------------------------------------------------------------------------------------------------------------
\vspace{2mm}
                \end{prob}
