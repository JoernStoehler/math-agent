\begin{prob}
  Finde den Fehler im folgenden \enquote{Beweis} per Induktion: \\
  \textit{Falscher Satz:} Alle Katzen haben dieselbe Farbe.

  \textit{Falscher Beweis:} Wir zeigen per Induktion, dass jede Menge von $n$
  Katzen nur Katzen derselben Farbe enthält. Da die Menge der Katzen endlich
  ist, erreichen wir so irgendwann ein $n_0 \in \N$, das so groß ist wie die
  Menge aller Katzen, und wenden dann unseren Satz auf die Menge aller Katzen
  an.\\
  Wir beginnen mit dem Induktionsanfang, also dem Fall einer einzigen Katze:
  Diese hat dieselbe Farbe wie sie selbst. Jede Menge, die nur eine Katze
  enthält, enthält also nur Katzen derselben Farbe. \\
  Die Induktionsvoraussetzung ist jetzt also, dass jede Menge von $n$ Katzen nur
  Katzen der gleichen Farbe enthält. Jetzt kommt der Induktionsschritt zu jeder
  Menge von $n + 1$ Katzen. Wir trennen dazu zuerst eine Katze vom Rest, es
  bleibt also eine Menge von $n$ Katzen übrig. Diese haben nach Voraussetzung
  alle dieselbe Farbe. Wir kennen die Farbe der einzelnen Katze aber noch nicht.
  Um diese herauszufinden, vereinen wir wieder all unsere Katzen und trennen
  dann eine andere Katze davon ab als bei der ersten Trennung. Die verbleibenden
  $n$ Katzen inklusive der einen, die wir vorher abgetrennt hatten, haben also
  dieselbe Farbe. Also müssen alle Mengen von $n + 1$ Katzen nur Katzen
  derselben Farbe beinhalten.
\end{prob}
