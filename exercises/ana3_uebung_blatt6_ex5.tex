\end{prob}

\vfill
Bitte geben Sie Ihre L"osungen bis sp"atestens \textbf{n�chsten Donnerstag, 20:00 Uhr} bei Ihrem Tutor ab. 
\end{document}

%------------------------------------------------------------------------------------------------------------------------
\begin{prob}[Aufgabe 1:]{\bf Integrale summieren Ma�e der Superniveaumengen} \vspace{.2cm} \\
%
Sei $\left( \Omega , \mathcal{A} , \mu \right)$ ein Ma�raum und $u \colon \Omega \mapsto \left[ 0 , \i \right]$ eine messbare Funktion. Dann
%
\[ \int u \, d \mu = \int_0^\i \mu \left( \left\{ u > x \right\} \right) \x. \]
%
\textit{Hinweis}: Zeigen Sie zun�chst, dass beide Seiten der Gleichung nur endlich sein k�nnen, falls $\left\{ u > 0 \right\}$ eine $\sigma$-endliche Menge ist. Eine Menge $A \in \mathcal{A}$ hei�t $\sigma$-endlich, falls sie durch eine Folge von Mengen endlichen Ma�es ausgesch�pft werden kann.
%Eine Menge $A \in \mathcal{A}$ hei�t $\sigma$-endlich bez�glich $\mu$, wenn
%%
%\[ \exists A_n \in \mathcal{A} \colon \mu(A_n) < \i \land \bigcup_{n \ge 1 } A_n = A. \]
%
% Fonseca/Leoni, Modern Methods in the Calculus of Variations: Lp Spaces, Thm. 1.123

Sei $r > 0$ und
%
\[ M = \left\{ \left( x , y \right) \in \R^2 \mid x \le 0 \le x + y, x^2 + y^2 \le r^2 \right\}. \]
%
Berechnen Sie $\int_M y \, d \lambda_2$.
%------------------------------------------------------------------------------------------------------------------------
\vspace{2mm}
    \begin{prob}[Ein Integral]
