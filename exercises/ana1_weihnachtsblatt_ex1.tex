\begin{prob}[Zenons Paradox]
  \textit{Diese Aufgabe geht auf den griechischen Philosophen \textsc{Zenon von
      Elea} zurück, der meinte, Zeit sei eigentlich unmöglich und daher eine
    Illusion. Dies versuchte er anhand mehrerer Paradoxien zu zeigen, wobei
    diese die bekannteste davon ist. Aus heutiger Perspektive ist es wohl eher
    so, dass
    Zenon noch keine Analysis kannte...}\\
  Achilles, der schnellste Mann Griechenlands, wurde von einer Schildkröte zum
  Wettlauf über 100 Meter herausgefordert. Als echter Ehrenmann gewährt Achilles
  der langsameren Schildkröte 90 Meter Vorsprung. Achilles ist 100 mal schneller
  als die Schildkröte. Doch laut Zenon wird Achilles die Schildkröte nie
  erreichen. Denn sobald Achilles die 90 Meter, die die Schildkröte Vorsprung
  hatte, zurückgelegt hat, ist die Schildkröte schon ein Stückchen weiter
  gelaufen. Bis Achilles diese neue Strecke hinter sich gebracht hat, ist die
  Schildkröte schon wieder ein wenig weiter. Egal wie oft Achilles die noch
  fehlende Distanz überwunden hat, die Schildkröte ist ihm immer noch voraus.
  Also wird Achilles die Schildkröte nie erreichen. Jetzt bist du gefragt: Wo
  liegt Zenons Denkfehler? Bei welcher Strecke wird Achilles die Schildkröte
  überholen? Und wer gewinnt das Wettrennen über die 100 Meter?
\end{prob}
