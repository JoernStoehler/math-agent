\begin{exercise}[Zwei Verallgemeinerungen der Produktregel]
  Die Leibnitzregel ist manchmal praktisch wenn man schnell und allgemein oft
  hintereinander ein Produkt von zwei Funktionen ableiten muss. Die
  Logarithmische Ableitung ist für den Fall, dass man mehr als zwei Faktoren hat
  sehr schnell und effizient.
  \begin{enumerate}[label=(\alph*)]
  \item \emph{Die Leibnitzregel} besagt, dass
    \begin{equation*}
      \begin{split}
        \left(u(x) \cdot v(x)\right)^{(n)} = \sum_{k=0}^{n} { \binom{n}{k}
          f^{(k)}(x) g^{(n-k)}(x) }
      \end{split}
    \end{equation*}
    gilt, wobei wir mit $f^{(i)}$ die $i$-te Ableitung einer Funktion $f$
    bezeichnen. Zeige sie mit Hilfe der vollständigen Induktion.
  \item Mit der \emph{logarithmischen Ableitung} einer Funktion
    $f \colon \R \to \R \setminus \set{0}$ bezeichnen wir den Quotienten
    \begin{equation*}
      \begin{split}
        f^{(l)}(x) \eqdef \frac{f^\prime(x)}{f(x)}
      \end{split}
    \end{equation*}
    Es seien nun $f_i\colon \R \to \R \setminus \set{0}$ für
    $i \in \set{1, \cdots, n}$ differenzierbare Funktionen und
    $f(x) = f_1(x) \cdot f_2(x) \cdot \cdots \cdot f_n(x)$ das Produkt aller
    $f_i$. Zeige dann die sehr schöne Aussage
    \begin{equation*}
      \begin{split}
        f^{(l)} = \sum\limits_{ i=1 }^{ n }{ f_i^{(l)} }
      \end{split}
    \end{equation*}
    und folgere eine Formel für die Berechnung der Ableitung eines Produktes von
    mehr als zwei Funktionen. Überlege noch, ob die Vorraussetzung, dass die
    $f_i$ nicht null werden für die Formel am Ende wirklich nötig war.
  \end{enumerate}
\end{exercise}
