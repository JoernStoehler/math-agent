        \begin{prob}[Glatte Bilder von Nullmengen]
%
Man zeige:
\begin{enumerate}[label = (\alph*)]
	\item Ist $\Phi \colon U \mapsto V$ ein Diffeomorphismus zwischen offenen
	Mengen $U,V\subset\R^n$ und $N\subset U$ eine Lebesgue-Nullmenge, so
	ist auch ihr Bild $\Phi(N)$ eine Nullmenge. \label{it. 1}
	\item Ist $M \subset \R^n$ eine $k$-dimensionale
	Untermannigfaltigkeit und $\{ \phi_i \colon \R^k \supset V_i \mapsto U_i \cap
	M\}_{i\in\N}$ ein abz\"ahlbarer Atlas, so ist $N\subset M$ genau dann eine $k$-dimensionale Nullmenge, wenn $\phi_i^{-1}(U_i\cap N)\subset V_i\subset\R^k$ f\"ur jedes $i\in\N$ eine Nullmenge ist.
\end{enumerate}

\textit{Hinweis zu \ref{it. 1}: Nutzen Sie den Transformationssatz f�r den Indikator der Nullmenge.}

%------------------------------------------------------------------------------------------------------------------------
\vspace{2mm}
        \end{prob}
