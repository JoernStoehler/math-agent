\begin{exercise}[Modifizierte harmonische Reihe]
  Es ist ein wohlbekanntes Ergebnis der Analysis, dass die harmonische Reihe
  \[
    \sum_{n = 0}^\infty \frac{1}{n}
  \]
  gegen $\infty$ divergiert. Sei $d \colon \N \to \mathcal{P}(\{1, \dots, 9\})$
  die Funktion, die einer natürlichen Zahl die Menge ihrer Ziffern in der
  Dezimaldarstellung zuordnet. Für eine natürliche Zahl $n \in \N$ und ihre
  Dezimaldarstellung $(m_0, m_1, \dots, m_k)$ (das heißt $m_k \neq 0$ und
  $n = \sum_{i = 0}^k m_i 10^i$) ist also $d(n) = \{m_0, m_1, \dots, m_k\}$.
  Definieren wir nun die Menge aller natürlichen Zahlen, die in der
  Dezimaldarstellung die Ziffer $7$ nicht enthalten. Formal betrachten wir also
  die Menge
  \[
    M \coloneqq \Set{n \in \N | 7 \notin d(n)}.
  \]
  Zeige jetzt, dass die modifizierte harmonische Reihe
  \[
    \sum_{n \in M} \frac{1}{n}
  \]
  konvergiert.
\end{exercise}
