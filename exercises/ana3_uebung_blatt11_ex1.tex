        \begin{prob}[De Rham-Kohomologie aller Grade]
F\"ur eine offene Menge $U\subset\R^n$ definieren wir die de Rham-Kohomologie aller Grade
$$
H_{dR}(U) := \bigoplus_{k=0}^nH_{dR}^k(U). 
$$
Zeigen Sie:
\begin{enumerate}[label = (\alph*)]
	\item Das Wedge-Produkt induziert via
	$$
	[\alpha]\wedge[\beta] := [\alpha\wedge\beta]
	$$
	auf $H_{dR}(U)$ die Struktur einer graduiert kommutativen $\R$-Algebra. Dabei ist $[ \alpha ]$ die �quivalenzklasse einer geschlossenen Form $\alpha \in \Omega(U)$ in $H_{dR}(U)$.
	\item Eine glatte Abbildung $f \colon U \mapsto V$ zwischen offenen Mengen $U, V \subset \R^n$ induziert einen graderhaltenden Algebren-Homomorphismus $f^* \colon H_{dR}(V) \mapsto H_{dR}(U)$; dieser ist ein Isomorphismus, wenn $f$ ein Diffeomorphismus ist. 
	\item Es gibt keinen Diffeomorphismus zwischen $\R^2$ und $\R^2 \setminus \{ 0 \}$.
\end{enumerate}

%------------------------------------------------------------------------------------------------------------------------
\vspace{2mm}
        \end{prob}
