\begin{prob}[Charakterisierung eines Pr\"ama\ss es]\label{ueb:Praemass}
Es sei $\mu:\AA\to\ol\R_+$ ein Inhalt. Dann sind \"aquivalent:

(a) $\mu$ ist $\sigma$-additiv (d.h.~ein Pr\"ama\ss);

(b) $\mu(A)=\lim_{n\to\infty}\mu(A_n)$ f\"ur jede Folge $A_n\in\AA$
mit $A_n\nearrow A\in\AA$;

(c) $\mu(A)\leq \sum_{i=1}^\infty\mu(A_i)$ f\"ur alle $A_i,A\in\AA$
mit $A\subset\bigcup_{i=1}^\infty A_i$.

Beweisen Sie außerdem: $(a) - (c)$ implizieren

(d) $\mu(A)=\lim_{n\to\infty}\mu(A_n)$ f\"ur jede Folge $A_n\in\AA$
mit $A_n\searrow A\in\AA$ und $\mu(A_1)<\infty$;

Falls $\mu$ endlich ist, d.h.~$\mu(\Om) < \infty$, so ist $(d)$ sogar \"aquivalent zu $(a) - (c)$.

F\"allt Ihnen ein Beispiel ein, das zeigt, dass auf die Bedingung $\mu(\Omega) < \infty$ für die G\"ultigkeit der R\"uckrichtung $(d) \implies (a) - (c)$ im Allgemeinen nicht verzichtet werden kann?
\end{prob}
