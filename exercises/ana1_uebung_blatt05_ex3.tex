\begin{exercise}[Eulersche Konstante]
  Wir sehen nun eine schöne Anwendung von \Cref{ex:squeeze-lemma}. Setzen wir
  dazu voraus, dass
  \[
    2
    <
    \lim_{n \to \infty} \left( 1 + \frac{1}{n} \right)^n
    =
    e
    <
    3
  \]
  gilt. Zeige davon ausgehend:
  \begin{enumerate}
  \item
    $\displaystyle \lim_{n \to \infty} \left( 1 + \frac{1}{n^2} \right)^n = 1$
  \item
    $\displaystyle \lim_{n \to \infty} \left( 1 + \frac{1}{\sqrt{n}} \right)^n =
    \infty$
  \end{enumerate}
\end{exercise}
