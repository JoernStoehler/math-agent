\begin{exercise}[Gleichmäßige Stetigkeit]
  Man zeige die folgenden Aussagen:
  \begin{enumerate}[label=(\alph*)]
  \item Seien $f$ und $g$ zwei gleichmäßig stetige Funktionen, dann folgt, dass
	  auch $f(g(x))$, $f(x) + g(x)$ und gleichmäßig stetig sind. Was ist mit $f(x) \cdot g(x)$ und wie verändert sich das sollten $f$ und $g$ zusätzlich beschränkt sein?
  \item Prüfe $\frac{1}{\sqrt{x}} \colon \R \setminus \Set{0} \to \R$ auf
    gleichmäßige Stetigkeit.
  \end{enumerate}
\end{exercise}
