\begin{exercise}[Trinominos]
  Ein \emph{Trinomino} ist wie ein Dominostein, nur aus drei Quadraten aufgebaut
  und von der Form \includestandalone[scale=0.15]{graphics/trinominos/trinom1}.
  Trinominos dürfen beliebig gedreht werden, und sollen im Folgenden benützt
  werden, um \enquote{Schachbretter} lückenlos zu pflastern. Dabei stimmt die
  Größe der Quadrate aus denen die Schachbretter aufgebaut sind überein mit der
  Größe der Quadrate aus denen die Trinominos gemacht sind. Ein Trinomino
  bedeckt so immer genau drei Quadrate.\\
  Zu zeigen ist nun, dass es immer (für alle $n \in \N$) möglich ist, folgende
  \enquote{Schachbretter} vollständig ohne Überlappen mit Trinominos zu
  pflastern:
  \begin{itemize}
  \item $\downarrow$ Ein $2^n \times 2^n$ Schachbrett dem die rechte untere
    Ecke fehlt
    \begin{figure}[!h]
      \centering
      \begin{minipage}[b]{0.38\textwidth}
        \includestandalone[width=\textwidth]{graphics/trinominos/trinom3}
      \end{minipage}
      \hspace{1cm}
      \begin{minipage}[b]{0.38\textwidth}
        \includestandalone[width=\textwidth]{graphics/trinominos/trinom4}
      \end{minipage}
    \end{figure}
  \item Ein $2^n \times 2^n$ Schachbrett dessen rechtes unteres Viertel fehlt
    $\uparrow$
  \end{itemize}
\end{exercise}
