\begin{exercise}
  Eine weitere Definition für Stetigkeit einer Funktion $f \colon U \to \R$ in
  dem Punkt $a$ ist die Folgende
  \begin{equation*}
    \fa{\eps > 0} \ex{\delta > 0} \fa{x \in U} \abs{x - a} < \delta \implies
    \abs{f(x) - f(a)} < \eps
  \end{equation*}
  Zeige, dass diese Definition von Stetigkeit schon die Definition von
  Stetigkeit aus der Vorlesung impliziert. Also dass für alle Folgen
  $(x_n)_{n \in \N}$ mit Grenzwert $a$ gilt
  \begin{equation*}
    \lim_{n \to \infty} f(x_n)
    =
    f(a).
  \end{equation*}

\end{exercise}
